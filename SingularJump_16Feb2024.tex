\documentclass[11pt,reqno]{amsart}
\usepackage[utf8]{inputenc}
\usepackage{latexsym,amssymb,longtable}
\usepackage{hyperref,enumerate}
\usepackage[eulergreek]{sansmath}

\usepackage{amsmath,amsthm,amsxtra}
\usepackage{amsfonts,eucal,mathrsfs,todonotes}
\usepackage{enumitem}
\usepackage{fullpage}
\usepackage{xcolor}
% \usepackage{stix}
\usepackage{upgreek}
\usepackage[notref,notcite]{showkeys}

\numberwithin{equation}{section}


\definecolor{ddorange}{rgb}{1,0.5,0}
\definecolor{ddcyan}{rgb}{0,0.2,1.0}




\newcommand{\N}{\mathbb{N}}
\newcommand{\CE}[2]{\calC \calE([#1,#2])}
\newcommand{\ENHCE}[2]{\calC \calE^{\mathrm{enh}}([#1,#2])}
\newcommand{\R}{\mathbb{R}}

\newcommand{\calA}{\mathcal{A}}
\newcommand{\calB}{\mathcal{B}}
\newcommand{\calC}{\mathcal{C}}
\newcommand{\calD}{\mathcal{D}}
\newcommand{\calE}{\mathcal{E}}
\newcommand{\calF}{\mathcal{F}}
\newcommand{\calG}{\mathcal{G}}
\newcommand{\calH}{\mathcal{H}}
\newcommand{\calI}{\mathcal{I}}
\newcommand{\calJ}{\mathcal{J}}
\newcommand{\calK}{\mathcal{K}}
\newcommand{\calL}{\mathcal{L}}
\newcommand{\calM}{\mathcal{M}}
\newcommand{\calP}{\mathcal{P}}
\newcommand{\calR}{\mathcal{R}}
\newcommand{\calV}{\mathcal{V}}
\newcommand{\calW}{\mathcal{W}}
\newcommand{\scrI}{\mathscr{I}}
\newcommand{\scrL}{\mathscr{L}}
\newcommand{\Lip}{\mathrm{Lip}}
\newcommand{\Lipb}{\mathrm{Lip}_{\mathrm{b}}}
\newcommand{\sign}{\text{sign}\,}
\newcommand{\ssubset}{\subset\joinrel\subset}

\newcommand{\p}{\mathfrak{p}}
\newcommand{\q}{\mathfrak{q}}

\newcommand{\eps}{\varepsilon}

\DeclareMathOperator{\id}{id}
\DeclareMathOperator{\I}{I}
\DeclareMathOperator{\Tr}{Tr}
\newcommand{\dnabla}{\overline\nabla}
\newcommand{\supp}{\text{supp}\,}
\renewcommand{\t}{\textbf{t}}
\newcommand{\s}{\textbf{s}}
\renewcommand{\r}{\textbf{r}}

\newcommand{\dd}{\mathrm{d}}
\newcommand{\pr}{\text{pr}}
%\newcommand{\teta}{\vartheta}
\newcommand{\x}{{\rm\bf x}}
\newcommand{\y}{{\rm\bf y}}
\newcommand{\z}{{\rm\bf z}}
\newcommand{\bxi}{\boldsymbol{\xi}}
\newcommand{\bta}{\boldsymbol{\eta}}
\newcommand{\bzeta}{\boldsymbol{\zeta}}
\newcommand{\bLambda}{\boldsymbol{\Lambda}}

\newcommand{\OLI}[1]{{\color{blue}#1}}
\newcommand{\alert}{\color{red}}

\newtheorem{theorem}{Theorem}[section]
\newtheorem{prop}[theorem]{Proposition}
\newtheorem{problem}[theorem]{Problem}
\newtheorem{cor}[theorem]{Corollary}
\newtheorem{corollary}[theorem]{Corollary}
\newtheorem{lemma}[theorem]{Lemma}
\newtheorem{proposition}[theorem]{Proposition}

\theoremstyle{definition}
\newtheorem{definition}[theorem]{Definition}
\newtheorem{remark}[theorem]{Remark}
\newtheorem{assumption}[theorem]{Assumption}
\newtheorem{example}[theorem]{Example}
 %\newcommand{\upphi}{\upshape \Phi}
 %\newcommand{\uppsi}{\upshape \Psi}
 %\newcommand{\upalpha}{\upshape \alpha}

%%% MACROS from PRST 
%%%%%%%%%%%%%%%%%%%% Specific notations %%%%%%%%%

\newcommand{\QED}{\mbox{}\hfill\rule{5pt}{5pt}\medskip\par}
\newcommand{\ep}{\varepsilon}
\let\e\ep
\let\eps\ep
\newcommand{\C}{\mathbb{C}}
\newcommand{\Bteta}{{\boldsymbol\vartheta}}
\renewcommand{\vec}[1]{\boldsymbol{#1}}
\newcommand{\foraa}{\text{for a.e.\ }}
\def\One{\mathbbm{1}}
\def\dd{\mathrm{d}}
\DeclareMathSymbol{\mtimes}{\mathord}{symbols}{"0A}
\newcommand{\argmin}{\mathop{\rm argmin}}
\newcommand{\dom}{\mathop{\rm dom}}
\newcommand{\edg}{E}
\newcommand{\edgY}{Y {\times} Y}
\newcommand{\CB}{\mathrm{C}_{\mathrm{b}}}
\newcommand{\ona}{\dnabla}
\newcommand{\odiv}[1]{\dnabla \cdot(#1)}
\renewcommand{\odiv}[1]{\mathop{\overline{\mathrm{div}}}#1}
\newcommand{\odivn}{\overline{\mathrm{div}}}
\newcommand{\DVT}[3]{\mathscr{W}(#1,#2,#3)}
\newcommand{\CER}[2]{\mathcal{A}{(#1,#2)}}
\newcommand{\CEIR}[1]{\mathcal{A}{(#1)}}
\newcommand{\ADM}[4]{\mathscr{A}{(#1,#2;#3,#4)}}
\newcommand{\DVTn}{\mathscr{W}}
\newcommand{\VarW}[3]{\mathbb W(#1;[#2,#3])}
\newcommand{\VarWn}{\mathbb W}
\newcommand{\Vk}[2]{\mathsf{V}_{#1}(#2)}
\newcommand{\Vkn}[1]{\mathsf{V}_{#1}}
\newcommand{\Mk}[2]{\mathsf{M}_{#1}(#2)}
\newcommand{\Mkn}[1]{\mathsf{M}_{#1}}
\newcommand{\Vlim}{\mathsf{V}}
\newcommand{\Mlim}{\mathsf{M}}
\newcommand{\GMM}[2]{\mathrm{GMM}(#1;#2)}
\newcommand{\GMMT}[4]{\mathrm{GMM}(#1;(#2, #3),#4)}
\newcommand{\nuovorel}{\mathscr{S}^-}% {_{\DVTn,\calS}}
\newcommand{\BV}{\mathrm{BV}}
\newcommand{\AC}{\mathrm{AC}}
\newcommand{\jump}[1]{\mathrm{J}_{#1}}
\newcommand{\Cc}{\mathrm{C}_{\mathrm{c}}}
\newcommand{\Cb}{\mathrm{C}_{\mathrm{b}}}
\newcommand{\Bb}{\mathrm{B}_{\mathrm b}}
\newcommand{\Bc}{\mathrm{B}_{\mathrm c}}
\newcommand{\Co}{\mathrm{C}_{0}}
\newcommand{\Fish}{\mathscr{D}}
\def\nlambda{n}
\DeclareMathOperator\conv{conv}
\newcommand{\sfp}{\mathsf{p}}

\newcommand{\weakto}{\rightharpoonup}
\newcommand{\weaksto}{\stackrel*\rightharpoonup}
\newcommand{\weaksigmatoabs}{\stackrel{\sigma}{\rightharpoonup}}
\newcommand{\hj}{\bj'}
\newcommand\cj{\tilde \bjmath}
\newcommand\cbj{\tilde \bjmath}
\newcommand\cw{w^\flat} %{\widetilde w}
\newcommand\hw{\widehat w}
\newcommand\symmap{\mathsf s_\#}
\newcommand\symmapn{\mathsf s}
\newcommand\wt{\widetilde}
\newcommand\Lebesgue{\mathcal L}


\newcommand{\CEI}[1]{\mathcal{CE}(#1)}
\newcommand{\CEP}[4]{\mathcal{CE}(#1,#2; #3,#4)}
\newcommand{\CEIP}[3]{\mathcal{CE}(#1;#2,#3)}
\newcommand{\CEn}{\mathcal{CE}}

%%%%%%%%%%%%%%%%%%%% Piecewise interpolations %%%%%%%%%
\newcommand{\piecewiseConstant}[2]{\overline{#1}_{\kern-1pt#2}}
\newcommand{\pwC}{\piecewiseConstant}
\newcommand{\underpiecewiseConstant}[2]{\underline{#1}_{\kern-1pt#2}}
\newcommand{\upwC}{\underpiecewiseConstant}
\newcommand{\sft}{\mathsf{t}}
\newcommand{\ttau}{t_{\tau}}
\newcommand{\Utau}{\rho_\tau}
\newcommand{\pwM}[2]{\widetilde{#1}_{\kern-1pt#2}}
%\newcommand{\gen}{\mathscr{Y}}
\def\gen(#1,#2){\calS_{#1}(#2)}
\newcommand\genn[1]{\calS_{#1}}
\newcommand{\nuovo}{\mathscr{S}}
\let\ol\overline


\newcommand{\KV}{K_V}
\newcommand{\Kop}{\boldsymbol K}
\newcommand{\Kops}{\boldsymbol K^*}
\newcommand{\Qop}{\boldsymbol Q}
\newcommand{\Qops}{\boldsymbol Q^*}

\newcommand{\pdot}{\boldsymbol{\cdot}}


%%%%%%%%%%%%Giuseppe's Macros
\newcommand{\teta}{\boldsymbol \vartheta}
\newcommand{\serifsigma}{{\sansmath \sigma}}
\newcommand{\tetapi}{\boldsymbol{\teta}_{\kappa}}
\newcommand{\tetapil}{\boldsymbol{\teta}_{\kappa}^d}
\newcommand{\tetapie}{\boldsymbol{\teta}_{\kappa_\eps}}
\newcommand{\tetapien}{\boldsymbol{\teta}_{\eps_n}}
\newcommand{\pinfty}{{+\infty}}
\newcommand{\mres}{\kern1pt\mathbin{\vrule height 1.6ex depth 0pt width
    0.13ex\vrule height 0.13ex depth 0pt width 1.3ex}}
\newcommand{\topref}[2]{\stackrel{\eqref{#1}}#2}
\newcommand{\Dom}{\calM^+}
\newcommand{\half}{\relax} %{\frac 12}
\newcommand{\thalf}{\relax} %{\tfrac 12}
\newcommand{\frA}{\mathfrak A}
\newcommand{\frB}{\mathfrak B}
\newcommand{\frc}{\mathfrak c}
\newcommand{\frk}{\mathfrak k}
\newcommand{\frl}{\mathfrak l}
\newcommand{\frs}{\mathfrak s}
\newcommand{\frf}{\mathfrak f}
\newcommand{\frm}{\mathfrak m}
\newcommand{\kernel}[2]{\boldsymbol {#1}_{#2}}
\def\calS{\mathscr E}
\def\calF{\mathscr F}
\def\calR{\mathscr R}
\def\Aalpha{\upalpha}
\newcommand{\Lebone}{\scrL}
\newcommand{\bnu}{\boldsymbol\upnu}
\newcommand{\bj}{{\boldsymbol j}}
\newcommand{\bsigma}{{\boldsymbol \varsigma}}
\newcommand{\bjmath}{{\boldsymbol\jmath}}

\newcommand{\rfield}{\mathrm F_0}
\newcommand{\field}{\mathrm F}
\newcommand{\OrmD}{\rmD}
\newcommand{\Gop}{\boldsymbol G}
\newcommand{\Fop}{\boldsymbol F}
\newcommand{\restr}[1]{\lower3pt\hbox{$|_{#1}$}}
\newcommand{\nchi}{{\raise.3ex\hbox{$\chi$}}}
%%%%%


\newcommand{\scrR}{\mathscr{R}}
\newcommand{\scrD}{\mathscr{D}}
\newcommand{\scrE}{\mathscr{E}}
\newcommand{\jj}{{\boldsymbol{j}}}
\newcommand{\bg}{{\boldsymbol{g}}}
\newcommand{\nn}{{\boldsymbol{n}}}
\newcommand{\bss}{{\boldsymbol{s}}}
\newcommand{\bG}{{\boldsymbol{G}}}
\newcommand{\Ed}{{E'}}
\newcommand{\dV}{\mathsf{d}}
\newcommand{\Mloc}{\mathcal{M}_{\mathrm{loc}}}
\newcommand{\pairing}[4]{ \sideset{_{#1 }}{_{ #2}}  {\mathop{\langle #3 , #4  \rangle}}}
\newcommand{\ej}{\eps_n}
\newcommand{\psih}{\mathfrak{f}}%{\mathfrak{f}_{\uppsi^*}}
\newcommand{\nbl}[1]{\|#1\|_{\mathrm{BL}} }
\newcommand{\ssigma}{\boldsymbol \sigma}
\newcommand{\nnu}{\boldsymbol \nu}
\newcommand{\bbeta}{\boldsymbol \beta}
\newcommand{\zzeta}{\boldsymbol \zeta}
\newcommand{\eeta}{\boldsymbol \eta}

\newcommand{\rmL}{\mathrm{L}}
\newcommand{\rmX}{\mathrm{X}}

\newcommand{\down}{\downarrow}
%\newcommand{\Ep}{E'}

\newcommand{\RBS}{\color{black}} %{\color{ddcyan}}
\newcommand{\RCR}{\color{magenta}}
\newcommand{\RNEW}{\color{black}} %{\color{ddcyan}}

 \newcommand{\TODOBS}[1]{\todo[inline, color=red!40]{#1}}
 \newcommand{\TODOR}[1]{\todo[inline, color=magenta!20]{#1}}
  \newcommand{\TODO}[1]{\todo[inline, color=cyan!20]{#1}}
   \newcommand{\TODOH}[1]{\todo[inline, color=yellow!20]{#1}}
  \newcommand{\RRR}{\color{magenta}}
 \newcommand{\EEE}{\color{black}}
 \newcommand{\WARNING}{\color{red}}
 \newcommand{\dpi}{{\color{magenta} d}}

\numberwithin{equation}{section}


\title[]{Gradient-flow structures for singular jump processes}


\begin{document}


\author{Jasper Hoeksema}
%
\address{}
\email{}



\author{Riccarda Rossi}
%
\address{R.\ Rossi, DIMI, Universit\`a degli studi di Brescia. Via Branze 38, I--25133 Brescia -- Italy}
\email{riccarda.rossi\,@\,unibs.it}



\author{Oliver Tse}
%
\address{}
\email{}




\maketitle


\centerline{Ricky, February 16th--27th, 2024}



\begin{abstract}

\end{abstract}


\section{Introduction}
We focus on Markov jump processes 
\[
aaa
\]
featuring a \emph{singular} jump kernel, namely...
\par
We aim to impart to these processes a (generalized) gradient-flow structure in terms of a suibable
Energy-Dissipation balance.
In particular, we are going to obtain  solutions to this variational formulation as limits of a solutions to a suitably regularized jump process.
%%%%
%%%%
%
\subsection*{\bf Plan of the paper.}
\subsection*{\bf List of symbols.}
Throughout the paper we will use the following notation.
\begin{center}\bigskip
\newcommand{\specialcell}[2][c]{%
  \begin{tabular}[#1]{@{}l@{}}#2\end{tabular}}
\begin{small}
\begin{longtable}{lll}
$\Lebone$ & Lebesgue measure on $\R$ & 
\\
$\ona$, $\odiv$ & graph gradient and divergence &\eqref{eq:def:ona-div}
\\
$ \Bb(Y;\R^m)$ & bounded Borel $\R^m$-valued maps & 
\\
$ \Bc(Y;\R^m)$ & bounded and compactly supported  Borel $\R^m$-valued maps & 
%$\upalpha(\cdot,\cdot)$ & multiplier in flux rate $\bnu_\rho$ & Ass.~\ref{ass:Psi}\\
%$\upalpha^\infty$, $\upalpha_*$ & recession function, Legendre transform & Section~\ref{subsub:convex-functionals} \\
%$\upalpha[\cdot|\cdot]$, $\hat\upalpha$ &  measure map, perspective function & Section~\ref{subsub:convex-functionals} \\
%$\CER ab$ & set of curves $\rho$ with finite action & \eqref{def:Aab}\\
%  $ \|\kappa_V\|_\infty$  \EEE & upper bound on $\kappa$ & Ass.~\ref{ass:V-and-kappa}\\
%$% \Cc,\, 
%  \Cb % \,\Co
%  $ & space of bdd, ct.\ functions with supremum norm\\
%$\CE ab$ & set of pairs $(\rho,\bj )$ satisfying the continuity equation & Def.~\ref{def-CE}\\
%%$\varGamma$ & lower-semicontinuous extension of $p$ & \eqref{eq:varGamma}\\
%  $\rmD_\upphi(u,v)$, $\rmD^\pm_\upphi(u,v)$ & integrands
%             % argument of the integral
%             defining the Fisher information $\Fish$ & \eqref{subeq:D}\\
%$\Fish$ & Fisher-information functional & Def.~\ref{def:Fisher-information}\\
%$\edg = V\times V$ & space of edges & Ass.~\ref{ass:V-and-kappa}\\
%$\calS$, $\rmD (\calS)$ & driving  entropy \EEE functional and its domain & \eqref{eq:def:S} \& Ass.~\ref{ass:S}\\
%$\rmF$ & vector field & \eqref{eq:184}\\
%  $\teta_\rho^\pm$,
%  % $\vec \teta_\rho$
%                & $\rho$-adjusted jump rates & \eqref{def:teta}\\
%$\tetapi$ & equilibrium jump rate & \eqref{nu-pi}\\
%$\kappa$ & jump kernel &\eqref{eq:def:generator} \& Ass.~\ref{ass:V-and-kappa}\\
%$\kernel\kappa\gamma$ & $\gamma \otimes \kappa$ & \eqref{eq:84}\\
%$\mathscr L$ & Energy-Dissipation balance functional &\eqref{eq:def:mathscr-L}\\
%  $\calM(\Omega;\R^m)$, $\calM^+(\Omega)$ & vector (positive) measures on $\Omega$ & Sec.~\ref{ss:3.1}\\
%$\bnu_\rho$ & edge measure in definition of $\calR^*$, $\calR$ &\eqref{eq:def:R*-intro}, \eqref{eq:def:R-intro}, \eqref{eq:def:alpha}\\
%%$p$ & $p = \Psi^*\circ \Psi'$ & \eqref{def:p}\\
%$Q$, $Q^*$ & generator and dual generator & \eqref{eq:fokker-planck}\\
%$\calR$, $\calR^*$ & dual pair of dissipation potentials & \eqref{eq:def:R*-intro}, \eqref{eq:def:R-intro}, Def.~\ref{def:R-rigorous}\\
%$\R_+ := [0,\infty)$ \\
%$\sfs$ & symmetry map $(x,y) \mapsto (y,x)$ & \eqref{eq:87}\\
%$\nuovorel$ & relaxed slope & \eqref{relaxed-nuovo}\\
%  $\Upsilon$ & perspective function associated with
%               $\Psi$ and $\upalpha$& \eqref{Upsilon}\\
%$V$ & space of states & Ass.~\ref{ass:V-and-kappa}\\
%$\upphi$ &  density of $\calS$ & \eqref{eq:def:S} \& Ass.~\ref{ass:S}\\
%$\Psi$, $\Psi^*$ & dual pair of dissipation functions & Ass.~\ref{ass:Psi}, Lem.~\ref{l:props:Psi}\\
%$\DVTn$ & Dynamic-Variational Transport cost & \eqref{def:W-intro} \& Sec.~\ref{sec:cost}\\
%  $\VarWn$ & $\DVTn$- action  & \eqref{def-tot-var}\\
%$\sfx,\sfy$ & coordinate maps $(x,y) \mapsto x$ and $(x,y)\mapsto y$ & \eqref{eq:87}\\
\end{longtable}
\end{small}
\end{center}

\TODOR{add symbols}


\section{Preliminaries of measure theory}

%\TODOR{We don't need to require $V$ complete, right? RIFLETTICI}

\paragraph{\bf Finite measures.}
Let $(Y,d)$ be separable metric space, and let 
$\mathfrak{B}(Y)$ be its associated Borel $\sigma$-algebra. 
We denote by 
 $\calM(Y;\R^m)$  the space of  $\sigma$-additive measures  on
 $\mu: \frB(Y) \to \R^m$ 
 of \emph{finite} total variation
 $\|\mu\|_{TV}: =|\mu|(Y)<\pinfty$, where for every $B\in\frB(Y)$
 \[
   |\mu|(B): = \sup \left\{ \sum_{i=0}^\pinfty |\mu(B_i)|\, : \ B_i \in \frB_Y,\, \ B_i \text{ pairwise disjoint}, \ B = \bigcup_{i=0}^\pinfty B_i \right\}.
 \]
 The set function  $|\mu|: \frB(Y) \to [0,\pinfty)$  is a positive
 finite
 measure on $\frB(Y)$ \cite[Thm.\ 1.6]{AmFuPa05FBVF}
 and $(\calM(Y;\R^m),\|\cdot\|_{TV})$ is a Banach space.
 In the case $m=1$, we will simply write $\calM(Y)$,
 and we shall denote the space of \emph{positive} finite
% (resp.~probability) 
measures on $\frB(Y)$ by $\calM^+(Y)$.
% (resp.~$\calP(Y)$).
   For $m>1$,
 we will identify any element $\mu \in \calM(Y;\R^m)$ with  a vector
 $(\mu^1,\ldots,\mu^m)$, with $\mu^i \in \calM(Y)$ for all
 $i=1,\ldots, m$.
 If $\varphi
 =(\varphi^1,\ldots,\varphi^m)\in \Bb(Y;\R^m)$ (the set of bounded $\R^m$-valued
 $\frB$-measurable
 maps), the duality between $\mu \in \calM(Y;\R^m)$ and $\varphi$
 can be expressed by 
\begin{equation}
\label{vector-duality-measures}
 \langle\mu,\varphi\rangle : = \int_{Y} \varphi \cdot \mu (\dd x) =
 \sum_{i=1}^m \int_Y  \varphi^i(x) \mu^i(\dd x).
 %\qquad \text{for all } \varphi = (\varphi^1,\ldots,\varphi^m)\in \Co(Y;\R^m) \,.
 \end{equation}
% \TODO{do we need the Lebesgue decomposition for Radon measures as well?}
%%%
\paragraph{\bf Radon measures.}
We call \emph{Borel measure} any measure $\mu: \mathfrak{B}(Y) \to [0,+\infty]$. If a Borel measure is finite on the compact subsets of $Y$, we will call it a \emph{positive Radon measure}.
We  call \emph{Radon (vector) measure} any  set function $\mu : \mathfrak{B}_{\mathrm{c}}(Y) \to \R^m$ (where $\mathfrak{B}_{\mathrm{c}}(Y) $ denotes the family of relatively compact Borel subsets of $Y$)  \RNEW such that its restriction to $\frB(K) $,
for any \emph{compact} set $K\subset Y$, \EEE  is a finite (vector) measure. We denote by $\Mloc^+(Y)$, $\Mloc(Y)$, and $\Mloc(Y;\R^m) $  the spaces of positive, real-valued, and vector-valued Radon measures. 
For $\mu \in  \Mloc(Y;\R^m) $  and $\varphi \in \Cc(Y;\R^m)$, the set of continuous $\R^m$-valued functions with compact support the integral in \eqref{vector-duality-measures}
is still well defined and induces a duality pairing between $ \Mloc(Y;\R^m) $  and $\varphi \in \Cc(Y;\R^m)$. 
 \par
We recall that, in addition, if $Y$ is a \emph{locally compact} separable (l.c.s.\ for short) metric space, any $\mu \in \Mloc^+(Y)$  is inner regular, 
namely \RCR for every $E \in \mathfrak{B}(Y)$
\begin{equation}
\label{inner-regularity}
\mu(E) = \sup\{ \mu(K) \, : \ K \subset E, \ K \text{ compact}\}\,,
\end{equation} \EEE
cf.\ e.g.\ \cite[Prop.\ 1.43]{AmFuPa05FBVF}. 
% \TODO{do we still need inner regularity? used it in the proof of  Lemma \ref{l:crucial-F}...}
%\TODO{CHECK SCHWARZ, Polish sufficient?}
%\TODO{We can state it in this way, because any Radon measure is $\sigma$-finite, right?}
 \paragraph{\bf Restriction, Lebesgue decomposition.} For every $\mu\in \Mloc(Y;\R^m)$ and $B\in \frB(Y)$
 we will denote by $\mu\mres B$ the restriction of $\mu$ to $B$, i.e.\ 
 $\mu\mres B(A):=\mu(A\cap B)$ for every $A\in \frB(Y)$.
 Let $(X,\mathfrak A)$ be another measurable space and let $\sfp:X\to
 Y$
 a measurable map. 
 For every $\mu\in \Mloc(X;\R^m)$ we will denote by
 $\sfp_\sharp\mu$ the push-forward measure obtained by setting
 \begin{equation}
   \label{eq:82}
   \sfp_\sharp\mu(B):=\mu(\sfp^{-1}(B))\quad\text{for every }B\in \frB(Y).
 \end{equation}
 For every pair $\mu\in \Mloc(Y;\R^m)$ and $\gamma\in \Mloc^+(Y)$
 there exist a unique (up to the modification
 in a $\gamma$-negligible set) $\gamma$-locally integrable map
 $\frac{\dd\mu}{\dd\gamma}:
 Y\to\R^m$, a $\gamma$-negligible set $N\in \frB$
 and a unique measure $\mu^\perp\in \Mloc(Y;\R^m)$
 yielding the \emph{Lebesgue decomposition}
 \begin{equation}
   \label{eq:Leb}
   \begin{gathered}
     \mu=\mu^{\mathrm{a}}+\mu^\perp,\quad \mu^{\mathrm{a}}=\frac{\dd\mu}{\dd\gamma}\,\gamma=
     \mu\mres(Y\setminus N),\quad \mu^\perp=\mu\mres N,\quad
     \gamma(N)=0\\
     |\mu^\perp|\perp \gamma,\quad |\mu|(Y)=\int_Y
     \left|\frac{\dd \mu}{\dd\gamma}\right|\,\dd\gamma+|\mu^\perp|(Y).
   \end{gathered}
 \end{equation}
%%%%
\paragraph{\bf Convergences of measures.}
The space 
$\calM(Y;\R^m)$ is clearly endowed with the notion of convergence induced by the total variation norm  $\|\cdot\|_{TV}$; we shall refer to it as \emph{strong convergence}. On
$\calM(Y;\R^m)$ we will 
  also consider \emph{setwise convergence}, defined, for $(\mu_n)_n, \mu \in \calM(Y;\R^m)$, by 
  \begin{equation}
  \label{setwise-convergence}
  \mu_n \to \mu \text{ \emph{setwise} }  \quad \text{if}  \quad \lim_{n\to\infty}  \mu_n(B) = \mu(B) \quad \text{for all } B \in \frB(Y)\,.
  \end{equation}
  \RCR In fact, the associated topology is the coarsest one on $\calM(Y;\R^m)$ making all the functions $ \frB(Y)\ni B
  \mapsto  \mu(B) $ continuous. 
 \EEE
  Among the various characterizations of setwise convergence, we recall  \cite[\S 4.7(v)]{Bogachev07} that \eqref{setwise-convergence} is equivalent to 
 \begin{enumerate}
   %  (namely the convergence notion induced by the coarsest topology that makes the maps $\mu \mapsto \mu(B)$, for any $B \in \frB$, continuous);  \EEE
 \item \emph{Convergence in duality with $\Bb(Y;\R^m)$}:
   \begin{equation}
     \label{eq:70}
     \lim_{n\to\pinfty}\langle \mu_n,\varphi\rangle=
     \langle \mu,\varphi\rangle
     \qquad
     \text{for every $\varphi\in \Bb(Y;\R^m)$}.
   \end{equation}
 \item \emph{Weak topology convergence in $ \calM(Y;\R^m)$}:
   the sequence $\mu_n$ converges to $\mu$ w.r.t.\ the weak topology
   of the Banach space $(\calM(Y;\R^m);\|\cdot\|_{TV})$.
   \end{enumerate}
   Furthermore,  \RNEW we recall that, 
   by  \cite[Thm.\ 4.7.25)]{Bogachev07}, given a sequence $(\mu_n)_n \subset \calM(Y;\R^m) $, 
   the following properties are equivalent: \EEE
   \begin{itemize}
   \item[(i)]    $(\mu_n)_n$ is relatively compact w.r.t.\ setwise convergence;
   \item[(ii)]    
   \RNEW   there exists $\gamma \in \calM^+(Y)$ such that \EEE
      \begin{equation}
         \label{eq:73}
         \forall\,\eps>0\ \exists\,\delta>0:
         \quad
         B\in \frB(Y),\ \gamma(B)\le \delta\quad
         \Rightarrow\quad
         \sup_{n}\mu_n(B)\le \eps;
       \end{equation}
          \item[(iii)]  \RNEW  there exists $\gamma \in \calM^+(Y)$ such that $\mu_n \ll \gamma$ for all $n\in \N$ and the sequence $\left(\frac{\dd \mu_n}{\dd \gamma}\right)_n$
          admits a subsequence weakly converging in the topology of $\rmL^1(Y,\gamma;\R^m)$.
       \end{itemize}
       \EEE
       \par
  Finally, we recall that  a sequence $(\mu_n)_n \subset \calM(Y;\R^m)$ converges \emph{narrowly} to some $\mu \in  \calM(Y;\R^m)$
if \eqref{eq:70} holds for every $\varphi \in \Cb(Y;\R^m)$, the space of continuous \emph{bounded} $\R^m$-valued functions. 
  \par
  When, in addition, $Y$ is also locally compact, on
$\Mloc(Y;\R^m)$ we will consider
\begin{itemize}
%\item[-] setwise convergence on relatively compact subsets of $Y$, i.e.\ 
%\[ 
%\lim_{n\to\infty}  \mu_n(B) = \mu(B) \quad \text{for all } B \in \frB_{\mathrm{c}}(Y);
%\]
\item[-] \emph{vague} convergence, namely convergence in duality against  all functions  $\varphi \in \Cc(Y;\R^m)$. 
\end{itemize}

%\TODO{do we need at all  setwise convergence on relatively compact sets?}


%%%%%%
\paragraph{\bf Convex functionals of measures} We will work with functionals depending on pairs of Radon measures
defined in this way:
let 
$\Upsilon:\R^m\to [0,\pinfty]$ be proper,  convex and lower semicontinuous
and let us denote
by $\Upsilon^\infty:\R^m\to [0,\pinfty]$ its recession function
\begin{equation}
  \label{recession-upsi}
  \Upsilon^\infty(z):=\lim_{t\to\pinfty}\frac{\Upsilon(tz)}t=\sup_{t>0}\frac{\Upsilon(tz)-\Upsilon(0)}t\,.
\end{equation}
We note that $\Upsilon^\infty$ is  convex, lower semicontinuous, and positively $1$-homogeneous,  with $\Upsilon^\infty(0)=0$. Hence, we define  \EEE
\begin{equation}
\label{def:F-F}
\begin{aligned}
&
\calF_\Upsilon:\Mloc(Y;\R^m) \times \Mloc^+(Y)\mapsto [0,\pinfty], \qquad
\\
&
\calF_\Upsilon(\mu|\nu) :=
\int_Y \Upsilon \Bigl(\frac{\dd \mu}{\dd \nu}\Bigr)\,\dd\nu+
\int_Y \Upsilon^\infty\Bigl(\frac{\dd \mu^\perp}{\dd |\mu^\perp|}\Bigr) \,
\dd |\mu^\perp|,\qquad \text{for }\mu=\frac{\dd \mu}{\dd \nu}\nu+\mu^\perp.
\end{aligned}
% \in \calM(Y;\R^m),\ \gamma\in \calM^+(Y),
% \begin{cases}
%   \displaystyle
%   \int_Y F\Bigl(\frac{\dd \mu}{\dd \gamma}\Bigr) \, \dd
%   \gamma&\text{if }\mu\ll\gamma,\\
%   +\infty&\text{otherwise.}
% \end{cases}
\end{equation}
\par
In \cite[Lemma 2.3]{PRST22} we collected some properties of functionals of this class, albeit defined on 
$\calM(Y;\R^m) \times \calM^+(Y)$; most of them extend to the present case. We  highlight the properties that will be used in what follows: in particular,
lower semicontinuity extends to \emph{vague} convergence. \RNEW The proof of the following result will be carried out in Appendix \ref{s:a.1} ahead. \EEE
% \TODO{highlight that this is an extension of the result in Ambro}
\begin{lemma}
\label{l:crucial-F}
The functional $\calF_\Upsilon:\Mloc(Y;\R^m) \times \Mloc^+(Y)\mapsto [0,\pinfty]$ enjoys the following properties:
\begin{enumerate}
\item \RNEW if $\Upsilon(0)=0$, then for every $\mu \in \Mloc(Y;\R^m) $ and $\nu,\, \nu' \in \Mloc^+(Y)$ there holds
\begin{equation}
\label{AC-monotonicity}
\nu \leq \nu' \ \Longrightarrow  \ \calF_\Upsilon(\mu|\nu') \leq  \calF_\Upsilon(\mu|\nu); \EEE 
\end{equation} \EEE
 \item if $\Upsilon$ is superlinear, then
\begin{equation}
\label{superlinearity}
  \calF_\Upsilon(\mu|\nu)<\infty\quad\Longrightarrow\quad
  \mu\ll\nu,\quad
  \calF_\Upsilon(\mu|\nu)=
\int_Y \Upsilon \Bigl(\frac{\dd \mu}{\dd \nu}\Bigr)\,\dd\nu;
\end{equation}
\item if $ \Upsilon$  positively $1$-homogeneous, then
    $\Upsilon\equiv \Upsilon^\infty$, $\calF_\Upsilon(\cdot|\nu)  \doteq \calF_\Upsilon$ is
    independent of $\nu$ and  satisfies
    \begin{equation}
      \label{eq:78}
       \calF_\Upsilon(\mu) \EEE =\int_Y
      \Upsilon\left(\frac{\dd\mu}{\dd\gamma}\right)
      \,\dd\gamma\quad
      \text{for every }\gamma\in \Mloc^+(Y)\text{ such that } \mu\ll\gamma;
    \end{equation}
    \item  $\calF_\Upsilon(\cdot|\cdot)$ is  sequentially  lower
    semicontinuous in $\Mloc(Y;\R^m) \times \Mloc^+(Y)$ with respect
    to vague convergence.
\end{enumerate}
\end{lemma}
%\TODO{move proof to an appendix}


%Note that when $\Upsilon$ is superlinear then $\uppsi^\infty(x)=\pinfty$
%in $\R^m\setminus\{0\}$
%\begin{equation}
%  \label{eq:5}
%  \text{$\uppsi$ superlinear,}\quad
%
%\end{equation}
%convex functionals of measures. Main result: lower semicontinuity w.r.t.\ vague convergence.
\paragraph{\bf Concave transformations of vector measures}
We will also need to extend  a construction set forth in \cite[Sec.\ 2.3]{PRST22}
for vector and positive finite measures, to  the case of vector and positive \emph{Radon} measures. Namely,
 let  $\R_+:=[0,\pinfty[$, $\R^m_+:=(\R_+)^m$, and let
$\upalpha:\R^m_+\to\R_+$ be a continuous and concave function. It is
obvious that $\upalpha$ is non-decreasing with respect to each variable.
As in \eqref{recession-upsi}, the recession function $\upalpha^\infty$ is defined
by
\begin{equation}
  \label{eq:1}
  \upalpha^\infty(z):=\lim_{t\to\pinfty}\frac{\upalpha(tz)}t=\inf_{t>0}\frac{\upalpha(tz)-\upalpha(0)}t,\quad
  z\in \R^m_+.
\end{equation}
We define 
\begin{equation}
  \label{alpha-mu-gamma}
  \begin{aligned}
  &
  \Aalpha:\Mloc(Y;\R^m_+)\times\Mloc^+(Y)\to\Mloc^+(Y), 
  \\
  &
  \Aalpha[\mu|\gamma]:=
  \upalpha\Bigl(\frac{\dd\mu}{\dd\gamma}\Bigr)\gamma+
  \upalpha^\infty\Bigl(\frac{\dd\mu}{\dd
    |\mu^\perp|}\Bigr)|\mu^\perp|\quad
  \mu\in \Mloc(Y;\R^m_+),\ \gamma\in \Mloc^+(Y),
  \end{aligned}
\end{equation}
where  $\mu=\frac{\dd\mu}{\dd\gamma}\gamma+\mu^\perp$ is the
Lebesgue decomposition of $\mu$ with respect to ~$\gamma$. 


\section{Setup and gradient system structure}
\label{s:3}
%%%
We start by collecting our conditions on the  vertex space $V$,
 on the reference measure $\pi$, which will be invariant under the evolution
generated by our generalized gradient system, and on the family of singular kernels $(\kappa(x,\cdot))_{x\in V} $ (whose `singularity' is  encoded in the fact that 
for each $x\in V$ $\kappa(x,\cdot)$ is a positive Radon measure on $V\setminus \{x\}$). 
Recall that $E =V \times V$ is the space of edges; 
a special role will be played by the subset obtained removing from $E$ its diagonal
\[
\Ed: = E \setminus \{ (x,x)\, : \ x \in V\}
\] 
and by the measure  $\tetapi$ on $\Ed$ defined by 
\begin{equation}
\label{teta-coupling}
	\tetapi(\dd x\dd y) := \kappa(x,\dd y)\pi(\dd x), \qquad \tetapi(A{\times}B) = \int_{A} \kappa (x,B) \pi(\dd x) 
\end{equation}
for all \emph{disjoint} (Borel) subsets $A,B \subset V$.
\RCR We postpone to Lemma \ref{l:properties-tetakappa} a thorough discussion of the properties of 
$\tetapi$ deriving from Assumption \ref{Ass:V} below. For its statement  we need to introduce  \EEE
%it will be  useful to use 
the symmetry map
$s: E \to E$, 
 $s(x,y):=(y,x)$. 
 \begin{assumption}[Vertex space and kernels]
 \label{Ass:V}
 \sl
We suppose that
\begin{enumerate}
\item the vertex space
\begin{equation}
\label{ass:V}
(V,\dV) \text{ is a locally compact separable metric space, with Borel $\sigma$-algebra $\mathfrak{B}(V)$; }
\end{equation}
\item the reference measure
$\pi \in \calM^+(V)$ is a finite positive measure
\item the kernels 
 $(\kappa(x,\cdot))_{x\in V} $ form a  Borel family of measures in $\Mloc^+(V{\setminus} \{x\})$, i.e.\
 \begin{subequations}
 \label{lambda-conditions}
\begin{equation}
\label{true-measurability}
 \text{ for all } f \in \Bc(E') 
\text{ the map } V \ni x \mapsto \int_{V{\setminus}\{x\}} f(x,y) \kappa(x,\dd y) \text{ is Borel,}
\end{equation}
such that  for all $x\in V$
\begin{equation}
\label{bounds}
\forall\, \eps>0 \ \int_{V{\setminus}B_\varepsilon(x)}\kappa(x,\dd y) <+\infty \qquad  \text{ with  } \qquad \lim_{\varepsilon\down 0}\int_{V\setminus B_\varepsilon(x)} \kappa(x,\dd y) = +\infty.
\end{equation}
\RCR Furthermore,
% there exists a Borel function $\lambda: E \to [0,\infty)$, such that 
%\[
%\exists\, c_\lambda >0 \ \forall\, (x,y) \in E \qquad \lambda(x,y) \geq c_\lambda \dV(x,y)\,,
%\]
%fulfilling
\begin{equation}
\label{mitigation of singularity}
\sup_{x\in V}\int_{V\setminus\{x\}} (1{\wedge} d^2(x,y)) \,\kappa(x,\dd y) =: c_\kappa <+\infty.
\end{equation}
\end{subequations} \EEE
\item Finally,  the \emph{detailed balance condition} holds, i.e.\
 the coupling 
$\tetapi \in \Mloc^+(\Ed)$ defined by  \eqref{teta-coupling}
satisfies 
\begin{equation}
\label{DBC}
s_\# \tetapi = \tetapi
\end{equation}
\end{enumerate}
 \end{assumption}
 \RNEW
 As an immediate consequence of \eqref{mitigation of singularity} we have that,
 while for every $x\in V$ $\kappa(x,\cdot)$ is positive Radon measure on $V\setminus \{x\}$,
the measure
\RCR $(1{\wedge} d^2(x,y))\kappa(x,\cdot )$ extends to the whole $V$. \EEE
\RNEW Likewise, 
  the formula 
 \begin{equation}
 \label{tetapi-lambda}
 \tetapil (A) :=  \iint_A \RCR (1{\wedge} d^2(x,y)) \EEE \, \tetapi (\dd x \dd y) \text{ for all } A \in \mathfrak{B}(E), \text{ defines a (\emph{finite}) measure in } \calM^+(E),
 \end{equation}
 where the extension to the whole if $E$ is carried out by setting
 \[
  \tetapil(E{\setminus}\Ed ) := \lim_{r\downarrow 0}\int_{V} \int_{B_r(x){\setminus}\{x\}}  \RCR (1{\wedge} d^2(x,y)) \EEE  \,\kappa(x,\dd y) \, \dd x \,.
 \]
 \begin{remark}
 \label{rmk:lambda}
 \upshape
Let us dig deeper  on our conditions on the kernels  $(\kappa(x,\cdot))_{x\in V} $:
 \begin{enumerate}
 \item
 In fact, by the local compactness of $V$, the measurability requirement \eqref{true-measurability} is equivalent to having
 that the map
 \[
 V \ni x \mapsto \int_{V{\setminus}\{x\}}  \RCR (1{\wedge} d^2(x,y)) \EEE  f(x,y) \kappa(x,\dd y) \text{ is Borel  \emph{for all }}  f \in \Bb(E').
 \]
% \item We can suppose without loss of generality that the function $\lambda$ is also \emph{symmetric}.  Indeed, let $\lambda_{\mathrm{sym}}(x,y): = \tfrac12 (\lambda(x,y){+}\lambda(y,x))$: relying on the detailed balance condition \eqref{DBC} we immediately check that 
% \[
% \int_{V\setminus\{x\}} (1{\wedge} \lambda_{\mathrm{sym}}^2(x,y)) \,\kappa(x,\dd y) =  \int_{V\setminus\{x\}} (1{\wedge} \lambda^2(x,y)) \,\kappa(x,\dd y)\,.
% \]
 \item Condition \eqref{mitigation of singularity} serves the purpose of \emph{taming} the singularity of the kernels at the diagonal of $E$. In fact, since $\lambda$ dominates $d$, 
 \eqref{mitigation of singularity} implies that 
 \[
 \sup_{x\in V}\int_{V\setminus\{x\}} (1{\wedge} d^2(x,y)) \,\kappa(x,\dd y)\leq \frac{c_\kappa}{c_\lambda^2}<+\infty\,,
 \]
 so that, in particular, the second of \eqref{bounds} is mitigated by the estimate
 \begin{subequations}
 \label{consequences-of-mitigation}
 \begin{equation}
 \label{conse-1}
 \lim_{\eps \downarrow 0} \sup_{x\in V}  \RCR \int_{B_1\setminus B_\varepsilon(x)}  d^2(x,y)\kappa(x,\dd y)<+\infty\,.
 \end{equation}
In turn, since for $R \gg 1$ we have \RCR $ (1{\wedge} d^2(x,y)) = 1$ \EEE for all $y\in V\setminus B_R(x)$,  \eqref{mitigation of singularity}  also guarantees that
 \begin{equation}
 \label{conse-2}
    \lim_{R \to \infty} \sup_{x\in V}\int_{V\setminus B_R(x)}  \kappa(x,\dd y)<+\infty\,.
\end{equation}
\end{subequations}
We remark that properties \eqref{consequences-of-mitigation} are weaker than the uniform integrability condition required in 
\cite[Assumption 1.1]{Erbar14} in the Euclidean setting of $V=\R^d$, $ d(x,y) = |x{-}y|$, namely that
\[
\lim_{R \to \infty} \sup_{x\in V}  \left(  \int_{V\setminus B_{1/R}(x)}  |x-y|^2 \,\kappa(x,\dd y)  {+}   \int_{V\setminus B_{R}(x)}
\kappa(x,\dd y)    \right)  =0\,.
\]
\end{enumerate}
 \end{remark}
 \RCR \noindent Before moving on,  let us  pin down an easy, albeit crucial, property of $\tetapi$.
 \TODOR{\EEE If we decide to kill Lemma \ref{l:cutoff}, then we can also kill Lemma \ref{l:properties-tetakappa}...}
 \begin{lemma}
 \label{l:properties-tetakappa}
 Under Assumption \ref{ass:V},   the coupling measure  $\tetapi\in \Mloc^+(\Ed)$ enjoys the following property
%   defined by \eqref{teta-coupling} is a Radon measure on $\Ed$ 
\begin{equation}
\label{null-property}
\text{for all } N,B \in \mathfrak{B}(V) \text{ with } N \cap B = \emptyset\,: 
\quad \pi(N) =0 \ \Rightarrow \ \tetapi(N{\times}B) = \tetapi(B{\times}N)=0\,.
\end{equation}
In particular, suppose that a given property  $\mathfrak{P} $ holds $\pi$-almost everywhere in $V$. Then, the property 
$\mathfrak{P}^{\wedge} $ defined for all  $ (x,y) \in E$   by 
\begin{equation}
\label{Pwedge}
\mathfrak{P}^{\wedge} (x,y):= \mathfrak{P}(x) \wedge  \mathfrak{P}(y) \quad  \text{ holds } \tetapi\text{-almost everywhere in } \Ed\,.
\end{equation} 
 \end{lemma}
 \begin{proof}
Let $(\eps_n)_n$ be a null sequence. By Fatou's Lemma 
\[
\tetapi(N{\times}B) = \int_N \int_B \kappa(x,\dd y ) \pi(\dd x ) \leq \liminf_{n\to\infty} \int_{N} \int_{B{\setminus}B_{\eps_n}(x)} \kappa(x,\dd y )  \pi (\dd x )\,.
\]
On the other hand, for $\eps_n<1$ we have, by  \eqref{mitigation of singularity}, 
\[
\begin{aligned}
\int_{N} \int_{B{\setminus}B_{\eps_n}(x)}\, \kappa(x,\dd y )  \pi (\dd x )  & \leq \int_{N} \int_{V{\setminus}B_{\eps_n}(x)} \frac1{1{\wedge}  d^2(x,y)}
  (1{\wedge} d^2(x,y)) \,\kappa(x,\dd y)\,   \pi (\dd x ) 
 \\ &  \leq c_\kappa \left(\frac1{\eps_n^2} {+} 1\right) \pi (N) =0\,,
 \end{aligned}
\]
and \eqref{null-property} follows. 
\par
In order to check \eqref{Pwedge}, suppose that $\mathfrak{P}$ holds for all $x\in B$, with   $B\in \mathfrak{B}(V)$ such that  $\pi(V{\setminus}B)=0$.
Then, $\mathfrak{P}^{\wedge}$ holds for all $(x,y)\in B\times B$, and by \eqref{null-property} we have 
\[
\tetapi (\Ed {\setminus} (B{\times}B
)) \leq \tetapi (B{\times}(V{\setminus}B)) + \tetapi ((V{\setminus}B){\times}B) =0\,. %+ \tetapi (\Ed \cap [(V{\setminus}B){\times}(V{\setminus}B)])  =0\,.
\]
 \end{proof}
 \EEE
% \TODO{I am not fully satisfied with the above remark, it still misses our motivation for allowing a general $\lambda$, in place of $d$. But indeed I am missing it myself. 
% If we assume that $\lambda \geq d$, we can't appreciate the fact that $\lambda $ is general when $d(x,y) \to \infty$, because then $1\wedge \lambda^2(x,y) =1$ and $\lambda$ 
% doesn't show up in the condition. When $d(x,y) \to 0$, then in principle we don't know that $\lambda(x,y) \to 0$, do we? Then, in the integral condition 
% on $V\setminus B_\varepsilon(x)$, with $\epsilon \down 0$, we can't establish that $1\wedge \lambda^2(x,y) =\lambda^2(x,y)$. We can only deduce \eqref{conse-1}....
% \\
% All in all, I can't yet fully understand the usage of a general $\lambda$....
% \\
% Finally, later on (on page \pageref{crucial-bound-kappaeps}) I have  another, maybe trivial, question on \eqref{mitigation of singularity}....
% }
% 
% \TODOBS{Oliver agrees with the above. In the setting of Assumption \ref{ass:V}(3), it is not meaningful to allow for a general $\lambda$. It would be if, in place of 
% constraining $\lambda$ by $\lambda \geq c d$, we allowed for more flexibility. Ideally, $\lambda$ should behave in this way:
% \[
% \lambda(x,y) \sim d(x,y) \  \text{if } d(x,y) \to 0
%\qquad 
% \lambda(x,y) \sim \frac1{d(x,y)} \  \text{if } d(x,y) \to +\infty
% \]
% Typical example:
% \[
% \lambda^2(x,y)= \frac{1{\wedge}d^2(x,y)}{1+d^2(x,y)}
% \]
% This would indeed be a generalization, which would however significantly affect the remainder of the analysis, starting from the choice of test functions for the CE.
% \\
% How about: 
% \\
% (1) keeping $\lambda$ general for the moment
% \\
% (2) fixing all details in the present setup
% \\
% (3) checking the more general setup in the end. If it is feasible, we could add the discussion in a section of `Generalizations' with the subsection `Integrability properties of the kernels'
% ???
%  }
 

 
 %\\
% Instead, it seems to me that we need to require $\lambda$ continuous, see the comments at the beginning of Sec.\ $4$}
 

 



Our variational structure  will also be  based on a driving energy functional, on a dissipation potential, and on a flux density map whose properties are collected below.
\begin{assumption}[Energy]
\label{Ass:E} 
\sl
The energy functional
$\scrE: \mathcal{M}^+(V) \to [0,+\infty]$  is of the form
\[
\scrE(\rho) = \begin{cases}
\displaystyle \int_V \upphi\left(\frac{\dd\rho}{\dd \pi} \right)\, \dd \pi & \text{if } \rho \ll \pi,
\\
+\infty  & \text{otherwise.}
\end{cases}
\]
We require the energy density to fulfill
\begin{equation}
\label{ass-phi}
\begin{gathered}
\upphi \in \mathrm{C}([0,+\infty] ) \cap \mathrm{C}^1(]0,+\infty] ), \  \text{$\upphi$ convex with  $\min \upphi =0$,}
\\
\lim_{r\to+\infty}  \frac{\upphi(r)}{r} =+\infty.
\end{gathered}
\end{equation}
\end{assumption}
%\RCR remark that  $\uppsi^*$ is non-decreasing on $ [0,+\infty)$ \EEE
\begin{assumption}[Dissipation]
\label{Ass:D}
\sl
The dual dissipation density $\uppsi^*:\R \to [0,+\infty)$ is convex, differentiable, even, with $\uppsi^*(0)=0$ and 
\begin{align}
\label{growth-psistar}
&
 \lim_{|\xi|\to+\infty} \frac{\uppsi^*(\xi)}{|\xi|} =+\infty,
\\
&
 \label{quadratic-at-0}
\RCR \lim_{\xi \to 0} \frac{\uppsi^*(\xi)}{|\xi|^2} =c_0 \in ]0,+\infty[\,. \EEE
 \end{align}
 \end{assumption}
%
\begin{assumption}[Flux density]
\label{Ass:flux-density}
The  mapping $\upalpha: [0,+\infty)\times [0,+\infty) \to [0,+\infty)$ is non-degenerate (not identically null),
%in the sense that 
%\begin{equation}
%\label{non-deg-alpha}
%\alpha(r,s)>0 \quad \text{if  } |r{-}s| 
%\end{equation}
 continuous, concave, symmetric, with 
\begin{equation}
\label{props-alpha}
\upalpha(u_1,u_2) = \upalpha(u_2,u_1) \qquad\forall\, (u_1,u_2) \in  [0,+\infty)\times [0,+\infty)\,.
\end{equation}
\end{assumption}
%%%%
\RCR
Before moving  on, let us pin down two key consequences of Assumption \ref{Ass:D} that will be extensively used later on. Other outcomes will be highlighted
in Section \ref{s:6.1}. 
\begin{lemma}
The function $\uppsi^*$ is non-decreasing on $[0,\infty)$ and 
there exists a convex, even, and superlinear function $\psih : \R \to [0,+\infty)$ such that
\begin{subequations}
\label{needed-control}
\begin{equation}
\label{needed-control-normal}
\uppsi^*(\eta\xi) \leq \eta^2 \psih(\xi) \qquad \text{for all } \eta \in [-1,1], \ \xi \in \R\,
\end{equation}
or, equivalently,
\begin{equation}
\label{needed-control-bound}
\forall\, \xi \in \R \ \forall\, M>0 \ \forall\, \beta \in [{-}M, M] \,: \qquad \uppsi^*(\beta\xi) \leq \frac1{M^2}\eta^2 \psih(M\xi )\,. 
\end{equation}
\end{subequations}
\end{lemma}
\begin{proof}
Since $0=\uppsi^*(0) = \min_{\R} \uppsi^*$, we have $(\uppsi^*)'(0)=0$ and thus $(\uppsi^*)'(\xi)\geq 0$ for all $\xi \geq 0$. Hence, the monotonicity  statement for 
$\uppsi^*$. 
\par
Now, from \eqref{quadratic-at-0} we infer that 
\begin{equation}
\label{easy-conseq}
\exists\, r>0 \ \ \forall\, \eta \in  [{-}r,r]\,:  \   \frac12 c_0 |\eta|^2\uppsi^*(\eta)\leq \frac32 c_0 |\eta|^2\,.
\end{equation}
To show \eqref{needed-control-normal} we distinguish two cases: 
\begin{enumerate}
\item $|\eta \xi| \leq r$: then 
\begin{equation}
\label{est4needed-1}
\uppsi^*(\eta \xi) \leq  \frac32 c_0 \eta^2 \xi^2\,. %\leq  \frac32 c_0 |\beta|^2  \frac{2}{c_0} \uppsi^*(\xi) = 3  |\beta|^2 \uppsi^*(\xi) \,.
\end{equation}
\item $|\eta \xi| > r$: then, 
\begin{equation}
\label{est4needed-2}
\uppsi^*(\eta \xi) \overset{(1)}{=}\uppsi^*(|\eta \xi|)\overset{(2)}{\leq}\uppsi^*( \xi)  \overset{(3)}{\leq}\frac{|\eta\xi|}{r}\uppsi^*( \xi)    \overset{(4)}{\leq}  \frac{|\xi|^2}{r} |\eta|^2 \uppsi^*( \xi)  
\,,
\end{equation}
where {\footnotesize (1)} follows from the evenness of $\uppsi^*$, {\footnotesize (2)} from its monotonicity on $[0,+\infty)$, and  {\footnotesize (3), (4)}  from the fact that $\tfrac{|\eta \xi|} r>1$. 
\end{enumerate}
All in all, we conclude that 
\[
\uppsi^*(\eta \xi) \leq  \eta^2 \left( \frac32 c_0  \xi^2 {+}  \frac{|\xi|^2}{r}  \uppsi^*( \xi)  \right) \doteq  \eta^2 \psih(\xi) \,.
\]
It is immediate to check that the thus defined function $\psih : \R \to [0,+\infty)$ is convex, even and with superlinear growth at infinity. This finishes the proof. 
\end{proof}

%\begin{remark}
%\label{rmk:primal-condition}
%\upshape
%It is immediate to check that \eqref{needed-control} is equivalent to the following condition, involving the primal dissipation potential $\uppsi$ and the convex conjugate $\psih^*$ of $\psih$
%\begin{equation}
%\label{primal-condition}
%\uppsi(v) \geq \eta^2 \psih^*\left(\frac v\eta \right)  \qquad \text{for all } \eta \in [-1,1], \ v \in \R\,.
%\end{equation}
%\end{remark} 
%\TODO{it seems to me that, while having \eqref{primal-condition} is useful, it is better to formulate \eqref{needed-control} as a hyp.}

%
%\TODO{should we keep \eqrefl{formal_definition}?
\paragraph{\bf The primal and dual dissipation potentials: formal definition.}
\RNEW After specifying our conditions  the  triple $(\kappa, \uppsi, \upalpha)$, we are in a position to 
 (formally) introduce  the dissipation potentials $\scrR$ and 
$\scrR^*$ via 
%: \calM^+(V)\times \Mloc(\Ed) \to [0,+\infty)$ and $\scrR^*: \calM^+(V)\times \Cc(\Ed) \to [0,+\infty)$
\begin{equation}
\label{formal_definition}
\begin{aligned}
&
\scrR(\rho,\jj): = \frac12 \iint_{E'} \uppsi \left( 2 \frac{\dd \jj}{\dd \bnu_\rho} \right) \, \bnu_\rho(\dd x \dd y) \quad\text{with } \bnu_\rho(\dd x \dd y) =\upalpha(u(x),u(y)) \, \tetapi (\dd x, \dd y ) \text{ and } u = \frac{\dd \rho}{\dd \pi}
\\
&
\scrR^*(\rho,\xi): = \frac12 \iint_{E'} \uppsi^*(\xi)  \,\bnu_\rho(\dd x \dd y)\,.
\end{aligned}
\end{equation}
Their rigorous definition is postponed to Def.\ \ref{def:primal-and-dual} ahead. \EEE
%\TODO{In principle, we should at first define $\calR^*$ on $\calM^+(V)\times \Cc(\Ed)$, right?}

%\TODO{Add  useful related results if needed}


\subsection{The continuity equation}
Recall the definition of the `graph gradient' 
$\ona$ 
%\footnote{\GGG The definition of divergence is the same, but I
%  inverted the order in the integrand, just to help the
%  intuition that when $j\ge0$ the opposite divergence is
%  the inflow minus the outflow. This is what
%  appears in the ODE $\partial_t \rho=-\odiv \bj $.
%  If you think that it is confusing, we can revert to the original
%formula. \MARK{I would prefer the original formulation; from what is here I found it difficult to reconstruct what is meant by the `divergence' (with or without the minus sign?)}
% Maybe we can keep the original formulation, and add the remark that ` $j\ge0$ the opposite divergence is
%  the inflow minus the outflow. This is what
%  appears in the ODE $\partial_t \rho=-\odiv \bj $' after introducing the continuity equation..}
\begin{subequations}
\label{eq:def:ona-div}
\begin{align}
\label{eq:def:ona-grad}
(\ona \varphi)(x,y) &:= \varphi(y)-\varphi(x) &&\text{for any Borel function }\varphi: V \to \R\,,
\intertext{and of  the `graph divergence operator' $ \odiv:\Mloc(\edg)\to\Mloc(V)$}                                     
       \label{eq:def:div}
  (\odiv \bj )(\dd x) &:= \int_{y\in V}
                         \bigl[\bj (\dd x,\dd y)-\bj (\dd y,\dd
                         x)\bigr]
                                              &&\text{for any }\bj \in \Mloc(\edg)\,.
\end{align}
\end{subequations}
Clearly, we have that 
\begin{equation}
  \label{eq:nabladiv}
  \iint_\edg \ona\varphi(x,y)\,\bj (\dd x,\dd y)=
  -\int_V \varphi(x) \,\odiv \bj (\dd x)\quad
  \text{for every }\varphi\in \Cc(V).
\end{equation}
 Hereafter, for a given function $\mu :I \to \calM(V)$, or
  $\mu : I  \to \calM(\edg)$, with $I=[a,b]\subset\R$,
  we shall often
  write $\mu_t$ in place of $\mu(t)$ for  a given $t\in I$ and denote
  the time-dependent function $\mu $ by  $(\mu_t)_{t\in I}$.
 Test functions for the continuity equation will be chosen in this space
 \begin{subequations}
 \label{Lip-bounded}
\begin{align}
&
\label{Lip-bounded-a}
\Lip_{\mathrm{b}}(V) := \Lip(V) \cap \mathrm{B}_{\mathrm{b}}(V),
\intertext{\RNEW which we consider  with the   norm}
&
\label{Lip-bounded-b}
\RNEW \| \varphi \|_{\Lip_{\mathrm{b}}(V)}: = \sup_{x\in V} |\varphi(x)| +  \sup_{x,y \in V, \ x \neq y} \frac{|\dnabla\varphi\,(x,y)|}{1{\wedge}d(x,y)}\,.
\end{align}
\end{subequations}
\RNEW Observe that the above norm in indeed equivalent to the usual norm for $\Lip_{\mathrm{b}}(V)$, defined as 
 $ \sup_{x\in V} |\varphi(x)| +  \sup_{x \neq y \in V} \frac{|\dnabla\varphi\,(x,y)|}{d(x,y)}$.
 In fact, we have opted for the definition in 
 \eqref{Lip-bounded-b} only to have more transparent estimates, cf.\  Remark \ref{rmk:lip-test-functions} ahead. \EEE
 %see Remark \ref{rmk:lip-test-functions} ahead. \EEE 
We are now in a position to introduce the continuity equation we will work with in this paper.
\begin{definition}[Solutions to the continuity equation]
\label{def:CE}
Let $[a,b] \subset \R$.
We denote by $\CE ab$  the set of pairs $(\rho,\jj)$ such that
\begin{itemize}
    \item[-] $\rho= (\rho_t)_{t\in [a,b]} $ is a family of time-dependent measures in $\mathcal{M}^+(V)$;
    \item[-] $\jj = (\jj_t)_{t\in [a,b]}$ is a measurable family of measures in  $\calM_{\mathrm{loc}}(\Ed)$ such that 
    \begin{equation}
    \label{crucial-bound-j}
\int_a^b \iint_\Ed \RCR (1{\wedge}d(x,y))\EEE \, |\jj_t|(\dd x \dd y)\,\dd t<+\infty
    \end{equation}
    \item[-] the continuity equation holds in this sense: For all $\varphi \in \Lip_{\mathrm{b}}(V) \text{ and all } [s,t]\subset [a,b]$,
    \begin{equation}
    \label{CE}
   \begin{aligned}
		\int_V \varphi(x)\, \rho_t(\dd x)  - \int_V \varphi(x)\, \rho_s(\dd x)   = \int_s^t \iint_\Ed \dnabla\varphi\,(x,y)\,\jj_r(\dd x\dd y)\,\dd r.
 % \\ & \text{ for all } \varphi \in \mathrm{B}_{\mathrm{b}}(V) \cap \mathrm{Lip}(V) \text{ and all } [s,t]\subset [a,b]\,.
  \end{aligned}
	\end{equation}
\end{itemize}	
In what follows, we will use the  short-hand notation $\bj_{\Lebone}$ for the measure on $[a,b]\times E'$ given by 
$\bj_{\Lebone}(A): = \iiint_{A} \bj_t(\dd x \dd y) \, \dd t$ for all $A\in \frB([a,b]{\times} E')$. 
\end{definition}
\RNEW 
\begin{remark}
\label{rmk:lip-test-functions}
\upshape
By our choice of test functions, the integral on the right-hand side of \eqref{CE} is well defined. Indeed, for every $[s,t]\subset [a,b]$ we have 
\begin{equation}
\label{obvious-estimate}
\left| \int_s^t \iint_\Ed \dnabla\varphi\,(x,y)\,\jj_r(\dd x\dd y)\,\dd r \right| \leq \| \varphi \|_{\Lip_{\mathrm{b}}(V)} \int_a^b 
 \iint_\Ed  \cdot (1{\wedge}d(x,y))   \, |\jj_t|(\dd x \dd y)\,\dd t<+\infty
\end{equation}
thanks to \eqref{crucial-bound-j}.
\end{remark} \EEE

%\TODOBS{If we allowed for more general integrability properties of the kernels (see previous red box), then the set of test functions should be 
%\[
%\Lip_{\lambda}(V) =\{ \varphi \in \mathrm{B}_{\mathrm{b}}(V)\, : \ 
%  \sup_{x,y \in V, \ x \neq y} \frac{|\dnabla\varphi\,(x,y)|}{\lambda(x,y)}<+\infty \}\,.
%\]
%Since $\lambda \sim \frac1{d}$ at infinity, in practice we are imposing that 
%\[
%|\dnabla\varphi\,(x,y)| \leq \sim \frac{C}{d} \quad \text{if } d \to \infty...
%\]
%which is definitely very strong. Perhaps, density results are hopeless... which means that in this new setup we would only prove the energy-dissipation inequality....
%}
%\begin{remark}
%\label{rmk:extension}
%\upshape
%In fact, the continuity equation \eqref{CE} could be extended to all functions
%\begin{equation}
%\label{Lipschitz-like}
%\varphi \in \Bb(V) \text{ fulfilling }
%	\|\varphi\|_\lambda := \sup_{x\ne y}\frac{|\varphi(y)-\varphi(x)|}{1{\wedge} \lambda(x,y)} <+\infty\,;
%\end{equation}
%clearly, since $\lambda \geq c_{\lambda} \dV$, any $\varphi \in \Lip_{\mathrm{b}}(V) $ satisfies \eqref{Lipschitz-like}. 
%Indeed, for all  such $\varphi$ 
%we have  that
%\[
%    \left| \int_0^T \iint_\Ed \dnabla\varphi\,(x,y)\,\jj_t(\dd x\dd y)\right| \le
%  \|\varphi\|_\lambda    \int_0^T \iint_\Ed (1{\wedge} \lambda(x,y))\,|\jj_t|(\dd x\dd y) <+\infty
%\]
%thanks to \eqref{crucial-bound}. 
%\end{remark}
 %  \TODO{Should we directly work with the space of functions \eqref{Lipschitz-like}?}


It can easily checked that the concatenation of two solutions to the continuity equation is again a solution to the continuity equation,
and that the family of solutions is closed 
under time rescaling.
\begin{lemma}[Concatenation and time rescaling]
\label{l:concatenation&rescaling}
\begin{enumerate}
\item Let $(\rho^i,\bj^i) \in \CE 0{T_i}$, $i=1,2$, with $\rho_{T_1}^1 = \rho_0^2$. Define $(\rho_t,\bj_t)_{t\in [0,T_{1}+T_2]}$ by 
\[
\rho_t: = \begin{cases}
\rho_t^1 & \text{ if } t \in [0,T_1],
\\
\rho_{t-T_1}^2 & \text{ if } t \in [T_1,T_1+T_2],
\end{cases}
\qquad \qquad 
\bj_t: = \begin{cases}
\bj_t^1 & \text{ if } t \in [0,T_1],
\\
\bj_{t-T_1}^2 & \text{ if } t \in [T_1,T_1+T_2]\,.
\end{cases}
\]
Then, $(\rho,\bj ) \in \CE 0{T_1+T_2}$.
\item
Let $\mathsf{t} : [0,\hat{T}] \to [0,T]$ be strictly increasing and absolutely continuous, with inverse $\mathsf{s}: [0,T]\to [0,\hat{T}]$. Then, $(\rho, \bj ) \in \CE 0T$ if and only if 
$\hat \rho: = \rho \circ \mathsf{t}$ and $\hat \bj : = \mathsf{t}' (\bj  {\circ} \mathsf{t})$  fulfill $(\hat \rho, \hat \bj ) \in \CE 0{\hat T}$. 
\end{enumerate}
\end{lemma}

%\TODO{Instead, we  no longer have that the mapping $t\mapsto \rho_t$ is continuous w.r.t.\ the TV-metric. I expect (but have not checked all details) that 
%\cite[Lemma 4.4]{PRST22}
%should carry over  to this setup by replacing the $\mathrm{TV}$-norm by the following distance construction. Instead, unlike in   \cite[Cor.\ 4.3]{PRST22} we can't extend 
%the CE to all Lip functions}

We consider the  norm on $\calM(V)$ obtained by taking the duality against bounded Lipschitz functions:
\begin{equation}
\label{nbl-norm}
\nbl{\rho}: = \sup\left \{ \left| \int_V \varphi(x) \rho(\dd x ) \right| \, : \varphi \in \Lip_{\mathrm{b}}(V), \RNEW \ \|\varphi\|_{\Lip_{\mathrm{b}}(V)}\leq 1  \EEE  \right\}\,.
\end{equation}

%\TODOBS{It is not even clear if, replacing $ \Lip_{\mathrm{b}}(V)$ by $ \Lip_{\lambda}(V),$ we would get a norm... lack of density results?}
We have the following result, to be compared with \cite[Lemma 4.4]{PRST22}. 
\begin{lemma}
\label{l:continuous-represent}
Let 
$(\rho_t)_{t\in I} \subset \calM^+(V)$ and $(\bj_t)_{t\in I}$ be measurable families
that are integrable with respect to ~$\Lebone$, fulfilling
  \begin{equation}
    -\int_a^b \eta'(t) \left( \int_V \zeta(x) \rho_t (\dd x ) \right) \dd
    t = \int_a^b \eta(t)\Big(\iint_\edg \dnabla\zeta(x,y)\,  \bj_t(\dd
    x\,\dd y)\Big)\,\dd t
    \,,\label{eq:90}
  \end{equation}
  holds for every $\eta \in
  \mathrm{C}_\mathrm{c}^\infty((a,b))$ and $\zeta \in \Lip_{\mathrm{b}}(V)$.
  \par
   Then there exists a unique curve $[a,b] \ni t
  \mapsto \tilde{\rho}_t \in \calM^+ (V)$ such that $\tilde{\rho}_t =
  \rho_t$ for $\Lebone$-a.e. $t\in [a,b]$. The curve $\tilde\rho$ is continuous in the $\nbl{\cdot}$-norm,  fulfills the estimate
  \begin{equation}
  \label{est:ct-eq-BL}	
    \nbl{\tilde \rho_{t_2}-\tilde \rho_{t_1}} \leq 
    \RCR
  \int_{t_1}^{t_2} \iint_{E'}  (1{\wedge}d(x,y)) \,  |\bj_t|(\dd x \dd y)\, \dd t  
  \EEE \qquad
  \text{ for  all }  t_1 \leq t_2 \in [a,b]\,. 
\end{equation}
%  and satisfies
%  \begin{equation}
%    \label{maybe-useful}
%    \int_V \varphi(t_2,\cdot) \,\dd\tilde\rho_{t_2} - \int_V \varphi(t_1,\cdot) \,\dd\tilde\rho_{t_1}
%    = \int_{t_1}^{t_2} \int_V \partial_t \varphi \,\dd\tilde\rho_t\,\dd t
%    +
%    \iint_{J\times \edg} \dnabla \varphi \,\dd \bj_\Lebone
%  \end{equation}
%  for all $\varphi \in \mathrm{C}^1([a,b];\Lip_{\mathrm{b}}(V))$ and $J=[t_1,t_2]\subset
%  [a,b]$.  
\end{lemma}
\RNEW The \emph{proof} follows the very same lines as that of  \cite[Lemma 4.4]{PRST22}, and it is thus omitted; we only remark that 
\eqref{est:ct-eq-BL}	 is an immediate consequence of estimate \eqref{obvious-estimate}. \EEE
%(common dominating measure, continuous representative)  carry over to this setup... however, I imagine that we will be able to sho}






%definition. properties: 

\subsection{The primal and dual dissipation potentials}
Based on Assumptions \ref{Ass:V},  \ref{Ass:D},  and 
\ref{Ass:flux-density},  we rigorously define the primal 
dissipation
 and dual dissipation
 potentials
$
\scrR$ and $\scrR^*$ formally introduced in \eqref{formal_definition}. 
 As in \cite{PRST22}, first of all with any $\rho \in \calM^+(V)$ we associate the couplings
 $ \teta_{\rho}^\pm$ defined 
  on $E'$ by 
\begin{equation}
\label{rig-def:tetarho}
\begin{aligned}
  \teta_{\rho}^-(\dd x\,\dd y) :=
  % {}&
  \rho(\dd x)\kappa(x,\dd y),\qquad
  %=
  %\rho(\delta\otimes\kappa)(\dd x,\dd y),\\
  \teta_{\rho}^+(\dd x\,\dd y) :=
  %{}&
  \rho(\dd y)\kappa(y,\dd x)=
  %=
  %\rho(\kappa\otimes \delta)(\dd x,\dd y)=
  s_{\#}\teta_\rho^-(\dd x\,\dd y),
\end{aligned}
\end{equation}
Observe that $\teta_\rho^\pm \in  \Mloc(E')$. We prove, for later use, the following continuity result, to be compared with \cite[Lemma 2.4]{PRST22}.
\begin{lemma}
\label{l:3.4}
Let $(\rho_n)_n,\rho \in \calM^+(V)$ satisfy $\rho_n\to \rho$ setwise. Then, $\teta_{\rho_n}^\pm \to \tetapi$ vaguely in $\Mloc(E')$. 
\end{lemma}
\begin{proof}
Clearly, it is sufficient to prove the assertion for $(\teta_{\rho_n}^-)_n$. 
Let us fix $f\in \Cc(E')$: then, the map $x \mapsto \int_{V{\setminus}\{x\}} f(x,y) \kappa(x,\dd y)$ is a bounded Borel function on $V$. Indeed, for every $x\in V$ there exists $\eta>0$
such that $\mathrm{supp}(f(x,\cdot)) \subset V{\setminus}B_\eta(x)$, so that 
\[
 \left| \int_{V{\setminus}\{x\}} f(x,y) \kappa(x,\dd y) \right| =\left| \int_{V{\setminus}B_\eta(x)} f(x,y) \kappa(x,\dd y) \right| 
\leq \|f\|_\infty \int_{V{\setminus}B_\eta(x)}  \kappa(x,\dd y) 
\]
with $\|f\|_\infty: = \sup_{(x,y)\in E'} |f(x,y)|$. Therefore,
\[
\begin{aligned}
&
\iint_{E'} f(x,y) \, \teta_{\rho_n}^- (\dd x \dd y) = \int_{V} \left(  \int_{V{\setminus}\{x\}} f(x,y) \,\kappa(x,\dd y) \right) \, \rho_n(\dd x) 
\\
&
\longrightarrow  \int_{V} \left(  \int_{V{\setminus}\{x\}} f(x,y) \,\kappa(x,\dd y) \right) \, \rho(\dd x)  = \iint_{E'} f(x,y) \, \teta_{\rho}^- (\dd x \dd y) \,.
\end{aligned}
\]
\end{proof}
Based on the construction of $\teta_\rho^\pm$, 
 we may rigorously define the measure $\bnu_\rho \in  \Mloc(E')$ from \eqref{formal_definition}
\RNEW  as concave transformation of the couplings $\teta_\rho^\pm$
  (cf.\ construction set forth in \eqref{alpha-mu-gamma}), \EEE namely
\begin{equation}
\label{rig-nu-rho}
\bnu_\rho: =  \Aalpha[\teta_\rho^-,\teta_\rho^+|\tetapi]
\end{equation}
where we have used the simplified notation $\Aalpha[\teta_\rho^-,\teta_\rho^+|\tetapi]$ in place of 
$\Aalpha[(\teta_\rho^-,\teta_\rho^+)|\tetapi]$. 
In the definition of $\scrR$ we will also resort to the construction from \eqref{def:F-F}.
\begin{definition}
\label{def:primal-and-dual}
We define $\scrR: \calM^+(V)\times \Mloc(E') \to [0,+\infty]$ and $\scrR^*: \calM^+(V)\times \Cc(E') \to [0,+\infty)$ via 
\begin{align}
&
\label{rigR}
\scrR(\rho,\bj): =  \frac12 \calF_\uppsi(2\bj |\bnu_\rho)\,,
\\
&
\label{Rig-Rstar}
\scrR^*(\rho,\xi): = \frac12 \iint_{\Ed} \uppsi^*(\xi)  \,\bnu_\rho(\dd x \dd y)\,,
\end{align}
\end{definition}
The following result collects some key facts about $\scrR$. Firstly, we  provide an equivalent representation, cf.\  
\eqref{equivalent-upsilon} below,
for 
$\scrR$ involving the convex and lower semicontinuous function
$\Upsilon: [0,+\infty) \times [0,+\infty) \times \R \to [0,+\infty]$
\begin{align}
\nonumber
&
\Upsilon(u,v,w): = \hat\uppsi (w,\upalpha(u,v))
\intertext{with  $\hat
    \uppsi:\R^{2}\to[0,\infty]$ 
    the 1-homogeneous, convex, perspective function associated  with \EEE
    $\uppsi$ by}
 &
 \nonumber
     \hat \uppsi(z,t):=
      \begin{cases}
        \uppsi(z/t)t&\text{if }t>0,\\
        \uppsi^\infty(z)&\text{if }t=0,\\
        \pinfty&\text{if }t<0.
      \end{cases}
\end{align}
Based on \eqref{equivalent-upsilon} we may prove \eqref{R-lsc}, which extends the sequential lower semicontinuity of $\scrR(\rho, \cdot)$, addressed in \cite[Lemma 4.10]{PRST22},
from setwise 
 to   \emph{vague} convergence. 
\begin{lemma}
The following properties hold:
\begin{enumerate}
\item
For $\scrR$ we have the equivalent representation
\begin{equation}
\label{equivalent-upsilon}
\scrR(\rho,\bj) = \frac12 \calF_\Upsilon ((\teta_\rho^-,\teta_\rho^+,2 \bj)|\tetapi);
\end{equation}
\item if $(\rho,\bj)$ fulfill $\scrR(\rho,\bj)<+\infty$ and $\rho \ll \pi$, then $\bj \ll \tetapi$ and 
\begin{equation}
\label{nice-representation}
\scrR(\rho,\bj)  =  \begin{cases}
    \displaystyle
    \frac12\iint_{E_\upalpha}% {_\upalpha(u)}
    \Psi\Bigl(\frac{w(x,y)}{\upalpha(u(x),u(y))}\Bigr)\upalpha(u(x),u(y))\,\tetapi(\dd
    x,\dd y) &\text{if }|\bj |(\Ed\setminus E_\upalpha)
    % {_\upalpha(u))
    =0,\\
    \pinfty&\text{if }|\bj |(\Ed\setminus E_\upalpha)>0, %_\upalpha(u))>0.
  \end{cases}
\end{equation}
where $w=\frac{\dd( 2\bj)}{\dd \tetapi}$ and $E_\upalpha: = \{ (x,y)\in \Ed\, : \ \upalpha(u(x),u(y))>0\}$;
 \item for all  $(\rho_n)\, \rho \in \calM^+(V)$ and $(\bj_n)_n,\, \bj \in \Mloc(E') $ there holds
 \begin{equation}
 \label{R-lsc}
\left[ \rho_n \to \rho \text{ setwise  and } \bj_n \to \bj \text{ vaguely } \right] \Longrightarrow \liminf_{n\to\infty} \scrR(\rho_n,\bj_n) \geq \scrR(\rho,\bj)\,.
 \end{equation}
 \end{enumerate}
\end{lemma}
\begin{proof}
Formula \eqref{equivalent-upsilon} follows by the same arguments as in \cite[Sec.\ 4.2]{PRST22}; likewise, the proof of the first part of Claim \textbf{(2)}
follows by   \cite[Lemma 4.10]{PRST22}. 
\par
Now, 
when $\rho \ll \pi$ then we also have $\teta_\rho^\pm \ll \tetapi$ and (recalling that $\teta_\rho^+= s_{\#}\teta_\rho^-$)
\[
\frac{\dd \teta_\rho^-}{\dd \tetapi}(x,y) = u(x), \qquad \frac{\dd \teta_\rho^-}{\dd \tetapi}(x,y) = u(y),
\]
so that 
\[
\bnu_\rho(\dd x \, \dd y) = \alpha \left(\frac{\dd \teta_\rho^-}{\dd \tetapi},\frac{\dd \teta_\rho^+}{\dd \tetapi} \right) \,\tetapi(\dd x \dd y) = \alpha(u(x),u(y)) \,  \tetapi(\dd x \dd y) \,.
\]
Since
 $\bj\ll \tetapi$, we can write $2\jj = w \tetapi$, so that 
 \[
\Upsilon \left(\frac{\dd \teta_\rho^-}{\dd \tetapi} ,\frac{\dd \teta_\rho^+}{\dd \tetapi},\frac{\dd 2\jj}{\dd \tetapi} \right)(x,y) =\hat{\uppsi}(w(x,y),\alpha( u(x),u(y)))
\] 
  and \eqref{nice-representation} follows by the definition of the perspective function $\hat\psi$. 
  \par
  Finally, let $(\rho_n)\, \rho \in \calM^+(V)$ and $(\bj_n)_n,\, \bj \in \Mloc(E') $ be as in \eqref{R-lsc}. By Lemma \ref{l:3.4} we infer that
  $(\teta_{\rho_n}^-,\teta_{\rho_n}^+,2\bj_n) \to (\teta_\rho^-,\teta_\rho^+,2\bj)$ vaguely in $\Mloc(E';\R^3)$, and the  assertion follows 
  from Lemma \ref{l:crucial-F}.
\end{proof}


%$ \scrR^*: \calM^+(V)\times \Cc(E') \to [0,+\infty)$.






\subsection{The $\scrR$-$\scrR^*$ formulation}
Prior to specifying our notion of weak solution, we need to property introduce the Fisher information. 
Formally, it is given by 
\[
\Fish(\rho) = \scrR^*\Bigl(\rho,-\thalf \ona \upphi(u)\Bigr)
=
 \frac12
\iint_\edg \Psi^*\bigl( -\thalf(\upphi'(u(y))-\upphi'(u(x))\bigr) \bnu_\rho(\dd x \, \dd y),\qquad \rho = u\pi\, .
\]
However, note that $\upphi$ need not be differentiable at $0$. Hence,  in order to
 rigorously define $\Fish$ in the present context, we mimick the ideas of \cite{PRST22}. Firstly, we introduce the function
$\Lambda_\upphi: \R_+\times \R_+ \to [-\infty,+\infty]$
\[
\Lambda_\upphi (u,v): =\begin{cases}
\phi'(v)-\phi'(u) & \text{if } u,v \in \R_+\times \R_+ \setminus \{ (0,0)\},
\\
0 & \text{if } u=v=0\,
\end{cases}
\]
where we have set $\upphi'(0) := \lim_{r\downarrow 0} \upphi'(r) \in [-\infty,+\infty) $. 
Hence, we define the function $\mathrm{D}_\upphi^+: \R_+\times \R_+ \to [0,+\infty]$ by
\[
\mathrm{D}_\upphi^+(u,v): = \begin{cases}
\uppsi^*\left( \Lambda_\upphi (u,v) \right) \upalpha (u,v) & \text{if } \upalpha (u,v)>0,
\\
0 
& \text{if } u=v=0,
\\
+\infty & \text{if } \upalpha(u,v)=0 \text{ with } u \neq v\,.
\end{cases}
\]
Finally, we consider
\[
\text{the \emph{lower semicontinous envelope} } \mathrm{D}_\upphi: \R_+\times \R_+ \to [0,+\infty]  \text{ of }   \mathrm{D}_\upphi^+\,.
\]
\begin{definition}[Fisher information]
\label{Def:Fisher}
The Fisher information $\Fish: \mathrm{dom}(\scrE)\to [0,+\infty]$ is defined as
\[
\Fish(\rho): = 
\begin{cases}
\frac12 \iint_{E'}  \mathrm{D}_\upphi(u(x),u(y)) \, \tetapi(\dd x \dd y) \qquad \text{if  } \rho = u \pi \text{ and  $ \mathrm{D}_\upphi(u(x),u(y)) \in L^1(E',\tetapi)$,}
\\
+\infty \text{ otherwise.}
\end{cases} 
\]
\end{definition}








We now highlight the crucial lower semicontinuity property of $\mathscr{D}$. 
\begin{proposition}
\label{prop:Fisher-lsc}
Suppose that  \RNEW either $\pi$ is purely atomic, or that the function  $\mathrm{D}_\upphi: \R_+\times \R_+ \to [0,+\infty]$ is convex. \EEE
 Then,
for all  $(\rho_n)_n\, \rho \in \mathrm{dom}(\scrE)$ we have 
\[
\rho_n\to \rho \text{ setwise in } \calM^+(V) \ \Longrightarrow \ \liminf_{n\to\infty} \scrD(\rho_n)\geq \scrD(\rho)\,.
\]
\end{proposition}
\RNEW The \emph{proof} follows the same lines as the argument for \cite[Prop.\ 5.3]{PRST22}, hence it is omitted. \EEE  

%\TODO{I don't see any changes to be
%made to  the proof of Prop.\ 5.3, do you? I think it carries over to the singular setting...}

%\TODO{Is this still true?}

Finally, we are in a position to formalize our concept of solution for the evolution system associated with 
$(V,\kappa,\upphi,\uppsi,\upalpha)$. 
\begin{definition} %[$(\calS,\calR,\calR^*)$ Energy-Dissipation balance]
\label{def:R-Rstar-balance}
%Let the Generalized Gradient Flow $\GGF$ satisfy Property \ref{PROPERTY:chain-rule}. 
We say that a 
curve $\rho: [0,T] \to \calM^+(V)$ is a  solution
of the $(\scrE,\scrR,\scrR^*)$ evolution system,
if it
 satisfies the following conditions:
\begin{enumerate}
%\item \label{def:RR*-1} $t \mapsto \calS(\rho_t)$ is absolutely continuous;
%\item \label{def:RR*-2}
\item $\calS(\rho_0)<\pinfty$;
\item There exists a measurable family $(\bj_t)_{t\in [0,T]} \subset \Mloc(\Ed)$
such that $(\rho,\bj)\in \CE0T$ \RNEW and the pair $(\rho,\bj)$ complies with the  {\em $(\scrE,\scrR,\scrR^*)$
  Energy-Dissipation balance}:  \EEE
\begin{equation}
\label{R-Rstar-balance}
\int_s^t \left( \scrR(\rho_r, \bj_r) + \Fish(\rho_r) \right) \dd r+ \calS(\rho_t)   = \calS(\rho_s)   \qquad \text{for all } 0 \leq s \leq t \leq T.
\end{equation}
\end{enumerate}
\end{definition}


%\TODOBS{Indeed, we should introduce 2 solution concepts, because, - see Theorem \ref{th:main} - we are going to have two distinct existence results.
%More precisely, we should 
% distinguish the above solution concept from the STRONGER one consisting of: (1) $\calS(\rho_0)<\pinfty$; (2) There exists a measurable family $(\bj_t)_{t\in [0,T]} \subset \Mloc(\Ed)$
%such that $(\rho,\bj)\in \CE0T$
%(3) the pair $(\rho, \bj)$ complies with the pointwise Energy-Dissipation balance}

%\TODOBS{We should find names for this notion of solution, and the above one.... and, possibly, be consistent with what we introduced in \cite{PRST22}, but I don't know if
%we'll manage that...
%\\
%Provisionally, I suggest to call `enhanced solutions' the solutions satisfying \eqref{enh-EDB}...
%}

\section{Main result}
\label{s:4}
We will prove our existence result for the evolution system associated with $(V,\kappa,\upphi,\uppsi,\upalpha)$ by approximating the singular
kernels $(\kappa(x,\cdot))_{x\in V}$ via a family of regularized kernels.
More precisely, we define 
\begin{equation}
\label{kappa-eps-reg}
	\kappa_\varepsilon(x,\dd y) := \frac{1\wedge\dpi^2(x,y)}{\varepsilon + (1{\wedge}\dpi^2(x,y))} \kappa(x,\dd y) \qquad \text{for all } x \in V
\end{equation}
%Observe that, while for every $x\in V$ $\kappa(x,\cdot)$ is positive Radon measure on $V\setminus \{x\}$,
%the measure
% $\kappa_\eps(x,\dd y)$ extends to the whole $V$
% for all $x\in V$
 In this way, we clearly obtain a \emph{finite} measure on the whole of $V$, fulfilling  
\begin{equation}
\label{crucial-bound-kappaeps}
\kappa_\varepsilon(x,\cdot) \le \frac1\eps (1{\wedge}\dpi^2(x,\cdot)) \kappa(x,\cdot) \qquad \text{for all } \eps>0.
\end{equation}
%%\TODO{is it trivial that  for every $x\in V$ $\kappa_\eps(x,\dd y)$ can be considered as a measure on the whole of $V$? I think that it should follow
%%from the fact that we consider $1\wedge\lambda^2(x,\cdot)  \kappa(x,\dd \cdot) $ as a measure on the whole of $V$..
%%is this an obvious consequence of \eqref{mitigation of singularity}?}
%Hence,
%\[
%\lim_{r\to 0} \int_{B_r(x)} \kappa^\varepsilon(x,\dd y)  \leq \frac1\eps \lim_{r\to 0} \int_{B_r(x)} (1{\wedge}\lambda^2(x,y)) \kappa(x,\dd y ) \leq C\,.
%\] 
%All in all, combining \eqref{crucial-bound-kappaeps} and 
Thus, by 
\eqref{mitigation of singularity} we infer that 
\begin{equation}
\label{back-2-bounded-kernel}
\|\kappa_\eps\|_\infty : = \sup_{x \in V} \kappa_\eps(x,V) \leq \frac1\eps c_\kappa\,.
\end{equation}
For later use, we also observe that,  since $\kappa_\eps (x, \cdot) \leq \kappa (x,\cdot)$, there holds
\begin{equation}
\label{mitigated-bound}
\sup_{x\in V} \int_{V}  (1{\wedge}\dpi^2(x,y)) \kappa_\eps(x,\dd y) \leq c_\kappa\,.
\end{equation}
We will denote by $\tetapie$ the induced coupling 
\begin{equation}
\label{tetapie}
\tetapie (\dd x \dd y) =\kappa_\eps(x, \dd y  ) \pi (\dd x );
\end{equation}
 observe that, now, $\tetapie \in \calM(E)$ and, clearly, it still satisfies the detailed balance \eqref{DBC}. 
\par
Let $\scrR_\eps: \calM^+(V) \times \calM(E) \to [0,+\infty]$ and $\scrR_\eps^*  : \calM^+(V)\times \Cb(E) \to [0,+\infty)$  the primal and dual  dissipation potentials associated with $\tetapie$ via Definition \ref{def:primal-and-dual}
(with  $\iint_{\Ed}$ replaced  by $\iint_{E}$), and let $\scrD_\eps$ the corresponding Fisher information functional 
\RNEW
\[
\Fish_\eps(\rho): = 
\begin{cases}
\frac12 \iint_{E}  \mathrm{D}_\upphi(u(x),u(y)) \, \tetapie(\dd x \dd y) \qquad \text{if  } \rho = u \pi \text{ and  $ \mathrm{D}_\upphi(u(x),u(y)) \in L^1(E,\tetapie)$,}
\\
+\infty \text{ otherwise.}
\end{cases} 
\]
\EEE
%clearly, in the definitions of $\scrR_\eps$ and  $\scrD_{\eps}$ we may replace $\iint_{\Ed}$ by $\iint_{E}$. 
We may thus consider the $(\scrE,\scrR_\eps,\scrR_\eps^*)$ energy-dissipation balance
modelled on \eqref{R-Rstar-balance}. 
\par
Under  Assumptions \ref{Ass:V},  \ref{Ass:E},   \ref{Ass:D},  and \ref{Ass:flux-density}, also 
in view of \eqref{back-2-bounded-kernel} 
 the  regularized  system $(V,\kappa_\eps,\upphi,\uppsi,\upalpha)$ complies with the assumptions of \cite[Thm.\ 5.7]{PRST22},
  ensuring the existence of 
a  solution to the   $(\scrE,\scrR_\eps,\scrR_\eps^*)$ evolution system in the sense of Def.\ \ref{def:R-Rstar-balance}. 
We point out that such solution in fact fulfills the continuity equation in an enhanced   sense, 
namely  \eqref{CE} holds, along any sub-interval $ [s,t]\subset [a,b]$, for all test functions $\varphi \in \Bb(V)$. 
% in particular, 
%the set of test functions for \eqref{CE} extends to $\Bb(V)$. 
 %which we recall in the following
%\begin{definition}
%\label{def:CE-enh}
%For a given $[a,b] \subset \R$.
%We denote by $\ENHCE ab$  the set of pairs $(\rho,\jj)$ such that $(\rho_t)_{t\in [a,b]} $ is a family of time-dependent measures in $\mathcal{M}^+(V)$ and
%\begin{itemize}
%\item[-] $\jj = (\jj_t)_{t\in [a,b]}$ is a measurable family of measures in  $\calM(E)$ such that 
%    \begin{equation}
%    \label{crucial-bound}
%\int_0^T \iint_E |\jj_t|(\dd x \dd y)\,\dd t<+\infty
%    \end{equation}
%    \item[-] the continuity equation \eqref{CE} holds, along any sub-interval $ [s,t]\subset [a,b]$, for all test functions $\varphi \in \Bb(V)$. 
%\end{itemize}	
%\end{definition}
%%%
%We setup the approximation and recall the existence result from [PRST]. with bounds. 
\RNEW
Besides stating the existence of solutions, the following result also subsumes the maximum principle proved in \cite[Thm.\ 6.5]{PRST22}. 
   \EEE
\begin{theorem}{\cite[Thm.\ 5.10, Thm.\ 6.5]{PRST22}}
\RNEW Let  $(V,\kappa,\upphi,\uppsi,\upalpha)$ comply with
 Assumptions \ref{Ass:V},  \ref{Ass:E},   \ref{Ass:D}, and \ref{Ass:flux-density}.
In addition, suppose that 
either $\pi$ is atomic, or the function 
$\mathrm{D}_\upphi: \R_+\times \R_+ \to [0,+\infty]  $
 is convex. \EEE
 Then,
 \begin{enumerate}
 \item For every $\rho^0\in \calM^+(V)$ with \RNEW  $\scrE(\rho^0)<+\infty$ \EEE there exists a curve $\rho: [0,T]\to \calM^+(V)$, with $\rho(0)=\rho^0$, solving the  $(\scrE,\scrR_\eps,\scrR_\eps^*)$ evolution system, i.e.\ there exists a measurable family $\jj = (\jj_t)_{t\in [a,b]} \subset \calM(E)$, with 
 $ (\rho,\jj) \in  \ENHCE 0T$, such that the pair  $ (\rho,\jj) $
 \RBS is an \emph{enhanced} solution of the 
  the $(\scrE,\scrR_\eps,\scrR_\eps^*)$ system. \EEE
 \item If  $\rho^0 = u^0 \pi$ with $u^0 \in L^\infty(V,\pi)$ such that
 \begin{equation}
 \exists\, 0 \leq  \underline{U} < \overline{U} \ \text{ for $\pi$-a.a.\ $x\in V$:} \qquad 
  0 \leq \underline{U} \leq u^0(x) \leq \overline{U}\,,
  \end{equation}
  then 
  $\rho = u \pi$ satisfies
  \begin{equation}
  \label{boundedness}
   0 \leq \underline{U} \leq u_t(x) \leq \overline{U} \qquad \text{for } (\scrL{\times}\pi)\text{-a.a. } (t,x) \in (0,T){\times}V\,.
  \end{equation}
 \end{enumerate}
\end{theorem}
\par
\paragraph{\bf Approximation of solutions to the singular system.}
%\TODO{Here  introduce the Orlicz space  $\rmL^{\upphi^*}(V,\pi)$and state our last assumption, namely the density of $\mathrm{Lip}_{\mathrm{b}}(V) $ in $\rmL^{\upphi^*}(V,\pi)$..}
%%%
 Let $(\ej)_{n\in \N} \subset (0,+\infty)$ be a null sequence. Correspondingly, let 
 $(\rho^{\ej},\jj^{\ej})_n \subset  \CE 0T$,
 \RCR satisfying the continuity equation for all test functions $\varphi \in \Bb(V)$, \EEE
  be solutions of the   $(\scrE,\scrR_{\ej},\scrR_{\ej}^*)$  evolution systems starting from initial data $(\rho_n^0)_n
 \subset \calM^+(V)$ satisfying suitable conditions.
 With the main result of this paper we are going to prove 
that, up to a subsequence, the pairs $(\rho^{\ej},\jj^{\ej})_n $ converge as $n\to\infty$ to a solution of the   $(\scrE,\scrR,\scrR^*)$  system.
We recall that, denoting $\Lebone$ the Lebesgue measure $\Lebone|_{(0,T)}$, a measurable family $\nu = (\nu_t)_{t\in [0,T]} \in \Mloc(Y)$,
with $Y\in \{ V,E, \Ed\}$, induces a measure in $\Mloc([0,T]{\times} Y)$ by setting $\nu_\Lebone (\dd t \dd x \dd y): = \nu_t(\dd x \dd y) \,\Lebone (\dd t)$. 

\par
 Before giving the main result of this paper,  we need to introduce one more condition, besides  Assumptions  \ref{Ass:V},  \ref{Ass:E},  \ref{Ass:D}, and \ref{Ass:flux-density}. 
 \RCR Prior to stating it, let us settle some preliminary definitions.
  %
% \TODOR{At this point, the argument bifurcates. There are two possible set of assumptions, corresponding to two possible arguments for the EXTENDED continuity equation. Now I am going to present both options. We need to decide if we want to present both of them....}
%%
%\noindent 
% \RCR \textbf{Option $1$:}
%In order to state our additional assumption, 
% we consider 
% the subset 
%   \begin{equation}
%   \label{X-subset}
%    \mathcal{X} := \left\{ \varphi\in \rmL^{1}(V,\pi) : \iint_\Ed \uppsi^*(\dnabla\varphi(x,y))\,  \tetapi(\dd x \dd y)  <\infty \right\} \subset \rmL^{1}(V,\pi)\,.
%   \end{equation}
%%   \TODO{Is $    \mathcal{X} $ a 
%%   reflexive Banach space???
%%This  is not clear to me...   %  \mathcal{X}$ is closed under convex combinations, but all linear combinations? 
%%    hence it is not clear to me if it is meaningful to 
%%   equip it  with the norm
%%   \[
%%    \|\varphi\|_\mathcal{X} := \|\varphi\|_{\rmL^{\upphi^*}(V;\pi)} + \|\dnabla\varphi\|_{\rmL^{\uppsi^*}(\Ed;\tetapi)}.
%%   \]
%%   In Section 6 I have tried to avoid referring to the Banach structure of $\mathcal{X}$....
%%   }
%We are now in a position to state our last condition for Theorem \ref{th:main}; we postpone a discussion of its validity to  Section \ref{s:lip}. 
%%%%%t
% \begin{assumption}
% \label{Ass:F}
% We require that 
% \begin{enumerate}
% \item
% The space $\Lip_{\mathrm{b}}(V) $ is  sequentially dense in $\mathcal{X} \cap \Bb(V)$ w.r.t.\ the 
% weak topology of $\rmL^{1}(V,\pi)$;
% \item $\mathcal{X} $ is sequentially  dense in $\rmL^{1}(V,\pi)$ w.r.t.\ the weak topology of  $\rmL^{1}(V,\pi)$.
%\end{enumerate}
% \end{assumption}   
% \EEE
% \TODOR{By the way, I don't think that it would suffice (without further assumptions), to directly require that   $\Lip(V) $ is sequentially dense in $\rmL^{1}(V,\pi)$...}
% 
 \noindent 
 \RCR 
  First of all, on the measure space 
  $(
 \Ed, \mathfrak{B}(\Ed), \tetapi  )$
  we introduce the \emph{Orlicz space}    associated with the \emph{Young} function $\uppsi^*$, namely
\[
 \rmL^{\uppsi^*}(\Ed;\tetapi): = \biggl\{ y \in \rmL^1(\Ed;\tetapi)\, : \ \exists\, \ell> 0 \ \iint_{\Ed} \uppsi^*\left(\frac{y(x,y)}{\ell}\right) \, \tetapi(\dd x \dd y) <+\infty\biggr\}\,,
 \]
with the associated Luxemburg norm
\[
\|y \|_{\rmL^{\uppsi^*}(\Ed;\tetapi)} : = \inf\left\{ \ell > 0 \, : \   \iint_{\Ed} \uppsi^*\left(\frac{y(x,y)}{\ell}\right) \, \tetapi(\dd x \dd y)  \leq 1\right \}\,;
\]
 we refer to, e.g.,  \cite{RaoRen} for a comprehensive  presentation of Orlicz spaces. We will also consider the space
 \begin{equation}
 \label{X-L-Orli}
 \rmX^{\uppsi^*}: = \Bigl\{ \varphi \in \Bb(V)\,: \ \overline{\nabla}\varphi \in \rmL^{\uppsi^*}(\Ed;\tetapi)\Bigr\}\,.
 \end{equation}
 Furthermore, let us  introduce the 
 %\WARNING what is the name of 
 %this space?? 
 \RCR \emph{small Orlicz} space
\[
 \mathcal{M}^{\uppsi^*}(\Ed;\tetapi): = \biggl\{ y \in \rmL^1(\Ed;\tetapi)\, : \ \forall\, \ell> 0 \ \iint_{\Ed} \uppsi^*\left(\frac{y(x,y)}{\ell}\right) \, \tetapi(\dd x \dd y) <+\infty\biggr\}\\,
 \]
and, accordingly,  
  \begin{equation}
 \label{X-M-Orli}
 \mathcal{X}^{\uppsi^*}: = \Bigl\{ \varphi \in \Bb(V)\,: \ \overline{\nabla}\varphi \in \mathcal{M}^{\uppsi^*}(\Ed;\tetapi)\Bigr\}\,.
 \end{equation}
Clearly, $\mathcal{M}^{\uppsi^*}(V;\pi)$ is a  subspace of $\rmL^{\uppsi^*}(\Ed;\tetapi)$, so that $ \mathcal{X}^{\uppsi^*}\subset \rmX^{\uppsi^*}$. 
 If, in addition,
the Young function satisfies the so-called $\Delta_2$ condition (cf.\ \cite[\S 2.3, Definition 1]{RaoRen}), 
then $\mathcal{M}^{\uppsi^*}(V;\pi) =\rmL^{\uppsi^*}(\Ed;\tetapi)$.
We are not going to impose $\Delta_2$-regularity on $\uppsi^*$ but, still, we will be able to show that, in the present context, the spaces
 $ \rmX^{\uppsi^*}$ and $ \mathcal{X}^{\uppsi^*}$ coincide (cf.\ Corollary \ref{cor:identical-spaces}). 
 %
%by 
%  slightly strengthening our conditions on $\uppsi^*$. Namely, we  require that, in addition to 
%  Assumption \ref{Ass:D} $\uppsi^*$ complies with
%  \begin{assumption}
% \label{Ass:D-additional}
% The  dual dissipation density  $\uppsi^*: \R \to [0,+\infty)$ is  non-decreasing and $\uppsi^*(\xi) \simeq |\xi|^2 
% $ as $\xi \to 0$, i.e.
% \begin{equation}
% \label{quadratic-at-0}
% \lim_{\xi \to 0} \frac{\uppsi^*(\xi)}{|\xi|^2} =c_0 \in ]0,+\infty[\,.
% \end{equation}
% \end{assumption}
We will also show  in 
Lemma \ref{l:Lipb-r-nice}
that, for every $\varphi 
 \in \Lipb (V)$, $\overline\nabla \varphi 
 \in  \mathcal{M}^{\uppsi^*}(\Ed;\tetapi)$, so that $\Lipb(V)\subset   \mathcal{X}^{\uppsi^*}$.
 This makes  our additional assumption  meaningful.
  \begin{assumption}
 \label{Ass:F-bis} 
 The space $\Lipb(V)$ is dense in $\mathcal{X}^{\uppsi^*}$ in the following sense: for every $\varphi \in \mathcal{X}^{\uppsi^*}$
 there exists a sequence $(\varphi_n)_n \subset \Lipb(V)$ such that as $n\to \infty$
 \begin{enumerate}
 \item $\varphi_n\to\varphi$ in $\rmL^1(V;\pi)$;
 \item $\overline\nabla \varphi_n \to \overline\nabla \varphi$ in $\rmL^{\uppsi^*}(\Ed;\tetapi)$.
 \end{enumerate}
 \end{assumption}
\noindent
 We postpone a discussion on 
the  validity of Assumption \ref{Ass:F-bis} 
  to  Section \ref{s:lip}. 
 \TODOR{\EEE I have had to strengthen the previous weak convergence in $
\rmL^1(V)$ to  strong  convergence in view of the cut-off argument of Lemma \ref{l:cutoff} but, if we decide to kill it, then we can go back to weak convergence...}
\EEE

 
% \TODOR{First of all, what we really need is that $\Lip_{\mathrm{b}}(V) $ is dense in $\mathcal{X} \cap \Bb(V)$  w.r.t.\ the topology induced by $\|\cdot\|_{ \rmL^{\upphi^*}(V;\pi)} $.
% For this, would be sufficient to have that  $\Lip(V) $ is dense in $\mathcal{X} $?????  I would expect so...
% 
% Moreover,
%  based on the examples that we can make, wouldn't it make more
% sense to directly require that  $\Lip(V) $ is dense in  $ \rmL^{\upphi^*}(V;\pi)$???} \EEE
 %%%
 \par
 We are in a position to state  the main result of this paper.
\begin{theorem}
\label{th:main}
Under  Assumptions \ref{Ass:V},  \ref{Ass:E},  \ref{Ass:D}, \ref{Ass:flux-density}, and  \RCR \ref{Ass:F-bis},
 %\ref{Ass:F}, 
   suppose that 
%either $\pi$ is atomic, or
 the function 
$\mathrm{D}_\upphi: \R_+\times \R_+ \to [0,+\infty]  $
 is convex.  \EEE
 Let $(\rho_n^0)_n, \rho^0
 \in \calM^+(V)$ satisfy %as $n\to\infty$
 \begin{subequations}
 \label{initial-thm:main}
 \begin{equation}
 \label{initial-conds-rhoj}
 \begin{cases}
% &
 %\displaystyle \sup_{n \in \N} \scrE(\rho_n^0) \doteq \mathsf{E}_0<+\infty, 
 &
 \rho_n^0 \to \rho \text{ setwise in } \calM^+(V) \text{ as } n \to \infty\,,
 \\
 & 
 \RNEW  \scrE(\rho_n^0)  \to \scrE(\rho)  \text{ as } n \to \infty\,. \EEE
 \end{cases}
 \end{equation}
 and
  \begin{equation}
  \label{strong-initial}
  \exists\, 0 \leq \underline{U} < \overline{U} \quad \forall\, n \in \N \text{ for $\pi$-a.a.\ $x\in V$:} \qquad 
  \underline{U} \leq u_n^0(x) \leq \overline{U}
\end{equation}
\end{subequations}
 Then,
 \begin{enumerate}
 \item
  There exist $(\rho,\bj) \in \CE 0T$ and (a not relabeled) subsequence such that the following convergences hold as $n\to\infty$
\begin{align}
\label{DBL}
&
\rho_t^{\ej}\to \rho_t  && \text{setwise in } \calM^+(V) \quad \text{for all } t\in [0,T];
\\
& 
\label{cvg-j}
\bj_{\Lebone}^{\ej}\to \bj_\Lebone \RNEW =\bj_t \Lebone \EEE  && \text{vaguely in } \Mloc([0,T]{\times}\Ed);
\end{align}
the measure $\rho$ satisfies
$\rho_t  = u_t \pi$  for all $t\in [0,T]$, 
for some $u\in \rmL^1(0,T;\rmL^1(V;\pi))$ such that 
\begin{equation}
\label{inherited-bounds-for-u}
\underline U \leq u(t,x) \leq \overline U   \text{ for $(\scrL{\otimes}\pi)$-a.a.\ $(t,x)\in (0,T) \times V$},
\end{equation}
and the pair  $(\rho,\bj)$ is \RBS  a  solution of the $(\scrE,\scrR,\scrR^*)$ evolution system in the sense of Definition \ref{def:R-Rstar-balance}. 
\item
\RCR If, in addition, \eqref{strong-initial} holds with a constant $\underline{U}>0$, then   $(\rho,\bj)$ is an \emph{enhanced} solution of the $(\scrE,\scrR,\scrR^*)$ evolution system,  namely 
the Energy-Dissipation balance \eqref{R-Rstar-balance} holds in pointwise form
\begin{equation}
\label{enh-EDB}
\scrR(\rho_t, \bj_t) + \Fish(\rho_t) = -\frac{\dd }{\dd t}  \calS(\rho_t)    = -\iint_{E'}\overline\nabla (\phi'(u_t) )\,  \bj_t (\dd x \dd y)  \qquad \foraa\, t \in (0,T).
\end{equation}
\end{enumerate}
 \EEE
\end{theorem}
In fact, as it will be clear from the proof, in case \textbf{(2)} we will also be able to show that  $(\rho,\bj)$ solves the continuity equation
in an enhanced sense, namely
 for a bigger set of test functions than $\Lipb(V)$, cf.\ Proposition \ref{prop:ENH-CE}. 
 %%
\TODOR{At this point we might add a remark on the difference between \eqref{R-Rstar-balance}  and \eqref{enh-EDB}}
\EEE

%
%\TODOH{We should add  a second result: if we replace \eqref{strong-initial} by}
%\RBS  \begin{equation}
%  \label{weak-initial}
%  \exists\, 0  <\overline{U} \quad  \forall\, n \in \N \text{ for $\pi$-a.a.\ $x\in V$:} \qquad 
%  0 \leq u_n^0(x) \leq \overline{U}
%\end{equation} \EEE
%\TODOH{then we obtain solutions in the sense of Def.\ \ref{def:R-Rstar-balance}.}
%Main theorem: convergence to a solution of the $R$-$R^*$ formulation.
\paragraph{\bf 
Outline of the proof.} In Sec.\ \ref{s:5} we are going to prove  convergences \eqref{DBL}\&\eqref{cvg-j} to a pair $(\rho,\bj):[0,T]\to \calM^+(V)\times \Mloc(\Ed)$ satisfying the 
\begin{description}
\item[\textbf{upper energy-dissipation estimate}]
\begin{equation}
\label{UEDE}
\int_0^t \left( \scrR(\rho_r, \bj_r) + \Fish(\rho_r) \right) \dd r+ \calS(\rho_t)   \leq \calS(\rho_0)   \qquad \text{Efor all } t\in [0,T]\,.
\end{equation}
\end{description}
We will conclude the proof in Sec.\ \ref{s:6} by showing that the curve $(\rho,\bj) $ satisfies the continuity equation in the sense of 
Def.\ \ref{def:CE}  and fulfills the 
\begin{description}
\item[\textbf{lower energy-dissipation estimate}]
\begin{equation}
\label{LEDE}
\int_0^t \left( \scrR(\rho_r, \bj_r) + \Fish(\rho_r) \right) \dd r+ \calS(\rho_t)   \geq \calS(\rho_0)   \qquad \text{for all } t\in [0,T]\,.
\end{equation}
\end{description}
%%%%%
%%%%%%
\section{Compactness and Upper energy-dissipation inequality}
\label{s:5}
\RNEW Preliminarily,  for the reader's convenience  we recall a refined version of the Ascoli-Arzel\`a theorem in metric spafes that will be used later on.
\begin{theorem}{\cite[Prop.\ 3.3.1]{AGS08}}
\label{thm:Ascoli}
Let $(\mathscr{S}, \mathsf{d})$ be a complete  metric space, also endowed with a topology $\serifsigma$ compatible with $\mathsf{d}$ in the sense that
for all $(x_n)_n,\, (y_n)_n \subset \mathscr{S}$ there holds
\begin{equation}
\label{compatibility-top-metr}
(x_n,y_n) \stackrel{\serifsigma}{\longrightarrow} (x,y) \ \Longrightarrow  \ \liminf_{n\to\infty} \mathsf{d}(x_n,y_n) \geq \mathsf{d}(x,y)\,. 
\end{equation}
Let $\mathsf{K}$ be a $\serifsigma$-sequentially compact subset of  $\mathscr{S}$, and let
$(v_n)_n$
be a sequence of curves  $v_n : [0,T]\to \mathscr{S}  $ such that 
\begin{subequations}
\label{conditions-for-Ascoli}
\begin{align}
\label{compactness}
& v_n(t) \in \mathsf{K} && \text{for all } t \in [0,T], \ n \in \N,
\\
& 
\label{equicontinuity}
\limsup_{n\to\infty} \mathsf{d}(v_n(t),v_n(s)) \leq \omega(s,t) && \text{for all } s,\, t \in [0,T]
\end{align}
\end{subequations}
where  $\omega : [0,T]{\times}[0,T]\to [0,\infty)$ is a   symmetric function for which there exists an (at most) countable subset $\mathscr{C}$ of $[0,T]$ such that
\[
\lim_{(s,t) \to (r,r)}\omega(s,t) = 0 \quad \text{for all } r\in [0,T]{\setminus}\mathscr{C}.
\]
Then, there exist an increasing subsequence $(n_k)_k$ and a limit curve $v: [0,T]\to \mathscr{S}$ such that 
\[
v_{n_k}(t) \stackrel{\serifsigma}{\to} v(t) \text{ for all } t \in [0,T] \ \text{ and } \ v: [0,T]\to \mathscr{S} \text{ is } \mathsf{d}\text{-continuous on } [0,T]{\setminus}\mathscr{C}\,. 
\]
\end{theorem}








\par
Furthermore, 
we record an observation that will be used several times in what follows.

%\TODO{In fact, in the statement below we might just have setw cvg in $\calM([0,T]{\times}E)$ for $(\eeta_n)$, see the upcoming comments}
\begin{lemma}
\label{l:obvious-useful}
Let $(\eeta_n)_n,\eeta \subset   \calM([0,T]{\times}E)$ fulfill
\[
\eeta_n \to \eeta \qquad \text{setwise in }  \RNEW \calM([0,T]{\times}E)\,, \RNEW
\]
and set $\zzeta_n: =  \frac1{1{\wedge}\dpi} \eeta_n$.
%for some  Borel function $\ell \colon E \to [0,\infty)$  such that $\lim_{d(x,y) \to \infty} \ell(x,y) =\infty$. 
 Then,
\[
\zzeta_n \to \zzeta  :=\frac1{1{\wedge}\dpi} \eeta  \qquad \text{vaguely in }  \calM([0,T]{\times}E')\,.
\]
\end{lemma}
\begin{proof}
For the measures $\zzeta_n$
 let us consider cylindrical  test functions $\varphi$, i.e.,
  $\varphi(t,x,y) = \phi(t) \gamma(x,y)$ with  $\eta\in \Cc([0,T])$ and $\gamma \in \Cc(E')$.
 We have
\[
\begin{aligned}
\iiint_{[0,T]{\times}\Ed} \varphi\, \zzeta_n(\dd t \dd x \dd t ) &  = 
\iiint_{[0,T]{\times}\Ed} \phi(t)  \frac{\gamma(x,y)}{1{\wedge}\dpi(x,y)} (1{\wedge}\dpi(x,y)) \,  \zzeta_n(\dd t \dd x \dd t )
\\
& = \iiint_{[0,T]{\times}\Ed} \phi(t)  \frac{\gamma(x,y)}{1{\wedge}\dpi(x,y)}\, \eeta_n (\dd t \dd x \dd t )
\\
& \stackrel{(1)}{\longrightarrow}
\iiint_{[0,T]{\times}\Ed} \phi(t)  \frac{\gamma(x,y)}{1{\wedge}\dpi(x,y)} \eeta(\dd t \dd x \dd t )
\\
& \qquad = \iiint_{[0,T]{\times}\Ed} \varphi\,  \zzeta(\dd t \dd x \dd t ) 
\end{aligned}
\]
where  convergence 
{\footnotesize (1)} follows from the fact that $\gamma$ has compact support in $\Ed$ and thus 
 $\gamma/(1{\wedge}\dpi) $ extends to a function in $ \Bb(E)$. 
\end{proof}
\EEE
%%%
\subsection{Compactness}
\label{ss:compactness}
%%
We now address the  compactness properties of the sequence  $(\rho^{\ej},\bj^{\ej})_n $.
With the short-hand notation $ \tetapien : = \teta_{\kappa_{\eps_n}}$,
we write   
\begin{equation}
\label{absolute-continuity-n}
\rho^{\ej} = u^{\ej} \pi
\text{ and  } 2\bj^{\ej} = w^{\ej} \tetapien 
\end{equation}
 (since $\rho_t^{\ej} \ll \pi$ and $\bj_t^{\ej} \ll \tetapien$
for $\scrL$-a.a.\ $t\in (0,T)$, by 
 \eqref{nice-representation})
for some $(u^{\ej})_n  \subset
\rmL^1(0,T; \rmL^1(V,\pi)) $ and  \RNEW $(w^{\ej})_n \subset \rmL^1(0,T; \rmL^1(E,\tetapien)) $.   \EEE
From the $(\scrE,\scrR_{\eps_n},\scrR_{\ej}^*)$  energy-dissipation balance written on the interval $[0,t]$ for any $t\in [0,T]$
\begin{equation}
\label{EDB-n}
\int_0^t \left( \scrR^{\ej}(\rho_r^{\ej}, \bj_r^{\ej}) + \Fish^{\ej}(\rho_r^{\ej}) \right) \dd r+ \calS(\rho_t^{\ej})   \leq \calS(\rho^0_n) 
\end{equation}
  it immediately follows that 
 \begin{align}
  \label{energy-bounded}
  &
 \sup_{n\in \N }\sup_{t\in [0,T]} \scrE(\rho_t^{\ej})    \leq \sup_{n\in \N }\scrE(\rho_n^0) =\mathsf{E}_0,
 \\
 \label{action-bounded}
 &
 \sup_{n\in \N} \int_0^T \scrR_{\eps_n}(\rho^{\ej}, \bj^{\ej}) \dd t \leq \mathsf{E}_0\,,
 \\
  \label{Fisher-bounded}
 &
 \sup_{n\in \N} \int_0^T \scrD_{\eps_n}(\rho^{\ej}) \dd t \leq \mathsf{E}_0\,,
 \end{align}
\par
For later convenience we remark that the action 
integrals $ \int_0^T \scrR_{\eps_n}(\rho^{\ej}, \bj^{\ej}) \dd t $ rewrite
taking into account that, by  \eqref{action-bounded}
 we have for $\scrL$-a.a.\ $t\in (0,T)$
 \begin{equation}
 \label{def-set-Eeps}
 |\bj_t^{\ej} |(E\setminus E_t^{\ej}) =0 \qquad \text{with }  E_t^{\ej} =  \{ (x,y)\in E\, : \ \upalpha(u_t^{\ej}(x),u_t^{\ej}(y))>0\}
 \end{equation}
 and hence
 \[
 \scrR(\rho_t^{\ej},\bj_t^{\ej})  = 
    \displaystyle
    \frac12\iint_{E_t^{\ej}}% {_\upalpha(u)}
    \uppsi\Bigl(\frac{w_t^{\ej}(x,y)}{\upalpha(u_t^{\ej}(x),u_t^{\ej}(y))}\Bigr)\upalpha(u_t^{\ej}(x),u_t^{\ej}(y))\,\tetapien(\dd
    x,\dd y) \EEE 
     \quad \text{for $\scrL$-a.a.\ $t\in (0,T)$.} 
 \] 
 \RBS We emphasize that the compactness result below holds under the sole conditions \eqref{initial-thm:main}. In particular, we may allow for the constant $\underline{U}$ in 
 \eqref{strong-initial}  to be zero. \EEE
\begin{proposition}[Compactness]
\label{prop:compactness}
There exist  
a curve $\rho: [0,T]\to \calM^+(V)$, continuous in the $\nbl{\cdot}$-norm, \RNEW a curve  $u \in \rmL^\infty(0,T; \rmL^\infty(V; \pi))$
satisfying \eqref{inherited-bounds-for-u}, 
and  $\bj_\Lebone \in \Mloc([0,T]{\times}\Ed)$, 
such that 
\begin{equation}
\label{disintegration-props-rho-j}
\begin{cases}
\rho_t(\dd x) = u_t\,   \pi(\dd x)  & \text{for } \Lebone\text{-a.a.\ } t \in (0,T),
\\
\bj_\Lebone(\dd t \dd x \dd y) = \bj_t(\dd x \dd y) \,\Lebone(\dd t) & \text{for a measurable family } \bj = (\bj_t)_{t\in[0,T]} \subset \Mloc(E')
\end{cases}
\end{equation}
such that, along a (not relabeled) subsequence, convergences \eqref{DBL} and \eqref{cvg-j} hold. Moreover, the curves $(u^{\ej})_n$ from 
\eqref{absolute-continuity-n} fulfill
\begin{equation}
\label{added-cvg-u}
u^{\ej} \weaksto u \quad \text{in }  \rmL^\infty(0,T; \rmL^\infty(V; \pi)).
\end{equation}
% as $n\to\infty$:
\end{proposition}
%\TODO{In the proof of Prop.\ \ref{prop:compactness} I have used  $\underline{U} \leq u_n^0 \leq \overline{U}$ to simplify the proof, but it shouldn't be necessary..  }
\begin{proof}
Clearly, by \eqref{initial-conds-rhoj} we have that $\sup_n  \rho_n^0(V)  \leq M_1$ for some $M_1>0$. Now, 
from the continuity equation we deduce the mass conservation property
\[
\rho_t^{\ej}(V) = \rho_0^{\ej}(V) = \rho_n^0(V)  \qquad \text{for all } t \in [0,T] \text{ and } j \in \N.
\]
In fact,  \RNEW by \eqref{boundedness}  \EEE  we also have 
\begin{equation}
\label{boundedness-un}
0 \leq  \underline{U} \leq u_t^{\ej}(x) \leq \overline{U} \qquad \text{for } (\scrL{\times}\pi)\text{-a.a. } (t,x) \in (0,T){\times}V\,,
\end{equation}
\RNEW so that there exists $u \in   \rmL^\infty(0,T; \rmL^\infty(V; \pi))$  such that convergence \eqref{added-cvg-u} holds (along a not relabeled subsequence). 
Clearly, $u$ inherits the bounds \eqref{boundedness-un}.
\EEE
We will now address the proof of some claims. 
\medskip
\paragraph{\bf Claim $1$:}
{\sl Consider the families of measures $\bg_{\Lebone}^{\ej} (\dd x \dd y \dd t) =   \bg_t^{\ej}(\dd x \dd y) \Lebone(\dd t) \in  \calM([0,T]{\times}E)$, with 
$\bg_t^{\ej}:=(1\wedge \dpi)\, \jj_t^{\ej}.$ Then,
there exists \RNEW  $\bG \in \calM((0,T){\times}E)$ \EEE such that, up to a not relabeled subsequence, there holds as $n\to\infty$}
\begin{equation}
\label{setwise-g}
 \bg_{\Lebone}^{\ej}  \to \bG \qquad \text{setwise  in }    \RNEW \calM([0,T]{\times}E)\,. \EEE 
\end{equation}
{\sl The measure  $\bG$ can be disintegrated as $\bG(\dd t\,\dd x\,\dd y)= \bg_\Lebone(\dd t\,\dd x\,\dd y)
=  \bg_t(\dd x\,\dd y)\,\Lebone (\dd t) $  \EEE for a measurable family $ \bg = (\bg_t)_{t\in[0,T]} \subset \calM(E).$}
\\
%Following an estimate in Remark~\ref{rem:flux_finite}, 
We start by observing that, for every $\beta>0$  and    for $\Lebone$-a.a.\ $t\in (0,T)$
%
%\TODO{I have replaced $\Ed$ by $E$ in estimate \eqref{calc1} and in \eqref{for-setwise-convergence}, it seems to me that we don't need to restrict to $\Ed$, correct? On the other hand, I have replaced $w_t^{\ej}(x,y)  $ by $|w_t^{\ej}(x,y) | $ }
%and Borel set $A\subset \Ed$
    \begin{equation}
    \label{calc1}
    \begin{aligned}
      |\bg_t^\varepsilon|(\RNEW E) \EEE 
        &= \RNEW \frac12  \iint_{E} \EEE \frac1\beta (1{\wedge}\dpi(x,y)) |w_t^{\ej}(x,y) |  \tetapien (\dd x \dd y )
        \\
        & 
        \stackrel{(1)}{=} \RNEW \frac12 \EEE \iint_{E_t^{\ej}}    \frac1\beta (1{\wedge}\dpi(x,y)) \frac{|w_t^{\ej}(x,y)|}{\upalpha(u_t^{\ej}(x), u_t^{\ej}(y))} \upalpha(u_t^{\ej}(x), u_t^{\ej}(y))
        \,   \tetapien  (\dd x \dd y )
       \\
       &    \stackrel{(2)}\leq  \frac1\beta 
       \Big[
   \RNEW \frac12 \EEE \iint_{E_t^{\ej}}   \uppsi\left(  \frac{w_t^{\ej}(x,y)}{\upalpha(u_t^{\ej}(x), u_t^{\ej}(y))}\right) \upalpha(u_t^{\ej}(x), u_t^{\ej}(y)) \,   \tetapien  (\dd x \dd y )
  \\
  & \qquad \qquad + \RNEW \frac12 \EEE  \iint_{E_t^{\ej}}   \uppsi^*\left( \beta  (1{\wedge}\dpi(x,y)) \right) \upalpha(u_t^{\ej}(x), u_t^{\ej}(y)) \,   \tetapien  (\dd x \dd y ) \Big]
  \\
  & 
    \stackrel{(3)}\leq  \frac1\beta 
       \Big[
\scrR_{\ej} (\rho_t^{\ej}, \bj_t^{\ej})
+ \RNEW \frac12 \EEE\psih(\beta) \iint_{E_t^{\ej}}   \left( 1{\wedge}\dpi^2(x,y) \right) \upalpha(u_t^{\ej}(x), u_t^{\ej}(y)) \,  \tetapien  (\dd x \dd y ) \Big]\,.
  \end{aligned}
  \end{equation}
 Indeed,
     {\footnotesize (1)} is due to the fact that $|\bj_t^{\ej}|(E {\setminus} E_t^{\ej}) = 0$ (cf.\ \eqref{def-set-Eeps}),  while 
     \RNEW
     {\footnotesize (2)} follows from the fact that $\uppsi$ is even, and eventually
      the estimate 
      {\footnotesize (3)} for $\uppsi^*$ in terms of $\psih$  follows from \eqref{needed-control}\,. \EEE
     Now, in order to estimate the second integral in the above formula, we use that, 
     \RCR by the concavity (and, a fortiori, continuity) of $\upalpha$,
%     for  fixed $(c,c') \in \mathrm{dom}(\upalpha^*)$ there holds
%     \[
%     \upalpha(r,s) \leq c r + c' s - \upalpha^*(c,c')\,.
%     \]
%     with $\upalpha^*(c,c') = \inf_{(r,s)\in \R_+\times\R_+} (c r +c's -\upalpha(r,s))$ the concave conjugate of $\upalpha$. Observe that $\upalpha^*\leq 0$. 
%     Therefore, 
     by \eqref{boundedness-un} we have for  $\pi$-a.a.\ $x,y\in V$
     \begin{equation}
     \label{for-later-use-uppalpha}
      \upalpha(u_t^{\ej}(x), u_t^{\ej}(y))\leq %c  u_t^{\ej}(x) +   c' u_t^{\ej}(y)-   \upalpha^*(c,c') \leq  (c{+}c')\overline{U}  - \upalpha^*(c,c')\doteq
       C_{\overline{U}}\,
     \end{equation}
     for some $ C_{\overline{U}}>0$. \EEE
     Also taking into account that 
       \[
   \iint_{E}  ( 1{\wedge}\dpi^2(x,y))\, \tetapien(\dd x \dd y )  \leq   \iint_{E}  ( 1{\wedge}\dpi^2(x,y))\, \tetapi(\dd x \dd y )  \leq c_k \pi(V)
   \]
   by \eqref{mitigation of singularity}, we thus obtain 
      \begin{equation}
    \label{calc2}
     \begin{aligned}
      \iint_{E_t^{\ej}}   \left( 1{\wedge}\dpi^2(x,y)) \right) \upalpha(u_t^{\ej}(x), u_t^{\ej}(y)) \,  \tetapien  (\dd x \dd y ) \leq   C_{\overline{U}}  c_\kappa \pi(V)  \doteq \overline{C}\,.
%     \\ & 
%     \leq   c   \iint_{E_t^{\ej}}   u_t^{\ej}(x)( 1{\wedge}\dpi^2(x,y)) \,  \tetapien  (\dd x \dd y )+ 
%      c'   \iint_{E_t^{\ej}}   u_t^{\ej}(y)( 1{\wedge}\lambda^2(x,y)) \,   \tetapien  (\dd x \dd y )
%      \\
%      & \quad 
%      - \upalpha^*(c,c') \iint_{E_t^{\ej}}   \left( 1{\wedge}\lambda^2(x,y)) \right) \,   \tetapien  (\dd x \dd y )
%    \\ &    \stackrel{(1)}\leq (c{+}c')   \iint_{E_t^{\ej}}   u_t^{\ej}(x) ( 1{\wedge}\lambda^2(x,y)) \,  \tetapien  (\dd x \dd y )    - \upalpha^*(c,c') \iint_{E_t^{\ej}}   \left( 1{\wedge}\lambda^2(x,y)) \right) \,   \tetapien  (\dd x \dd y )
%    \\
%    &  \stackrel{(1)}\leq
    \end{aligned}
  \end{equation}
%    where for {\footnotesize (1)} we have used the detailed balance condition \eqref{DBC} and  the symmetry of $\lambda$
%    \RNEW (cf.\ Remark \ref{rmk:lambda}),  \EEE
%     while  {\footnotesize (2)} follows the fact that 
%%    from \eqref{boundedness-un} and \eqref{mitigated-bound}, which yields
%  . \EEE
 %  \int_V \int_{V{\setminus}\{x\}} ( 1{\wedge}\lambda^2(x,y)) \kappa_{\ej}( x \dd y ) \, \pi(\dd x) 
 \par
    Mimicking the above calculations we conclude that 
    for every Borel subset \RNEW $A \subset [0,T]\times E$  \EEE and every $\beta>0$
    \[
    \begin{aligned}
|\bg_{\Lebone}^{\ej}|(A)  & =  \RNEW \frac12 \EEE    \iiint_{A} \frac1\beta (1{\wedge}\dpi(x,y)) w_t^{\ej}(x,y)\,\tetapien(\dd x \dd y ) \dd t 
\\
& \leq \frac1\beta 
       \Big[
\int_0^T \scrR_{\ej} (\rho_t^{\ej}, \bj_t^{\ej}) \dd t 
+ \RNEW \frac12 \EEE\psih(\beta)    \iiint_{A}   \left( 1{\wedge}\dpi^2(x,y)) \right) \upalpha(u_t^{\ej}(x), u_t^{\ej}(y)) \,\tetapien(\dd x \dd y )  \dd t
\Big]\,
\\
& \leq \frac1\beta \mathsf{E}_0
+ \frac{\psih(\beta)}\beta  \frac{C_{\overline{U}}}2  \iiint_{A}
  \left( 1{\wedge}\dpi^2(x,y)) \right)\,\tetapien(\dd x \dd y ) \dd t \,,
%  \\
%  & \leq \frac1\beta \mathsf{E}_0
%+ M_2 \frac{\psih(\beta)}\beta   \iiint_{A} 
%  \left( 1{\wedge}\dpi^2(x,y)) \right) \tetapi(\dd x \dd y ) \dd t 
  \end{aligned}
    \]
    where we have 
    %used the place-holder $M_2=\RNEW \tfrac12 \EEE(c{+}c')\overline{U}  - \upalpha^*(c,c') $ and  
     relied on estimate \eqref{action-bounded}.
    % and ultimately used that
    %$ \tetapi_{\ej} \leq \teta$ 
    %since 
    %$\kappa_{\ej}(x,\cdot) \leq \kappa(x,\cdot)$. 
   In this way, exploiting the arbitrariness of $\beta$  we may prove that 
   \begin{equation}
   \label{for-setwise-convergence}
   \forall\, \eta>0 \ \exists\, \delta>0 \, : \ \RNEW A \in \mathfrak{B}([0,T]\times E), \EEE \ \teta_\Lebone^{\dpi}  (A)<\delta \ \Longrightarrow \ \sup_n |\bg_{\Lebone}^{\ej}|(A) <\eta
   \end{equation}
   where we have used the place-holder $\teta_\Lebone^{\dpi}$ for the positive finite measure $ (1{\wedge}\dpi^2)(\tetapi)_\Lebone$.
   Therefore, we have shown that the sequence $(\bg_{\Lebone}^{\ej})_n$ is relatively compact w.r.t.\ setwise convergence in \RNEW $\calM((0,T){\times} E)$. \EEE
   \par
   \RNEW It remains to show that $\bG = \bg_t \Lebone$ for a measurable family $(\bg_t)_{t\in[0,T]} \subset \calM(E)$.
   For this, it is sufficient to 
    apply the disintegration result from  \cite[Cor.\ 10.4.15]{Bogachev07}. Hence, we need to 
   show that the first marginal  of $\bG$, i.e.\ the push-forward measure $(\pi_0)_{\#}\bG$ via the projection $\pi_0: [0,T]{\times}E
   \to [0,T]$, is absolutely continuous w.r.t.\ the Lebesgue measure $\Lebone$. 
    It is immediate to deduce this property from  
  estimates \eqref{calc1}--\eqref{calc2}, 
  which yield for all $I\subset [0,T]$ and $\beta>0$
  \[
  ((\pi_0)_{\#}\bG) (I) \leq \liminf_{n\to\infty} \int_I   |\bg_t^{\ej}|(\Ed)  \dd t 
  \leq \frac1{\beta} \sup_n \int_I \scrR_{\ej} (\rho_t^{\ej}, \bj_t^{\ej}) \dd t +  \overline{C} \frac{ \psih(\beta)}{ 2\beta} \Lebone(I) \,.
  \]
   All in all, we have shown
   {\bf Claim $1$}.  \EEE
  % \RRR Possibly, give more details on the disintegration of $\bG$, which gives the disintegration of $\bj$.. \EEE
  \medskip
  \paragraph{\bf Claim $2$:}  {\sl  Convergence \eqref{DBL} holds along a further, not relabeled, subsequence, \RNEW and
 $ \rho_t(\dd x) = u_t\,   \pi(\dd x) $ with $u$ from \eqref{added-cvg-u}.} \EEE 
  \\
  Indeed,
  \RNEW on the one hand, it follows from \eqref{boundedness-un} that 
  \begin{equation}
  \label{setwise-compact}
     \forall\, \eta>0 \ \exists\, \delta>0 \, : \  A \in \mathfrak{B}(V),  \pi (A)<\delta \ \Longrightarrow \ \sup_n \sup_{t\in [0,T]} \rho_t^{\ej} (A) <\eta\,.
  \end{equation}
  \RNEW This shows that the family $( \rho_t^{\ej} )_{t\in [0,T],\, n \in \N}$ fulfills the analogue of property 
  \eqref{eq:73} with respect to the measure $\gamma =\pi$, which characterizes (sequential) compactness w.r.t.\ setwise convergence, \RCR and thus the setwise 
  topology, 
   in $\calM(V)$. 
   Therefore, the `compactness' condition   \eqref{compactness}  in Theorem \ref{thm:Ascoli}  holds with $\mathsf{K} $ a setwise  compact subset of $\calM(V) $. \EEE
%%   
%%   In turn, since setwise convergence is 
%%  equivalent to convergence w.r.t.\ the weak topology of $(\calM(V),  \|\cdot\|_{\mathrm{TV}})$, we conclude  that   
%%  \TODOBS{\EEE Just a caveat: is weak compactness equivalent to sequential weak compactness in  $\calM(V) $?? because, in order to apply  Theorem \ref{thm:Ascoli}
%%  we need a topology! Otherwise, we should argue as in \cite{PRST22} and argue with the densities $u$, instead of the measures $\rho$ }
    \par
  On the other hand, \EEE
   estimate
  \eqref{est:ct-eq-BL}	 (which holds, all the more, along the solutions $(\rho^{\ej}, \bj^{\ej}) \in \ENHCE 0T$ of the continuity equation), guarantees that 
  \[
     \nbl{\rho_{t_2}^{\ej}- \rho_{t_1}^{\ej}} \leq 
  \int_{t_1}^{t_2} \RNEW  |\bg_t^{\ej}|(E) \EEE \, \dd t \qquad \text{for  }\Lebone\text{-a.a. }  t_1 \leq t_2 \in [0,T]\,,
  \]
    \RNEW cf.\ \eqref{nbl-norm} for the definition of $\nbl{\cdot}$. \EEE
  Combining it with estimates \eqref{calc1}--\eqref{calc2} we conclude that for all $\beta>0$
  \begin{equation}
  \label{Ascoli}
  \begin{aligned}
    \nbl{ \rho_{t_2}^{\ej} {-} \rho_{t_1}^{\ej}} &  \leq 
   \frac1\beta 
       \Big[ \int_{t_1}^{t_2}
\scrR_{\ej} (\rho_t^{\ej}, \bj_t^{\ej}) \dd t 
+ \psih(\beta) \int_{t_1}^{t_2}  \frac{C_{\overline{U}}}2 c_\kappa \pi(V)  \dd t \Big]
\\
& 
\leq \frac1\beta (\mathsf{E}_0 +   C_{\overline{U}} c_\kappa \pi(V) \psih(\beta)|t_2{-}t_1| )\,.
\end{aligned}
  \end{equation}
  By \eqref{Ascoli}, the equicontinuity  condition \eqref{equicontinuity} holds  with 
the function $\omega: [0,T]{\times}[0,T]\to [0,+\infty)$ given by \EEE
\[
\omega(r,s): = \inf_{\beta>0} \frac1\beta (\mathsf{E}_0 +   C_{\overline{U}} c_\kappa \pi(V) \psih(\beta)|r{-}s| )\,,
\]
which 
satisfies $\lim_{(r,s)\to (t,t} \omega(r,s)=0$ for all $t\in[0,T]$.
%  \TODOR{We need to settle this:
% is there a topology inducing setwise convergence? or, can we rely on the weak topology of $(\calM(V), \|\cdot\|_{TV})$ thanks to the characterization of 
%setwise convergence on page  \pageref{eq:70}?}
\par
\RNEW All in all,
we are in a position to apply Theorem \ref{thm:Ascoli} to the sequence $(\rho_n)_n$, in the space $\mathscr{S} :=\calM(V) $ endowed with the 
\RCR setwise topology on 
 $\calM(V)$  \EEE
 %{\color{red} the 
%topology (???)} of setwise convergenc
 and with the metric
$\mathsf{d}$ induced by $\nbl{\cdot}$, for which the compatibility condition  \eqref{compatibility-top-metr} clearly holds.
%Thanks to \eqref{setwise-compact}, 
%he sequence $(\rho_n)_n$ satisfies condition \eqref{compactness} with $\mathsf{K} $ a setwise compact subset of $\calM(V) $. In turn, by 
%\eqref{Ascoli} the equicontinuity  condition \eqref{equicontinuity} holds  with 
%the function $\omega: [0,T]{\times}[0,T]\to [0,+\infty)$ given by \EEE
%\[
%\omega(r,s): = \inf_{\beta>0} \frac1\beta (\mathsf{E}_0 +  M_2 c_\kappa \pi(V) \psih(\beta)|r{-}s| )\,,
%\]
%which 
%satisfies $\lim_{(r,s)\to (t,t} \omega(r,s)=0$ for all $t\in[0,T]$.
 Hence, by 
\RNEW Theorem \ref{thm:Ascoli} we  conclude  convergence \eqref{DBL} along a suitable subsequence. 
 A straightforward argument yields   $ \rho_t(\dd x) = u_t\,   \pi(\dd x) $ for $\Lebone$-a.a.\ $t\in (0,T)$.  We thus \EEE
conclude \textbf{Claim~$2$}.
\medskip

\paragraph{\bf Claim $3$:} {\sl define $\bj_{\Lebone}\in \Mloc([0,T]{\times}\Ed)$ by 
\[
\bj_{\Lebone}(\dd t \dd x \dd y) = \frac1{1{\wedge}\dpi(x,y)} \bg_{\Lebone}( \dd t \dd x \dd y)
\]
 Then, convergence \eqref{cvg-j} \RNEW and the disintegration property \eqref{disintegration-props-rho-j} (with $\bj_t (\dd x \dd y)=  \frac1{1{\wedge}\dpi(x,y)} \bg_t(\dd x \dd y) \in \Mloc(\Ed)$) hold.}
\\
\RNEW It suffices to apply Lemma \ref{l:obvious-useful} to with the choices 
$\eeta_n: = \bg_{\Lebone}^{\ej}$ and 
$\zzeta_n: = \bj_{\Lebone}^{\ej}$. \EEE
%%%%
\RNEW This finishes the proof of Prop.\ \ref{prop:compactness}. 
\end{proof}
\medskip
As an immediate consequence of convergences  \eqref{DBL} and \eqref{cvg-j} we have the following
\begin{corollary}
The pair $(\rho,\bj) $ satisfies   the continuity equation on the interval  $[0,T]$  in the sense of Def.\ \ref{def:CE}. 
\end{corollary}



\par
\RNEW For later use in the proof of Lemma \ref{l:UEDE}, we also record the following convergences. \EEE
\begin{corollary}
 Consider the measures $(\ssigma_t^{\ej})_{t\in [0,T]} \subset \calM^+(E)$ defined by
\[
\ssigma_t^{\ej} (\dd x \dd y) =  \frac{1\wedge \dpi^2(x,y)}{\ej + (1{\wedge}  \dpi^2(x,y))} \upalpha(u_t^{\ej}(x),u_t^{\ej}(y) ) \, \tetapi(\dd x \dd y)\,,
    \]
     and let $\ssigma_{\Lebone}^{\ej} \subset \calM([0,T]{\times}E)$ be given by 
    $\ssigma_{\Lebone}^{\ej} (\dd t \dd x \dd y) = \ssigma_t^{\ej} (\dd x \dd y) \, \Lebone(\dd t)$.
\RNEW    Then, there exists $ \ssigma_\Lebone \in   \Mloc^+([0,T]{\times}\Ed)$, with 
$\ssigma_\Lebone(\dd t \dd x \dd y )= \ssigma_t (\dd x \dd y) \Lebone (\dd t)$ for some $\ssigma = (\ssigma_t)_{t\in[0,T]} \subset \Mloc^+(\Ed)$, such that 
\begin{equation}
\label{key-absolute-continuity}
\ssigma_t (A) \leq  \bnu_{\rho_t}(A) =  \Aalpha[\teta_{\rho_t}^-,\teta_{\rho_t}^+|\tetapi](A)  \qquad \text{for  all } A \in  \mathfrak{B}_{\mathrm{c}}(\Ed),  \
 \text{for }\Lebone\text{-a.a.\ } t \in (0,T),
\end{equation}
and 
        \begin{align}
    \label{vague-signa}
  \ssigma_{\Lebone}^{\ej} \to \ssigma_\Lebone   && \text{vaguely in }  \Mloc([0,T]{\times}\Ed)\,.
\end{align}
\end{corollary}  
%    
%        Then, there exist
%    and
%      $\ssigma_{\Lebone} = \frac1{1\wedge\lambda^2} \nnu_{\Lebone} \in \Mloc^+([0,T]{\times}\Ed)$ such that, up to a (not relabeled) subsequence,
%    \begin{align}
%  \label{setwise-mu}
%&  \nnu_{\Lebone}^{\ej} \to    \nnu_\Lebone&& \text{setwise in }  \calM([0,T]{\times}\Ed),
%\\
%&
%  \label{vague-signa}
%  \ssigma_{\Lebone}^{\ej} \to \ssigma_\Lebone   && \text{vaguely in }  \Mloc([0,T]{\times}\Ed)\,.
%\end{align}
%
%\TODO{do $\nn$ and $\bss$ disintegrate?}
\begin{proof}
\RNEW Let 
    $(\nnu_{t}^{\ej})_{t \in [0,T]} \in \calM^+([0,T]{\times}E)$ \RNEW be given by
   \[
    \nnu_{t}^{\ej}(\dd x \dd y) = (1{\wedge} \dpi^2(x,y))\ssigma_t^{\ej} (\dd x \dd y)  \quad \text{and set} \qquad \nnu_{\Lebone}^{\ej}  = \nnu_t^{\ej} \, \Lebone\,.
    \]
By repeating the very same calculations as in \eqref{calc2}, we find for any $A\in \frB(E)$,
\begin{equation}
\label{crucial-for-nnu}
\begin{aligned}
\nnu_t^{\ej}(A)  & = \iint_A (1{\wedge} \dpi^2(x,y)) \,\ssigma_t^{\ej}(\dd x \dd y) \\ & \leq \iint_A (1{\wedge} \dpi^2(x,y))
 \upalpha(u_t^{\ej}(x),u_t^{\ej}(y) ) \, \tetapi(\dd x \dd y) 
 \leq  C_{\overline{U}}  \iint_A (1{\wedge} \dpi^2(x,y)) \tetapi(\dd x \dd y)\,.
 \end{aligned}
\end{equation}
Then, a further integration of  \eqref{crucial-for-nnu} w.r.t.\ the Lebesgue measure
$\Lebone$ reveals that 
 the sequence $(\nnu^{\ej})_n$ enjoys the analogue of estimate \eqref{for-setwise-convergence}.
Thus there exists   $\nnu \in \calM^+([0,T]{\times}E)$ % with $\nnu_\Lebone(\dd t \dd x \dd y )= \nnu_t (\dd x \dd y) \Lebone (\dd t)$ for some $\nnu = (\nnu_t)_{t\in[0,T]} \subset \calM^+(\Ed)$,
 such that   
$ \nnu_{\Lebone}^{\ej} \to    \nnu $ setwise in $  \calM([0,T]{\times}E) $  \RNEW
 along a suitable subsequence.
From \eqref{crucial-for-nnu}  we also infer  for every $I \subset [0,T]$
\[
((\pi_0)_{\#} \nnu) (I) \leq \liminf_{n\to\infty} \int_I \nnu_t^{\ej}(E') \dd t \leq M_2 \,  \Lebone(I) \, \iint_{E'} (1{\wedge} \dpi^2(x,y)) \tetapi(\dd x \dd y)  \,,
\]
which shows that  the first marginal of $\nnu$ is absolutely continuous w.r.t.\ $\Lebone$. Then, again by \cite[Cor.\ 10.4.15]{Bogachev07}
  $\nnu$ can be disintegrated w.r.t.\ $\Lebone$ in terms of a family   $(\nnu_t)_{t\in [0,T]} \subset \calM^+(E)$. 
%  
%  \TODO{same comment as for Claim $1$ in the proof of Prop.\ \ref{prop:compactness}:  for $\nnu^{\ej}$, we could replace setwise convergence in $  \calM([0,T]{\times}\Ed) $ by 
%  setw cvg  in $  \calM([0,T]{\times}E)$, correct? We should be consistent with that...}
 \par
 Clearly,
 applying  Lemma \ref{l:obvious-useful} with  %$\ell = \lambda^2$, 
 $\eeta_n = \nnu_{\Lebone}^{\ej} $
 and $\zzeta_n = \ssigma_{\Lebone}^{\ej} $ we infer convergence  \eqref{vague-signa}. 
 It remains to prove 
 \eqref{key-absolute-continuity}.
 For this, we will indeed first show that 
 \begin{equation}
 \label{intermediate-inequality}
 \int_a^b  \nnu_t(A) \, \dd t \leq \int_a^b \iint_A  (1{\wedge} \dpi^2(x,y)) \, \bnu_{\rho_t} (\dd x \dd y) \dd t \qquad \text{for all } A \in \mathfrak{B}(E) \text{ and  }  (a,b) \subset (0,T)\,.
 \end{equation}
 Indeed, we  have 
 \[
 \begin{aligned}
  \int_a^b  \nnu_t(A) \, \dd t \stackrel{(1)}{=} \lim_{n\to\infty} \int_a^b \nnu_t^{\ej}(A) \, \dd t  
  &  \stackrel{(2)}{\leq} \limsup_{n\to\infty}\iiint_{(a,b){\times}A}
 \upalpha(u_t^{\ej}(x),u_t^{\ej}(y) ) \, \teta_\Lebone^{ \dpi}(\dd t \dd x \dd y) 
 \\
 &
  \stackrel{(3)}{\leq} \iiint_{(a,b){\times}A}
 \upalpha(u_t(x),u_t(y)) \, \teta_\Lebone^{ \dpi}(\dd t \dd x \dd y) 
\end{aligned}
 \]
 where {\footnotesize (1)}  follows from the setwise convergence $ \nnu_{\Lebone}^{\ej} \to    \nnu $,  {\footnotesize (2)} is due to  \eqref{crucial-for-nnu} (recall that 
  $\teta_\Lebone^{\dpi}$ is a place-holder for  $ (1{\wedge}\dpi^2)(\tetapi)_\Lebone$). % \RCR and {\footnotesize (3)}  ensues from the Fatou Lemma. \EEE
  Finally, 
 {\footnotesize (3)}  ensues from combining  the fact that, in fact, 
 \[
 u^{\ej} \weaksto u \text{ in } \rmL^\infty((0,T){\times}V;  \teta_\Lebone^{ \dpi})
 \]
 (cf.\ \eqref{added-cvg-u}) due to  \eqref{boundedness}, with a variant of the Ioffe theorem, cf.\ \cite[Thm.\ 21, p.\ 171]{Valadier90}. Hence, \eqref{intermediate-inequality} 
 follows, implying that 
 \begin{equation}
 \label{localized-test-functions}
 \begin{aligned}
 \int_a^b \iint_{E} \phi(x,y) \, \nnu_t(\dd x \dd y) \dd t  & \leq  \int_a^b \iint_{E} (1{\wedge} \dpi^2(x,y)) \phi(x,y) \, \bnu_{\rho_t} (\dd x \dd y) \dd t 
 \\ & 
  \qquad \text{for all } \phi \in \Bb(E) \quad \text{with } \phi \geq 0 \text{ and all }  (a,b) \subset (0,T).
  \end{aligned}
  \end{equation}
 Let us now fix an arbitrary  test function $\varphi \in \Cc(\Ed)$ with $\varphi \geq 0$: we have that 
 \[
 \begin{aligned}
 \int_a^b \iint_{E} \varphi(x,y) \, \ssigma_t(\dd x \dd y)\,  \dd t  &  =  \int_a^b \iint_{E} \frac{\varphi(x,y)}{1{\wedge} \dpi^2(x,y)} \, \nnu_t(\dd x \dd y)\,  \dd t  
 \\
 & 
 \stackrel{(1)}{\leq} 
  \int_a^b \iint_{E} \frac{\varphi(x,y)}{1{\wedge} \dpi^2(x,y)} (1{\wedge} \dpi^2(x,y))  \, \bnu_{\rho_t} (\dd x \dd y)   \dd t  
  \\
  &
  = 
 \int_a^b \iint_{E} \varphi(x,y) \, \bnu_{\rho_t} (\dd x \dd y)   \dd t \,,
 \end{aligned}
\]
 where {\footnotesize (1)} follows from \eqref{localized-test-functions}, since $ \frac{\varphi(x,y)}{1{\wedge} \dpi^2(x,y)} $ extends to a function in $\Bb(E)$. Therefore, 
 \eqref{key-absolute-continuity} follows. 
 
   \end{proof} \EEE
\subsection{Upper energy-dissipation inequality}
\label{ss:5.2}
Ultimately, we are in a position to prove the following 
\begin{lemma}
\label{l:UEDE}
The pair $(\rho,\bj)$ satisfies the upper energy-dissipation estimate \eqref{UEDE}. 
\end{lemma}
\RBS Again, we emphasize that so far we have relied on the sole conditions \eqref{initial-thm:main} where, in particular, \eqref{strong-initial}  may hold with
$\underline{U}=0$. \EEE
\begin{proof}
We take the limit as $n\to\infty$ in \eqref{EDB-n}. By \eqref{DBL} we have 
\begin{equation}
\label{lsc-energies}
\scrE(\rho_t) \leq \liminf_{n\to\infty} \scrE(\rho_t^{\ej}) \qquad \text{for all } t \in [0,T].
\end{equation}
\par
Next, we
rewrite $\int_0^t \scrR^{\ej}(\rho^{\ej}, \jj^{\ej}) \dd r$  in terms of the measure $(\ssigma_t^{\ej})_t$.
In what follows, we use the place-holder $m^{\ej} = \tfrac{1{\wedge} \dpi^2}{\ej +(1{\wedge} \dpi^2)}$:
we find  for  $\scrL$-a.a.\ $t\in (0,T)$ and  for all  $ (x,y)\in E_t^{\ej} $
\[
\frac{w_t^{\ej}(x,y)}{\upalpha(u_t^{\ej}(x),u_t^{\ej}(y))} = \frac{w_t^{\ej}(x,y) \, m^{\ej}(x,y)}{\upalpha(u_t^{\ej}(x),u_t^{\ej}(y)) \, m^{\ej}(x,y)} =
\frac{\dd (\RNEW 2\bj_t^{\ej})}{\dd \ssigma_t^{\ej}} (x,y) \,.
\]
Therefore
 \[
 \begin{aligned}
 \scrR(\rho_t^{\ej},\bj_t^{\ej})   & 
 = \frac12\iint_{E_t^{\ej}}% {_\upalpha(u)}
    \uppsi\Bigl(\frac{w_t^{\ej}(x,y)}{\upalpha(u_t^{\ej}(x),u_t^{\ej}(y))}\Bigr) \upalpha(u_t^{\ej}(x),u_t^{\ej}(y))\,\tetapi \dd
    x,\dd y)
    \\
     & =  \frac12\iint_{E_t^{\ej}}% {_\upalpha(u)}
    \uppsi\Bigl( 2\frac{\dd \bj_t^{\ej}}{\dd \ssigma_t^{\ej}} (x,y)  \Bigr)\frac1{m^{\ej}(x,y)}\upalpha(u_t^{\ej}(x),u_t^{\ej}(y))\, m^{\ej}(x,y) \,\tetapi(\dd
    x,\dd y)
    \\ &  \geq  \frac12\iint_{E_t^{\ej}}% {_\upalpha(u)}
    \uppsi\Bigl( 2 \frac{\dd \bj_t^{\ej}}{\dd \ssigma_t^{\ej}} (x,y)  \Bigr)\,\ssigma_t^{\ej}(\dd
    x,\dd y)\,.
\end{aligned}
\]
where we have used that $1/m^{\ej} \geq 1$. 
All in all, %using the short-hand $\Upsilon(v):= \frac12 \uppsi(2v)$, 
we have 
\begin{equation}
\label{lsc-R}
\begin{aligned}
\liminf_{n\to\infty} \int_0^t 
 \scrR(\rho_r^{\ej},\bj_r^{\ej}) \dd r  & =  \frac12 \liminf_{n\to\infty}  \iiint_{[0,t]{\times}\Ed} \uppsi \left(2\frac{\dd \jj_{\Lebone}^{\ej}}{\dd \ssigma_{\Lebone}^{\ej}}  \right)  \ssigma_{\Lebone}^{\ej} (\dd r \dd x \dd y) \\ & 
  = \liminf_{n\to\infty}  \calF_{\uppsi}\left( \jj_{\Lebone}^{\ej}|\ssigma_{\Lebone}^{\ej} \right) 
  \\
  &
  \geq 
   \frac12 \calF_{\uppsi}\left( 2\jj_{\Lebone}|\ssigma_{\Lebone} \right) 
 = \frac12 \iiint_{[0,t]{\times}\Ed} \uppsi \left(2\frac{\dd \jj_{\Lebone}}{\dd \ssigma_{\Lebone}}  \right)  \ssigma_{\Lebone} (\dd r \dd x \dd y)
 \\
 & \qquad \qquad \qquad = \frac12 \int_0^t \iint_{\Ed} \uppsi \left( 2\frac{\dd \jj_{r}}{\dd \ssigma_{r}}  (x,y) \right) \,  \ssigma_r (\dd x \dd y)\, \dd r
 \end{aligned}
 \end{equation}
 since $ \jj_{\Lebone}^{\ej} \to \jj_\Lebone$, $ \ssigma_{\Lebone}^{\ej} \to \ssigma_\Lebone$
 \RNEW vaguely in $\Mloc([0,T]{\times}\Ed)$, 
  and $\mathcal{F}_{\uppsi}$ is lower semicontinuous w.r.t.\ vague convergence by Lemma 
 \ref{l:crucial-F}. 
 Now,
 since $\bj_t \leq \bnu_{\rho_t}$ for $\Lebone$-a.a.\ $t\in (0,T)$ by \eqref{key-absolute-continuity}, recalling the monotonicity property
 \eqref{AC-monotonicity} from Lemma \ref{l:crucial-F}
  we have that 
 \[
 \begin{aligned}
 \frac12 \int_0^t \iint_{\Ed} \uppsi \left(2\frac{\dd \jj_{r}}{\dd \ssigma_{r}}  (x,y) \right) \,  \ssigma_r (\dd x \dd y)\, \dd r  & \geq \frac12  \int_0^t 
  \iint_{\Ed} \uppsi \left(2\frac{\dd \jj_{r}}{\dd \bnu_{\rho_r}}  (x,y) \right) \,  \bnu_{\rho_r}(\dd x \dd y)\, \dd r 
  \\
  & 
  =
 \int_0^t \scrR(\rho_r,\bj_r) \dd r\,.
 \end{aligned}
 \] \EEE
 \par
Finally, in order to take the limit as $n\to\infty$ in the Fisher information term, we use that, by \eqref{Fisher-bounded}, 
 \[
 \begin{aligned}
\Fish(\rho_t^{\ej})  = \frac12 \iint_{E'}  \mathrm{D}_\upphi(u_t^{\ej}(x),u_t^{\ej}(y)) \, \tetapien(\dd x \dd y)
\end{aligned}
\]
Therefore, introducing the measures $\teta_{\Lebone}^{\rho^{\ej},\pm}(\dd t \dd x \dd y) = \teta_{\rho^{\ej},t}^{\pm} (\dd x \dd y) \, \Lebone (\dd t)$ and 
$\teta_{\ej,\Lebone}(\dd t \dd x \dd y) = \teta_{\ej}(\dd x \dd y) \, \Lebone (\dd t)$, we rewrite
\[
\begin{aligned}
\int_0^t \Fish(\rho_r^{\ej}) \dd r  & = 
\frac12 \iiint_{[0,t]{\times}E'}   \mathrm{D}_\upphi\left( \frac{\dd \teta_{\Lebone}^{\rho^{\ej},-}}{\dd \teta_{\ej,\Lebone}}, \frac{\dd \teta_{\Lebone}^{\rho^{\ej},+}}
{\dd \teta_{\ej,\Lebone}}\right)\, 
\teta_{\ej,\Lebone}(\dd r \dd x \dd y) \doteq   \calF_{\Xi} (\bbeta_\Lebone^{\ej}|\teta_{\ej,\Lebone}) 
\end{aligned}
\]
where the convex functional $\Xi: \R_+ \times \R_+ \to [0,+\infty]$ is defined by $\Xi(w,z) = \tfrac12 \mathrm{D}_\upphi(w,z)$ and we have used the place-holder
$\bbeta_\Lebone^{\ej} =(\teta_{\Lebone}^{\rho^{\ej},-},\teta_{\Lebone}^{\rho^{\ej},+})$. 
Now, it can be immediately checked that the setwise convergence of $\rho_t^{\eps_n}$ to $\rho_t$ for 
all $t\in [0,T]$
 gives that 
\[
(1{\wedge} \dpi^2) \teta_{\Lebone}^{\rho^{\ej},\pm} \to (1{\wedge} \dpi^2) \teta_{\Lebone}^{\rho,\pm}  \qquad \text{setwise in }  \calM([0,t]{\times}\Ed)
\]
with 
\[
\teta_{\Lebone}^{\rho,\pm} (\dd t \dd x \dd y)  = \teta_{\rho_t}^{\pm} (\dd x \dd y) \, \Lebone(\dd t)\,.
\]
Thus, \RNEW by Lemma \ref{l:obvious-useful} we have \EEE
\[
 \teta_{\Lebone}^{\rho^{\ej},\pm} \to  \teta_{\Lebone}^{\rho,\pm}  \qquad \text{vaguely in }  \Mloc([0,t]{\times}\Ed)\,.
\]
Since we also have $\teta_{\ej,\Lebone} \to \teta_{\Lebone}$ vaguely in $ \Mloc([0,t]{\times}\Ed)$, again by Lemma  \ref{l:crucial-F} we conclude that 
(with the place-holder $\bbeta_\Lebone =(\teta_{\Lebone}^{\rho,-},\teta_{\Lebone}^{\rho^,+})$) 
\[
\begin{aligned}
\liminf_{n\to\infty} \int_0^t \Fish(\rho_r^{\ej}) \dd r 
&  \geq    \calF_{\Xi} (\bbeta_\Lebone|\teta_{\Lebone}) 
%\frac12 
 %\iiint_{[0,t]{\times}E'}   \Xi \left(\frac{\dd (\teta_{\Lebone}^{\rho,-},\teta_{\Lebone}^{\rho,+})}{\dd \teta_{\ej,\Lebone}} \right)\, \teta_{\Lebone}(\dd r \dd x \dd y)
 \\
 &
=  \frac12 \iiint_{[0,t]{\times}E'}   \mathrm{D}_\upphi\left( \frac{\dd \teta_{\Lebone}^{\rho,-}}{\dd \teta_{\Lebone}}, \frac{\dd \teta_{\Lebone}^{\rho,+}}
{\dd \teta_{\Lebone}}\right) \, \teta_{\Lebone}(\dd r \dd x \dd y)
=\int_0^t \Fish(\rho_r) \dd r \,.
\end{aligned}
\]
Taking into account that $\scrE(\rho_n^0)\to \scrE(\rho_0)$, the upper energy-dissipation inequality \eqref{UEDE} follows.
\end{proof}

\section{Chain rule and lower  energy-dissipation inequality}
\label{s:6}
%\TODOH{Outline of the section:
%\\
%1. extend the continuity equation, in such a way that $\varphi = -\dnabla (\phi' \circ u) $ can be chosen as test function for the CE
%\\
%2. deduce the validity}


In this section
\begin{enumerate}
\item
in Proposition \ref{prop:ENH-CE} below we will  
show that the pair $(\rho,j)$ is in fact a solution of the continuity equation in an \emph{enhanced} sense,
in comparison to that of Def.\ \ref{def:CE}.
\end{enumerate}
 Precisely in the proof of  Prop.\ \ref{prop:ENH-CE}, 
\RCR  we will resort to the density property assumed in Assumption  \ref{Ass:F-bis}. \EEE
%\WARNING or whatever condition we adopt 
 We highlight that we
will succeed in proving 
Proposition \ref{prop:ENH-CE}  only under the additional condition that $u= \frac{\dd \rho}{\dd \pi}$ is bounded from above and away from zero, cf.\ 
\eqref{strong-bounds} below. 
Thus, Prop.\ \ref{prop:ENH-CE}  will apply to the pair
$(\rho,\jj)\in\CE 0T$ obtained by the approximation procedure set forth in Section \ref{s:4}, only under the condition that the constant
$\underline U$ from \eqref{strong-initial} is \emph{strictly positive}.
 \begin{enumerate}
     \setcounter{enumi}{1}
     \item 
Exploiting this improved continuity equation, in Proposition
\ref{prop:CR} ahead
 we will  show that, along solutions of the continuity equation satisfying \eqref{strong-bounds}, the chain rule identity holds.
 \item
 In this way, we will conclude the proof of part \textbf{(2)} of Theorem \ref{th:main}. 
 \item
Eventually,  we will  prove  part \textbf{(1)} via an approximation argument. 
\end{enumerate}
\subsection{An enhanced continuity equation}
\label{s:6.1}
In order to extend the class of test functions for the continuity equation, we will crucially rely on the information that the limit pair $(\rho,\bj) \in \CE0T$ constructed in Sec.\ \ref{s:5}
have finite entropy and finite action:
\begin{equation}
\label{finite-entropy+action}
    \sup\nolimits_{t\in [0,T]} \scrE(\rho_t) <\infty,\qquad \int_0^T \scrR(\rho_t,\bj_t)\, \dd t <\infty.
\end{equation}
{\color{gray} In view of \cite[Rmk.\ 4.12]{PRST22}, we may assume without loss of generality that for $\Lebone$-a.a.\ $t\in (0,T)$ the measure $\bj_t$ is skew-symmetric, i.e.\
$s_{\#} \bj_t = -\bj_t$ with $s: E \to E$ the symmetry map $s(x,y) = (y,x)$.}
\TODOR{We no longer need the gray part above, right?}
 \EEE
%Recall that a pair $(\rho,j)$ is said to satisfy the continuity equation if for all $\varphi\in \Lip_b(V)$:
%\[
%    \int_V \varphi(x)\,\rho_t(dx) - \int_V \varphi(x)\,\rho_s(dx) = \int_s^t \iint_\Ed \dnabla \varphi(x,y)\,\bj_r(dxdy)\,dr\qquad\text{for any $(s,t)\subset(0,T)$}.
%\]
%In order to prove the chain rule, we will need to extend the class of functions for $\varphi$ in the continuity equation. In particular, we will extend it to functions $\varphi\in B_b(V)$ for which

%\TODO{Warning: in the proof of Prop.\ \ref{prop:ENH-CE} I have  used the bounds $\underline U \leq u(t,x) \leq \overline U $  }
% that the initial condition $ 0< \underline{U} \leq  u^0 \leq \overline{U}$. I believe, indeed, that we also need $u^0 \geq \underline{U} >0$....}
\begin{proposition}
\label{prop:ENH-CE} 
 Let $(\rho,\jj)\in\CE 0T$ fulfill \eqref{finite-entropy+action} \RBS  and
 suppose that  $\rho_t = u_t \pi$ with 
 \begin{equation}
 \label{strong-bounds}
\exists\, 0<\underline U \leq \overline U \, : \quad \underline U   \leq u_t(x) \leq \overline U   \text{ for $\pi$-a.a.\ $x \in V$ and for all } t\in [0,T].
\end{equation}
\EEE
 %^ emanate from an initial datum $u^0 $ such that $ 0\leq u^0 (x) \leq \overline{U}$ for $\pi$-a.a.\ $x\in V$
 %for some $\overline{U}>0$. \EEE
% Further, let $\varphi\in B_b(V)$ satisfy \eqref{eq:test_extend}.
 Then
 \begin{equation}
 \label{desired-CE}
    \int_V \varphi(x)\,\rho_t(\dd x) - \int_V \varphi(x)\,\rho_s(\dd x) = \int_s^t \iint_\Ed \dnabla \varphi(x,y)\,\bj_r(\dd x \dd y)\,\dd r\qquad\text{for any $[s,t]\subset[0,T]$}.
 \end{equation}
 for all test functions
 $\varphi \in \Bb(V)$ fulfilling
  \begin{equation}
  \label{eq:test_extend}
    \int_0^T \iint_\Ed \uppsi^*(\dnabla \varphi(x,y))\,\upalpha(u_t(x),u_t(y))\,\tetapi(\dd x \dd y)\,\dd t <\infty.
\end{equation}
\end{proposition}
%\TODO{I guess that, indeed, we may prove \eqref{desired-CE} for all test functions in $\Bb(V) \cap \mathcal{X}$, which is maybe easier to read...}

%%%%\TODOR{Here the argument bifurcates.}
%%%%\medskip
%%%%
%%%%\hrule
%%%%\hrule
%%%%\hrule
%%%%\smallskip
%%%%
%%%%
%%%%\noindent \RCR \textbf{Oliver's argument:} \EEE 
%%%%\begin{proof}
%%%%\RBS Preliminarily, we observe that, since $u$ satisfies   \eqref{strong-bounds}, showing \eqref{desired-CE} for all test functions $\varphi \in \Bb(V)$ fulfilling
%%%%\eqref{eq:test_extend} is equivalent to showing it for all test functions  $\varphi \in \Bb(V) \cap \mathcal{X}$, with  $ \mathcal{X}$ from \eqref{X-subset}.  \EEE
%%%%\par
%%%%    The proof follows from an approximation argument.
%%%%    \RNEW Since the measures $\bj_t$ may be supposed skew-symmetric, we may safely assume  \EEE that 
%%%%    the function
%%%%   \RNEW  $w:=d(2\jj)/d\tetapi$  \EEE fulfills  \RNEW $w_t(x,y)=-w_t(y,x)$ for all $(x,y) \in \Ed$ and $\Lebone$-a.a.\ $t\in (0,T)$. \EEE
%%%%    Consider the regularized flux
%%%%    \[
%%%%      \RNEW  \jj_M^\Lebone  \EEE (\dd t\dd x \dd y):= w_t(x,y) \chi_{A_M}(t,x,y)\,(\tetapi)_\Lebone (\dd t \dd x \dd y)\qquad  M>0,
%%%%    \]
%%%%    where 
%%%%    \[
%%%%    A_M := \left\{(t,x,y)\in [0,T]\times \Ed : \frac{|w_t(x,y)|}{1{\wedge}  \dpi^2(x,y)} \le M, \right\}.
%%%%    \]
%%%%    Notice that for $\scrL{\otimes}\pi$-almost every $(t,x)\in(0,T)\times V$, we have
%%%%    \[
%%%%        \int_{V\setminus\{x\}} |w_t(x,y)| 1_{A_M}(t,x,y)\,\kappa(x,\dd y) \le M \int_{V\setminus\{x\}}( 1{\wedge}  \dpi^2(x,y) ) \,\kappa(x,\dd y) \le c_\kappa M.
%%%%    \]
%%%%    In particular, the function
%%%%    \[
%%%%        \textsf{G}_M(t,x) := \int_{V\setminus\{x\}} w_r(x,y) 1_{A_M}(t,x,y)\,\kappa(x,\dd y)
%%%%    \]
%%%%    is an element of $ \rmL^\infty((0,T){\times} V;\scrL\otimes\pi)$ for every $M>0$,  \RNEW and $\bj_{\Lebone}^M$ extends to a \emph{finite} measure in $\calM([0,T]{\times}E)$. \EEE
%%%%    
%%%% %   Now consider the associated regularized curve obtained from solving the continuity equation   
%%%%     \RNEW It can be calculated
%%%%     (cf.\ also \cite[Thm.\ 6.5]{PRST22}),
%%%%      that the curve 
%%%%     $\rho^M:[0,T]\to \calM(V)$ defined by  \EEE
%%%%    %More precisely, we set fo 
%%%%       \begin{align}\label{eq:approximation}
%%%%   \rho_t^M := u_t^M \pi \quad \text{with }      u_t^M(x) := u_0(x) -2\int_0^t \textsf{G}_M(r,x) \dd r
%%%%    \end{align}
%%%%    fort $\pi$-almost every $x\in V$ and all $t\in [0,T]$ solves the continuity equation,  with flux $j^M$,
%%%%  %  \RNEW in the enhanced sense of Definition \ref{def:CE-enh}. \EEE
%%%%     Moreover, since $G_M\in \rmL^\infty((0,T){\times} V;\scrL{\otimes} \pi)$
%%%%     \RNEW we immediately find that $u_t^M\in \rmL^\infty(V;\pi)$. Indeed,
%%%%%     
%%%%%     
%%%%%     \RNEW and since   $\scrE(\rho^0)<+\infty$, so that  $u_0\in \rmL^\upphi(V,\pi)$,   the Orlicz space defined from the Young function $\upphi$, \EEE
%%%%%      we easily deduce that
%%%%%   for all $t\in [0,T]$ we have  $u_t^M\in \rmL^\upphi(V;\pi)$.
%%%%%   \RNEW Furthermore, 
%%%%%   again   
%%%%   by \cite[Thm.\ 6.5]{PRST22} we have that 
%%%%   \begin{equation}
%%%%   \label{uM-above-bounded}
%%%%  0\leq  u_t^M(x) \leq \overline{U} \quad \text{for } \pi\text{-a.a.\ $x\in V$ and every } t \in [0,T]\,.
%%%%   \end{equation}   
%%%%  %  \TODOR{Note that I have had to assume $u^0$ bounded from above in order to obtain \eqref{uM-above-bounded}, which is used to have weak$^*$ compactness of the $u^M$... }
%%%%   
%%%%   
%%%%   
%%%%%. We then consider the {\color{red} reflexive Banach?} space
%%%%   %\[
%%%%   % \mathcal{X} := \left\{ \varphi\in \rm\rmL^{\upphi^*}(V,\pi) : \iint_\Ed \uppsi^*(\dnabla\varphi)\,\dd\teta <\infty \right\} \subset \rm\rmL^{\upphi^*}(V,\pi),
%%%%  % \]
%%%%  % equipped with the norm
%%%%   %\[
%%%%   % \|\varphi\|_\mathcal{X} := \|\varphi\|_{\rm\rmL^{\upphi^*}(V,\pi)} + \|\dnabla\varphi\|_{\rm\rmL^{\uppsi^*}(\Ed,\teta)}.
%%%%   %\]
%%%%\par
%%%% \RNEW    Testing \eqref{eq:approximation} with $\varphi \in \mathcal{X}$ we find 
%%%% \[
%%%% \langle u_t^M{-}u_s^M, \varphi\rangle_{\rmL^1(V;\pi)} = 2 \int_s^t \langle  \textsf{G}_M(r,\cdot), \varphi \rangle_{\rmL^1(V;\pi)} \,\dd r\, \quad \text{for all } 0 \leq s \leq t \leq T.
%%%% \]
%%%% Therefore,  estimating from above,
%%%%    \RNEW we obtain
%%%%    \begin{align}
%%%%    \label{first-step-ascoli}
%%%%       \left |  \langle u_t^M{-}u_s^M,\varphi \rangle_{\rmL^1(V;\pi)} \right| &\le 
%%%%        & \stackrel{(1)}{=}  \iiint_{A_M {\cap} (s,t){\times}\Ed} \left| w_r(x,y)  ({-}\dnabla \varphi(x,y))\right|   \, (\tetapi)_\Lebone (\dd r \dd x \dd y)\,,
%%%%                \end{align}    
%%%%                where for {\footnotesize (1)} we have used that 
%%%%                \[
%%%%                \begin{aligned}
%%%%   &   \iiint_{A_M {\cap} (s,t){\times}\Ed} w_r(x,y)  \varphi(x)   \, (\tetapi)_\Lebone (\dd r \dd x \dd y)  
%%%%   \\
%%%%   &   = \frac12    \iiint_{A_M {\cap} (s,t){\times}\Ed} w_r(x,y)  \varphi(x)   \, (\tetapi)_\Lebone (\dd r \dd x \dd y)       
%%%%   +   \frac12    \iiint_{A_M {\cap} (s,t){\times}\Ed} w_r (y,x)  \varphi(y)   \, (\tetapi)_\Lebone (\dd r \dd x \dd y)  
%%%%      \\
%%%%        &   = \frac12    \iiint_{A_M {\cap} (s,t){\times}\Ed} w_r(x,y)  \varphi(x)   \, (\tetapi)_\Lebone (\dd r \dd x \dd y)  -     
%%%%      \frac12    \iiint_{A_M {\cap} (s,t){\times}\Ed} w_r (x,y)  \varphi(y)   \, (\tetapi)_\Lebone (\dd r \dd x \dd y)   
%%%%      \end{aligned}
%%%%                \]
%%%%                by the skew-symmetry of $w_r(\cdot,\cdot)$ combined with the detailed balance condition, cf.\ \eqref{DBC}. 
%%%%                Next, arguing as for 
%%%%                \eqref{calc1} we estimate  with Young's inequality 
%%%%                \begin{align}
%%%%                 &  \int_s^t \iint_{\Ed}\left| w_r(x,y)  ({-}\dnabla \varphi(x,y))\right|   \, \tetapi (\dd x \dd y) \dd r
%%%%                   \nonumber
%%%%             \\ &     \leq \int_s^t  \iint_{E_\upalpha^r}    \frac{|w_r(x,y)|}{\upalpha(u_r(x), u_r(y))} \upalpha(u_r(x), u_r(y))
%%%%        \,   \tetapi  (\dd x \dd y ) \dd r 
%%%%        \nonumber
%%%%       \intertext{with the set $E_\upalpha^r = \{ (x,y)\in \Ed\, : \ \upalpha(u_r(x), u_r(y))>0 \} $ fulfilling $|\bj_r|(\Ed{\setminus}E_\upalpha^r )=0$ for 
%%%%       $\Lebone$-a.a.\ $r\in (0,T)$, cf.\ \eqref{nice-representation},}
%%%%       &   \leq    \int_s^t \iint_{E_\upalpha^r}   \uppsi\left(  \frac{w_r(x,y)}{\upalpha(u_r(x), u_r(y))}\right) \upalpha(u_r(x), u_r(y)) \,   \tetapi  (\dd x \dd y )\dd r \nonumber
%%%%     \\
%%%%     & \qquad   +
%%%%  \int_s^t \iint_{E_\upalpha^r}   \uppsi^*\left( \dnabla \varphi(x,y)\right) \upalpha(u_r(x), u_r(y)) \,   \tetapi  (\dd x \dd y ) \dd r           \nonumber
%%%%              \\
%%%%              & \leq \int_s^t \scrR (\rho_r, \bj_r)  \dd r +   C_{\overline{U}} (t{-}s) \iint_{\Ed}   \uppsi^*\left( \dnabla \varphi(x,y)\right) \tetapi(\dd x \dd y) \,.         \nonumber
%%%%               \end{align} 
%%%%              Inserting this into \eqref{first-step-ascoli} we find that 
%%%%               \begin{equation}
%%%%               \label{equicontinuity-M}
%%%%               \widehat{\mathsf{d}}(u_t^M,u_s^M) \leq \omega(s,t) \qquad \text{for all } s,t\in [0,T]
%%%%               \end{equation}
%%%%               with $\omega(s,t): =  \int_s^t \scrR (\rho_r, \bj_r)  \dd r +   C_{\overline{U}} (t{-}s) $ and $\widehat{\mathsf{d}}$ {\color{red} the distance???}  on $\rmL^\upphi(V;\pi)$ defined by 
%%%%               \[
%%%%                \widehat{\mathsf{d}}(u_0,u_1): = \sup_{\varphi \in \mathcal{X},  \, \|\dnabla\varphi\|_{\rmL^{\uppsi^*}(\Ed;\tetapi)}\leq 1}  \left |  \langle u_1 {-}u_0, \varphi \rangle_{\rmL^1(V;\pi)} \right| \,.
%%%%               \]
%%%%    %%  
%%%%           \TODO{This is exaclty the point where I need that $\mathcal{X}$ is sequentially dense in $\rmL^1$ w.r.t.\ the weak topology of $\rmL^1$: to ensure that 
%%%%            $  \widehat{\mathsf{d}}$ is a distance, in particular that $  \widehat{\mathsf{d}}(u_0,u_1)=0$ yields $u_0=u_1$....
%%%%           }
%%%%    \par           
%%%%Ultimately, we are in a position to apply Theorem \ref{thm:Ascoli} to the sequence $(u^M)_M \subset \rmL^\infty (V;\pi)$ (hereafter we assume that $M\in \N$), choosing
%%%%\emph{(i)} as $\serifsigma$ the weak$^*$ topology in $ \rmL^\infty (V;\pi)$: then, \eqref{uM-above-bounded} guarantees condition \eqref{compactness});
%%%%\emph{(ii)}  the distance $ \widehat{\mathsf{d}}$: then, \eqref{equicontinuity-M} guarantees the equicontinuity estimate \eqref{equicontinuity}. We thus conclude that  there exists
%%%%$\bar u \in  \rmL^\infty(0,T;\rmL^\infty (V;\pi))$ such that, along a (not relabeled) subsequence, there holds
%%%%\[
%%%%u_t^M \weaksto \bar{u}_t \quad \text{weakly$^*$ in } \rmL^\infty(V;\pi) \quad\text{for all } t \in [0,T]\,. 
%%%%\]
%%%%A fortiori, setting $\rho_t =  \bar{u}_t  \pi$ we have that 
%%%%\begin{equation}
%%%%\label{setwise-rhoM}
%%%%\rho_t^M \to \bar\rho_t  \quad \text{setwise} \quad\text{for all } t \in [0,T]\,. 
%%%% \end{equation}
%%%%
%%%%%\TODO{This is the point where I have used \eqref{uM-above-bounded}. it seems to me that it is cleaner to apply the Ascoli theorem on the level of densities, instead of measures..}
%%%%               
%%%%
%%%%%        
%%%%%        &  \|u_0\|_{\rmL^{\upphi}(V;\pi)}\|\varphi\|_{\rmL^{\upphi^*}(V;\pi)} +2\int_0^t \iint_{\Ed} \psi^*(\varphi(x,y))   \alpha(u_r(x), u_r(y)) \,  \tetapi(\dd x \dd y) \dd r 
%%%%%        + 2\int_0^t  \iint_{\Ed}\psi^* \left( \frac{w_r(x,y)}{\alpha(u_r(x), u_r(y))}  \right)  \alpha(u_r(x), u_r(y)) \,  \tetapi(\dd x \dd y) \dd r 
%%%%%        \\
%%%%%         \|u_0\|_{\rmL^{\upphi}(V;\pi)}\|\varphi\|_{\rmL^{\upphi^*}(V;\pi)} +2  \|\varphi\|_{\rmL^{\upphi^*}(V;\pi)} 
%%%%%        
%%%%%        
%%%%%        
%%%%%         \int_0^t \iint_\Ed  \uppsi(w_r)\,\tetapi\,\dd r\right)\|\varphi\|_\mathcal{X}\,.
%%%%%    \end{align*}
%%%%%   .
%%%%%   
%%%%%   
%%%%%   
%%%%%    $\sup_{t\in[0,T]}\|u_t^M\|_{\mathcal{X}'} <\infty$, from which one easily concludes via an application of the Ascoli--Arzel\`a yheorem the existence of a (not relabelled) subsequence and a limit curve $\bar u:[0,T]\to \mathcal{X}'$ such that
%%%%%    \[
%%%%%        u_t^M\to \bar u_t\quad\text{in $\mathcal{X}'$ for all $t\in[0,T]$}.
%%%%%    \]
%%%%%%    \TODO{I have 2 issues with the above lines: 
%%%%%%    1. we should first of all observe that 
%%%%%%    \[
%%%%%%    \int_0^t \iint_\Ed  \uppsi(w_r^M (x,y))\,\tetapi(\dd x \dd y)\,\dd r \leq    \int_0^t \iint_\Ed  \uppsi(\frac{w_r (x,y))\,\tetapi(\dd x \dd y)\,\dd r
%%%%%%    <+\infty
%%%%%%    \]
%%%%%%    but I can no longer reconstruct the last bound. Shouldn't we rather estimate 
%%%%%% \begin{align*}
%%%%%%        \left|\langle \varphi,u_t^M\rangle_{\rmL^\upphi(V;\pi)}\right| &\le |\langle \varphi,u_0\rangle_{\rmL^\upphi(V;\pi)}| +  2\int_0^t |\langle \varphi, \textsf{G}_M(r,\cdot)\rangle_{\rmL^\upphi(V;\pi)}|\,\dd r \\
%%%%%%        &\le  %2\left(\|u_0\|_{\upphi} +  t + \int_0^t \iint_\Ed  \uppsi(w_r)\,\tetapi\,\dd r\right)\|\varphi\|_\mathcal{X}\,.
%%%%%%    \end{align*}
%%%%%%    2. in order to apply the  Ascoli--Arzel\`a Theorem, I guess that we should also provide an estimate for  }
%%%%%%    
%%%%%    
%%%%%    
%%%%%    
%%%%%    
%%%%%    Furthermore, by an approximation argument {\color{red}[as below]}, one establishes
%%%%
%%%%\par
%%%%We now send $M\to \infty $ in the continuity equation
%%%%    \[
%%%%        \int_V\varphi(x)\,\rho_t^M (\dd x) - \int_V\varphi(x)\, \rho_s^M  (\dd x) = \iiint_{[s,t]{\times}\Ed} \dnabla\varphi(x,y)\, \jj_\Lebone^M(\dd r \dd x \dd y)\quad\text{for all $\varphi\in \Bb(V) \cap \mathcal{X}$}.
%%%%    \]
%%%%    Clearly, thanks to \eqref{setwise-rhoM}
%%%%  we have
%%%%    \[
%%%%        \int_V \varphi \,\dd\rho_t^M \longrightarrow \int_V \varphi \,\dd\bar\rho_t\qquad\text{for all $t\in[0,T]$}.
%%%%    \]
%%%%    For the right-hand side, we use that 
%%%%\[
%%%%\iiint_{A_M {\cap} [s,t]{\times}\Ed } w_r(x,y) \dnabla\varphi(x,y) (\tetapi)_\Lebone (\dd r \dd x \dd y)  \longrightarrow 
%%%%\int_s^t \iint_{\Ed } w_r(x,y) \dnabla\varphi(x,y) \tetapi (\dd x \dd y)\, \dd r  
%%%%\]    
%%%%    by the  Dominated Convergence Theorem: indeed for $\tetapi$-a.a.\ $(x,y) \in \Ed$ and almost all $r\in (s,t)$ we have 
%%%%    \[
%%%%  |  \chi_{A_M}(r,x,y)  w_r(x,y) \dnabla\varphi(x,y) | \leq \uppsi \left(\frac{w_r(x,y)}{\upalpha(u_r(x),u_r(y))} \right) \upalpha(u_r(x),u_r(y))  + \uppsi^*(\dnabla\varphi(x,y)) \upalpha(u_r(x),u_r(y))
%%%%    \]
%%%%    and by  \eqref{finite-entropy+action}  and \eqref{eq:test_extend}  the above right-hand side provides an admissible 
%%%%     dominating function. \EEE
%%%%  %   \TODO{exploiting the fact that $ \upalpha(u_r(x),u_r(y)) \leq C$, we could extend the above argument to all $\varphi\in \mathcal{X} \cap \Bb(V)$}
%%%%     
%%%%     
%%%%   \RNEW   Altogether, we then obtain that the pair $(\bar\rho, \jj)$ satisfies the continuity equation for all test functions in $\Bb(V)$ such that \eqref{eq:test_extend} holds.
%%%%   A fortiori, exploiting the fact that $u$ satisfies   \eqref{strong-bounds}, we conclude that 
%%%%    \[
%%%%        \int_V\varphi\,\dd\bar\rho_t - \int_V\varphi\,\dd\bar\rho_s = \iiint_{[s,t]\times\Ed} \dnabla\varphi\,\dd\jj_\Lebone \quad\text{for all $\varphi\in \mathcal{X} \cap \Bb(V)$}.
%%%%    \]
%%%%    Finally, we claim that $\bar\rho = \rho$. \RNEW  Taking into account that  the pair $(\rho,\bj)$ satisfies the continuity equation in the sense of Def.\ \ref{def:CE}, we 
%%%%    gather that 
%%%%      \[
%%%%        \int_V\varphi\,\dd\rho_t - \int_V\varphi\,\dd\rho_s = \iiint_{[s,t]\times\Ed} \dnabla\varphi\,\dd\jj_\Lebone \quad\text{for all $\varphi\in  \Lip_\textrm{b}(V)$}.
%%%%    \]
%%%%  \RBS  Then, we gather that 
%%%%      \[
%%%%        \int_V \varphi(x)\, \bar\rho_t (\dd x) = \int_V \varphi(x) \, \rho_t (\dd x)  \qquad\text{for all $\varphi \in  \Lip_{\mathrm{b}}(V)$ and $t\in[0,T]$}.
%%%%    \]
%%%%\RBS   Since, by assumption, $\Lip_{\mathrm{b}}(V)$ is sequentially dense in $ \mathcal{X} \cap \Bb(V)$ w.r.t.\ the weak topology of
%%%%$\rmL^1(V;\pi)$, 
%%%%we 
%%%%then conclude
%%%%    \[
%%%%        \int_V \varphi(x)\, \bar\rho_t (\dd x) = \int_V \varphi(x) \, \rho_t (\dd x)  \qquad\text{for all $\varphi\in \mathcal{X} \cap \Bb(V)$ and $t\in[0,T]$}.
%%%%    \]
%%%%    Then,  we have proven the 
%%%%validity of the continuity equation for all $\varphi \in \Bb(V) $ fulfilling \eqref{eq:test_extend}.
%%%%\end{proof}
%%%%\medskip
%%%%
%%%%\hrule
%%%%\hrule
%%%%\hrule
%%%%\medskip
%%%%
%%%%
%%%%\noindent \RCR \textbf{Jasper's argument:} 

\RCR Prior to carrying out the proof of Prop.\ \ref{prop:ENH-CE} we need to derive a key estimate  from
%We start by  deriving from %Assumption  \ref{Ass:D-additional}
the asymptotic condition
 $\uppsi^*(\xi) \simeq |\xi|^2 
 $ as $|\xi| \to 0$.
\begin{lemma}
Let $\uppsi^*$ comply with Assumption \ref{Ass:D}.  Then,
\begin{equation}
\label{Olli-est}
\forall\, \beta>0 \ \forall\, M>0  \ \exists\, C_{\beta,M}>0: \quad \uppsi^*(\beta\xi)\leq C_{\beta,M} \uppsi^*(\xi)\qquad\forall\, \xi \in [{-}M,M].
\end{equation}
\end{lemma}
\begin{proof}
Clearly,
if $|\beta|\leq 1$ then $\uppsi^*(\beta \xi) \leq \uppsi^*(\xi)$ since $\uppsi^* $ is non-decreasing. Now let $|\beta | \geq 1$ and let $r>0$
be as in  \eqref{easy-conseq}.
%%%
We  estimate $\uppsi^*(\beta \xi)$ distinguishing two cases:
\begin{enumerate}
\item $|\beta \xi| \leq r$: then $|\xi| \leq \tfrac{r}{|\beta|} <r$, since $|\beta|>1$. By  \eqref{easy-conseq}  we have
\begin{equation}
\label{easy-case1}
\uppsi^*(\beta \xi) \leq  \frac32 c_0 |\beta\xi|^2  \leq  \frac32 c_0 |\beta|^2  \frac{2}{c_0} \uppsi^*(\xi) = 3  |\beta|^2 \uppsi^*(\xi) \,.
\end{equation}
\item $|\beta \xi| > r$: then, $|\xi|> \frac r\beta$, so that, by monotonicity of $\uppsi^*$, we find $\uppsi^*(\tfrac r{\beta}) 
\leq \uppsi^*(\xi)$, and thus
\begin{equation}
\label{easy-case2}
\uppsi^*(\beta \xi) \leq  \frac{\uppsi^*(\xi)}{\uppsi^*(\tfrac r{\beta}) }\uppsi^*(\beta \xi) \leq \frac{\uppsi^*(\xi)}{\uppsi^*(\tfrac r{\beta}) }\uppsi^*(\beta M)
\,,
\end{equation}
where for the last estimate we have used that $|\xi| \leq M$.
\end{enumerate}
 Combining \eqref{easy-case1} and \eqref{easy-case2} we conclude estimate \eqref{Olli-est}
with
 $
C_{\beta,M}= \max\left\{ 3  |\beta|^2, \frac{\uppsi^*(\beta M)}{\uppsi^*(\tfrac r{\beta}) }\right\}\,.
$
\end{proof}
As an immediate consequence of this result, we have that the spaces introduced in \eqref{X-L-Orli} and \eqref{X-M-Orli}  do coincide.
\begin{corollary}
\label{cor:identical-spaces}
Let $\uppsi^*$ comply with Assumption \ref{Ass:D}.
Then,  $\mathcal{X}^{\uppsi^*} =  \rmX^{\uppsi^*}$.
\end{corollary}
\begin{proof}
Clearly, it suffices to prove that $  \rmX^{\uppsi^*} \subset \mathcal{X}^{\uppsi^*}$. For this, let us fix 
$\varphi \in  \rmX^{\uppsi^*}$. Since $\varphi \in \Bb(V)$, we have that $\overline\nabla \varphi 
\in \Bb(E)$ and thus we can apply estimate \eqref{Olli-est} with $M = \sup_{(x,y)\in E}|\overline \nabla \varphi(x,y)|$, which then ensures that 
\[
\forall\, \beta>0: \ \ \iint_{\Ed} \uppsi^*(\beta \overline\nabla \varphi (x,y)) \tetapi(\dd x \dd y )\leq C_{\beta,M} \iint_{\Ed} \uppsi^*(\overline\nabla \varphi (x,y)) \tetapi(\dd x \dd y )<\infty\,.
\]
Hence, $\overline \nabla \varphi \in \mathcal{M}^{\uppsi^*}(\Ed;\tetapi)$, and thus $\varphi \in  \mathcal{X}^{\uppsi^*}$.
\end{proof}
\par
We now verify that  bounded Lipschitz functions belong to the space $\mathcal{X}^{\uppsi^*} $. 
%
\begin{lemma}
\label{l:Lipb-r-nice}
Let $\uppsi^*$ comply with Assumption \ref{Ass:D}.  Then, $\Lipb(V) \subset \mathcal{X}^{\uppsi^*} $.
\end{lemma}
\begin{proof}
Let us fix an arbitrary $\varphi \in \Lipb(V)$, with norm  $\lambda:=\| \varphi \|_{\Lip_{\mathrm{b}}(V)}$. 
 Recall
\eqref{easy-conseq}. Then, by the monotonicity of $\uppsi^*$ we have that 
\[
\begin{aligned}
&
\iint_{\Ed} \uppsi^*(\overline \nabla \varphi(x,y)) \tetapi (\dd x \dd y ) 
\\
& \leq \iint_{\Ed} \uppsi^*(\lambda (1{\wedge}d(x,y))) \tetapi (\dd x \dd y )
\\ & 
=  \iint_{A_r} \uppsi^*(\lambda (1{\wedge}d(x,y))) \tetapi (\dd x \dd y ) + \iint_{\Ed{\setminus}A_r} \uppsi^*(\lambda (1{\wedge}d(x,y))) \tetapi (\dd x \dd y ) 
\doteq I_1+I_2\,,
\end{aligned}
\]
with 
$A_r = \{ (x,y)\in \Ed\, : \ \lambda (1{\wedge}d(x,y)) \leq r\} $. Now, by \eqref{easy-conseq}
\[
I_1 \leq  \frac{3}2 c_0 \iint_{A_r} \lambda^2 (1{\wedge}d(x,y))^2 \,  \tetapi (\dd x \dd y )<+\infty\,,
\]
while, again by the monotonicity of $\uppsi^*$  and condition \eqref{mitigation of singularity}, we have 
\[
I_2 \leq \uppsi^*(\lambda)\, \tetapi (\Ed{\setminus}A_r) <+\infty\,.
\]
All in all, we have shown that $\iint_{\Ed} \uppsi^*(\overline \nabla \varphi(x,y))\, \tetapi (\dd x \dd y )<+\infty$, so that $\varphi \in \rmX^{\uppsi^*}$. 
In view of Corollary \ref{cor:identical-spaces}, we conclude that 
$\varphi \in \mathcal{X}^{\uppsi^*}$.
\TODO{Shorten prove above!}
\end{proof}
\par
Finally, prior to  carrying out the proof of Prop.\ \ref{prop:ENH-CE}, we pin down a key fact: for any $(\xi_n)_n \subset \rmL^{\uppsi^*} (\Ed; \tetapi)$ we have that 
\begin{equation}
\label{characterization-of-Orlicz}
\|\zeta_n \|_{ \rmL^{\uppsi^*} (\Ed; \tetapi)} \to 0 \quad \text{if and only if} \quad \lim_{n\to\infty}\iint_{\Ed} \uppsi^*(\beta \,  \zeta_n(x,y)) \, \tetapi (\dd x \dd y) =0 \text{ for all } \beta >0\,.
 \end{equation}
 \TODOR{\EEE Which reference for \eqref{characterization-of-Orlicz}? I have found it only in the (unpublished) lecture notes by C.\ L\'eonard...}
 %Indeed, we have 
% \begin{equation}
 %\label{cvg-small-Orl}
% \lim_{n\to\infty} \iint_{\Ed}   \uppsi^* \big[ \beta \big(\overline\nabla \widehat{\varphi}_n(x,y){-} \overline\nabla \varphi(x,y)\big) \big]  \, \tetapi (\dd x \dd y) =0\qquad \text{for all } \beta>0\,.
 %\end{equation
%%%
\begin{proof}[Proof of Proposition \ref{prop:ENH-CE}]
Preliminarily, we observe that, since $u$ enjoys the bounds   \eqref{strong-bounds}, showing \eqref{desired-CE} for all test functions $\varphi \in \Bb(V)$ fulfilling
\eqref{eq:test_extend} is equivalent to showing it for all test functions  $\varphi \in  \rmX^{\uppsi^*}$. Thus, let us fix  $
\varphi \in  \rmX^{\uppsi^*}=  \mathcal{X}^{\uppsi^*}$. By Assumption \ref{Ass:F-bis}, there exists a sequence $(\varphi_n)_n \subset 
\Lipb(V)$ suitably 
approximating $\varphi$.
% Furthermore, by Lemma \ref{l:cutoff}  we may suppose  that 
%\begin{equation}
%\label{unifornly-bounded-varphin} 
%\exists M>0\, \  \forall\, n \in \N \ \forall\, (x,y)\in E\,: \qquad  |\overline \nabla (\varphi_n{-}\varphi)(x,y)| \leq M 
%\end{equation}
%%
 We now send $n\to \infty$ in  the continuity equation tested by $\varphi_n$, i.e.\ in 
  \[
        \int_V\varphi_n(x)\,\rho_t (\dd x) - \int_V\varphi_n(x)\, \rho_s  (\dd x) = \iiint_{[s,t]{\times}\Ed} \dnabla\varphi_n(x,y)\, \jj_\Lebone(\dd r \dd x \dd y)  \text{ for all } 0\leq s \leq t\leq T\,.
            \]
            Since $\varphi_n \weakto \varphi$ in $\rmL^1(V;\pi)$, we pass to  the limit on the left-hand side. 
            As for the right-hand side, 
            we will show that 
            \begin{equation}
            \label{RHSto0}
             \iiint_{[s,t]{\times}\Ed}  \dnabla\varphi_n(x,y)  \, \jj_\Lebone(\dd r \dd x \dd y)  \longrightarrow      \iiint_{[s,t]{\times}\Ed}   \dnabla\varphi(x,y)  \, \jj_\Lebone(\dd r \dd x \dd y) \,.
            \end{equation}
          For this, we use that 
            \[
            \begin{aligned}
          & \iiint_{[s,t]{\times}\Ed}   \left|\dnabla\varphi_n(x,y){-} \dnabla\varphi(x,y) \right|  \, \jj_\Lebone(\dd r \dd x \dd y) 
         \\
         & =\frac1{\beta}\int_s^t  \iint_{E_\upalpha^r} \beta  \left|\dnabla\varphi_n(x,y){-} \dnabla\varphi(x,y) \right|
          \frac{w_r(x,y)}{\upalpha(u_r(x), u_r(y))} \upalpha(u_r(x), u_r(y)) \tetapi(\dd x \dd y ) \, \dd r
         \\
         & \stackrel{(1)}\leq \frac1{\beta}\int_s^t  \iint_{E_\upalpha^r} \uppsi^*(\beta (\dnabla\varphi_n(x,y){-} \dnabla\varphi(x,y) )) \upalpha(u_r(x), u_r(y)) \tetapi(\dd x \dd y ) \, \dd r
        \\
        & \quad 
         +  \frac1{\beta}\int_s^t  \iint_{E_\upalpha^r} \uppsi\left( \frac{w_r(x,y)}{\upalpha(u_r(x), u_r(y))} \right) \upalpha(u_r(x), u_r(y)) \tetapi(\dd x \dd y ) \, \dd r
         \end{aligned}
                     \]
          with the set $E_\upalpha^r = \{ (x,y)\in \Ed\, : \ \upalpha(u_r(x), u_r(y))>0 \} $ fulfilling $|\bj_r|(\Ed{\setminus}E_\upalpha^r )=0$ for 
       $\Lebone$-a.a.\ $r\in (0,T)$, cf.\ \eqref{nice-representation}, and {\footnotesize (1)} due to Young's inequality.
       Now, for every fixed $\beta>0$ we have
       \[
       \begin{aligned}
       &
       \int_s^t  \iint_{E_\upalpha^r} \uppsi^*(\beta (\dnabla\varphi_n(x,y){-} \dnabla\varphi(x,y) )) \upalpha(u_r(x), u_r(y)) \, \tetapi(\dd x \dd y ) \, \dd r
       \\
       & 
       \leq C_{\overline U}   (t{-}s)  \iint_{\Ed} \uppsi^*(\beta (\dnabla\varphi_n(x,y){-} \dnabla\varphi(x,y) )) \, \tetapi(\dd x \dd y ) \, \dd r \doteq  C_{\overline U} (t{-}s) I_n
       \end{aligned}
       \]
       with $ C_{\overline U}$ from \eqref{for-later-use-uppalpha}.
       % and  $C_{\beta,M}$ from \eqref{Olli-est}, where $M$ fulfills \eqref{unifornly-bounded-varphin}. 
       All in all, we conclude that for every $\beta>0$
       \[
       \begin{aligned}
       &
   \limsup_{n\to \infty}     \iiint_{[s,t]{\times}\Ed}   \left|\dnabla\varphi_n(x,y){-} \dnabla\varphi(x,y) \right|  \, \jj_\Lebone(\dd r \dd x \dd y) 
\\ &    \leq  C_{\overline U}  (t{-}s) \lim_{n\to\infty} I_n + \frac1{\beta}\int_s^t   \scrR(\rho_r,\bj_r)\, \dd r
   =   \frac1{\beta}\int_s^t   \scrR(\rho_r,\bj_r)\, \dd r\,,
   \end{aligned}
       \]
       since $I_n \to 0$ by Assumption \ref{Ass:F-bis}(2) and the characterization of Orlicz convergence provided by \eqref{characterization-of-Orlicz}. 
         By the arbitrariness of $\beta$, we conclude \eqref{RHSto0}. We have thus shown that 
        for all $ 0\leq s \leq t\leq T$
         \[
        \int_V\varphi(x)\,\rho_t (\dd x) - \int_V\varphi(x)\, \rho_s  (\dd x) = \iiint_{[s,t]{\times}\Ed} \dnabla\varphi(x,y)\, \jj_\Lebone(\dd r \dd x \dd y) \qquad \forall\, \varphi \in  \rmX^{\uppsi^*}\,,
            \]
       which
        finishes the proof. 
\end{proof}       
       
         
           
           





\RBS 
\subsection{A chain rule identity}
Exploiting Proposition \ref{prop:ENH-CE}, we show that along a solution of the continuity equation satisfying bounds \eqref{finite-entropy+action} \emph{and} \eqref{strong-bounds}, the chain rule holds. 
\begin{proposition}[Chain-rule]
\label{prop:CR}
 Let $(\rho,\jj)\in\CE 0T$ fulfill \eqref{finite-entropy+action}   and \eqref{strong-bounds}. Then,
 \begin{equation}
 \label{chain-rule}
  -  \frac{\dd}{\dd t} \calS(\rho_t) =
     \iint_{\Ed} ({-}\dnabla \phi'\circ u_t)(x,y)\, \jj_t(dxdy)
     \leq  \scrR(\rho_t,\jj_t) + \scrD(\rho_t) \quad \foraa\, t \in (0,T)\,.
 \end{equation} \EEE
 In particular, the pair $(\rho,\jj)$ satisfies the lower energy-dissipation inequality \eqref{LEDE}. 
\end{proposition}
\begin{proof}
    We begin by showing the absolute continuity of $t\mapsto \calS(\rho_t)$. 
    \RBS Preliminarily, we recall that,  in view of \eqref{strong-bounds},
    \begin{equation}
    \label{needed-4-rigour}
     \upalpha(u_t(x),u_t(y)) \geq \underline{\upalpha}: = \min_{(u,v)\in [\underline U, \overline U]{\times} [\underline U, \overline U]} \upalpha(u,v)>0
     \quad \text{for } \pi\text{-a.a. } x,y \in V \quad \text{for all } t \in [0,T]\,.
         \end{equation}
  \EEE  Now, 
    by convexity, we obtain for all $0\leq s \leq t \leq T$
    \begin{align*}
       &  \calS(\rho_t) - \calS(\rho_s)
       \\
        &\le \int_V \phi'(u_t(x)) [u_t(x)-u_s(x)]\,\pi(\dd x) \\
        &\stackrel{(1)}{=} \int_s^t \iint_{\RBS E'} (\dnabla \phi'\circ u_t)(x,y)\,\jj_r(\dd x \dd y) \dd r \\
        &\stackrel{(2)}{=} \int_s^t \iint_{\RBS E'}  \RBS  \frac1{\upalpha(u_r(x),u_r(y))} \EEE (\dnabla \phi'\circ u_t)(x,y) 
        \RBS \upalpha(u_r(x),u_r(y)) \EEE
        \,\jj_r(\dd x \dd y) \dd r \\ 
        &\stackrel{(3)}\le \int_s^t \scrR(\rho_r,\jj_r)\,dr + \int_s^t \iint_{\RBS E'}  \uppsi^*({-}\dnabla (\phi'\circ u_t)(x,y))\,\upalpha(u_r(x),u_r(y))\,\tetapi(\dd x \dd y) \dd r \\
        &\le \int_s^t \scrR(\rho_r,\jj_r)\,dr + \int_s^t \iint_{\RBS E'}  \uppsi^*((\dnabla \phi'\circ u_t)(x,y))\, \upalpha(u_t(x),u_t(y)) \frac{\upalpha(u_r(x),u_r(y))}{\upalpha(u_t(x),u_t(y))}
        \,\tetapi(\dd x \dd y) \dd r  \\
        &\le \int_s^t \scrR(\rho_r,j_r)\,dr + \RBS \frac{C_{\overline{U}}}{ \underline{\upalpha}} \EEE  \Fish(\rho_t)|t-s|\,.
    \end{align*}
    \RBS Here,
    {\footnotesize (1)} follows from choosing the test function $\varphi =  \phi'\circ u_t $ in the continuity equation, thanks to Proposition \ref{prop:ENH-CE}; 
    {\footnotesize (2)}     is justified by \eqref{needed-4-rigour}, while     {\footnotesize (3)}  ensues from Young's inequality and from the fact that $\uppsi^*$ is an even function.
Analogously, we have 
    \begin{align*}
         \calS(\rho_t) - \calS(\rho_s)
        &\geq \int_V \phi'(u_s(x)) [u_t(x)-u_s(x)]\,\pi(\dd x) 
        \\
        &  = - \int_s^t \iint_{\RBS E'} {-}\dnabla (\phi'\circ u_s)(x,y)\,\jj_r(\dd x \dd y) \dd r
       % \\
      %  & \geq  - \int_s^t \scrR(\rho_r,\jj_r)\,dr - \int_s^t \iint_{\RBS E'}  \uppsi^*({-}\dnabla (\phi'\circ u_s)(x,y))\, \upalpha(u_t(x),u_t(y)) \frac{\upalpha(u_r(x),u_r(y))}{\upalpha(u_t(x),u_t(y))}
        %\,\tetapi(\dd x \dd y) \dd r 
        \\
        & \geq - \int_s^t \scrR(\rho_r,j_r)\,dr -  \frac{C_{\overline{U}}}{ \underline{\upalpha}}   \Fish(\rho_s)|t-s|\,.
\end{align*}
All in all, we obtain
\[
    |\calS(\rho_t) - \calS(\rho_s)| \le \int_s^t \scrR(\rho_r,j_r)\,dr + \frac{C_{\overline{U}}}{ \underline{\upalpha}}\Bigl( \Fish(\rho_t)+ \Fish(\rho_s)\Bigr)|t-s| \qquad \text{for all } 0 \leq s \leq t \leq T\,.
\]
Applying \cite[Lemma 1.2.6]{AGS08}, we infer that the mapping 
$[0,T]\ni t \mapsto \calS(\rho_t)$ is in $W^{1,1}(0,T)$. 
\par
Let us  now fix a point $t$, out of a negligible set, where $\frac{\dd}{\dd t}  \calS(\rho_t)$ exists.
With the same calculations as above 
we obtain
\begin{align*}
    \frac{1}{h}\bigl[\calS(\rho_t) - \calS(\rho_{t+h})\bigr] &  \leq \frac1h \int_V \phi'(u_t(x)) [u_t(x)-u_{t+h}(x)]\,\pi(\dd x) 
    \\
    &
    = -  \int_{t}^{t+h} \iint_{E'} {-}\dnabla (\phi'\circ u_t)(x,y)\,\jj_r(\dd x \dd y) \dd r 
    \end{align*}
    and, analogously,
    \[
     \frac{1}{h}\bigl[\calS(\rho_t) - \calS(\rho_{t+h})\bigr] \geq   \int_{t}^{t+h} \iint_{E'} {-}\dnabla (\phi'\circ u_{t+h}))(x,y)\,\jj_r(\dd x \dd y) \dd r 
    \]
    Dividing both inequalities by $h>0$ and letting $h\down0$ we conclude that 
    \[
   -  \frac{\dd}{\dd t} \calS(\rho_t) =
     \iint_{\Ed} {-}\dnabla (\phi'\circ u_t)(x,y)\, \jj_t(dxdy)\,,
\] 
   whence \eqref{chain-rule}.
%    
%    
%    \le \frac{1}{h}\int_t^{t+h} \calR(\rho_r,j_r)\,dr + \frac{1}{h}\int_t^{t+h} \iint_\Ed \uppsi^*(-(\dnabla \phi'\circ u_t)(x,y))\,\upalpha_r(x,y)\,\teta(dxdy)\,dr.
%\end{align*}
%Passing to the limit $h\to 0^+$ gives
%\begin{align*}
%    -\frac{d}{dt} \calS(\rho_t) &\le \calR(\rho_t,j_t) + \iint_E \uppsi^*(-(\dnabla \phi'\circ u_t)(x,y))\,\upalpha_t(x,y)\,\teta(dxdy) 
%    = \calR(\rho_t,j_t) + \calD(\rho_t)
%\end{align*}
%
\end{proof}

\subsection{The lower energy-dissipation inequality and conclusion of the proof}
We now extend the validity of \eqref{LEDE}.
\begin{proposition}
\label{prop:LEDE}
 Let $(\rho,\jj)\in\CE 0T$ fulfill \eqref{finite-entropy+action} 
and suppose that for $u_t = \frac{\dd \rho_t}{\dd \pi}$ the upper bound
\[
\exists\, \overline U>0 \, : \quad 0 \leq  u_t(x) \leq \overline U \quad \text{for } \pi\text{-a.a. } x \in V \quad \text{for all } t \in [,T]
\]
holds.
Then, the pair $(\rho,\jj)$ satisfies the lower energy-dissipation inequality 
\begin{equation}
\label{LEDE-t}
\int_0^t  \left(  \scrR(\rho_r,\jj_r) + \scrD(\rho_r) \right) \dd r + \calS(\rho_t)  \geq   \calS(\rho_0) \qquad\text{for all } t \in [0,T]\,.
\end{equation}
\end{proposition}
\begin{proof}
Let us use $\mathcal{L}_{[0,t]}(\rho,\jj) $ as a place-holder for the left-hand side of \eqref{LEDE-t} and consider the curve of measures
$\rho^\theta: [0,T] \to\calM^+(V)$
\[
\rho_t^\theta: = (1{-}\theta)\rho_t+ \theta \pi \quad t \in [0,T], \ \theta \in [0,1]\,. 
\]
Clearly, $(\rho^\theta, (1{-}\theta) \jj) \in \CE 0T$; furthermore, by convexity of $\calS$ and $\scrR$, 
\[
\begin{cases}
\displaystyle     \sup\nolimits_{t\in [0,T]} \scrE(\rho_t^\theta) \leq (1{-}\theta)   \sup\nolimits_{t\in [0,T]} \scrE(\rho_t)<+\infty,
\smallskip
\\
\displaystyle  \int_0^T \scrR(\rho_t^\theta, (1{-}\theta) \bj_t)\, \dd t \leq (1{-}\theta)\int_0^T \scrR(\rho_t, \bj_t)\, \dd t <+\infty.
\end{cases}
\]
What is more, $u_t^\theta =\frac{\dd \rho_t^\theta}{\dd \pi} $ satisfies \eqref{strong-bounds}. Therefore, Proposition \ref{prop:CR} applies, yielding 
$  \calS(\rho_0^\theta) \leq \mathcal{L}_{[0,t]}(\rho^\theta, (1{-}\theta)\jj)$. Thus, by convexity of $ \mathcal{L}_{[0,t]}$ we infer
\[
 \calS(\rho_0^\theta) \leq \mathcal{L}_{[0,t]}(\rho^\theta, (1{-}\theta)\jj) \leq (1{-}\theta) \mathcal{L}_{[0,t]}(\rho, \jj) + \theta \mathcal{L}_{[0,t]}(\pi, \boldsymbol{0}) = (1{-}\theta) \mathcal{L}_{[0,t]}(\rho, \jj)  +\theta  \calS(\pi)
\]
since we can immediately check that $ \mathcal{L}_{[0,t]}(\pi, \boldsymbol{0}) = \calS(\pi)$. Then, it suffices to send $\theta \downarrow 0$ in the above inequality, and \eqref{LEDE-t} ensues.
\end{proof}
\par
We are now in a position to conclude the 
\underline{\textbf{proof of Theorem \ref{th:main}.}} 
\begin{itemize}
\item[-]
As for \text{part (1)}, we  apply Proposition \ref{prop:LEDE} to the pair $(\rho,\jj) \in \CE0T$ obtained in Section \ref{s:5}. 
Since $(\rho,\jj)$ also satisfies the upper energy-dissipation estimate thanks to Lemma \ref{l:UEDE}, we conclude that $(\rho,\jj)$ is a solution of the $(\calS,\scrR,\scrR^*)$
system in the sense of Definition \ref{def:R-Rstar-balance}. 
\item[-] As for \text{part (2)}, it suffices to apply Proposition \ref{prop:CR}. 
\end{itemize}
\QED

\EEE
\section{More on Assumption \ref{Ass:F-bis}}
\label{s:lip} 

\TODOR{For the time being I have moved Lemma \ref{l:cutoff} here....}
%%
\RCR
First of all, we  improve the approximation property
required in  Assumption \ref{Ass:F-bis}. Namely,  Lemma  \ref{l:cutoff} below we show that, if   Assumption \ref{Ass:F-bis} holds,
we can indeed
construct a sequence $(\widehat{\varphi}_n)_n $ such that, in addition to the convergences
in Ass.\ \ref{Ass:F-bis}, we have that 
$\|\overline\nabla \widehat{\varphi}_n \|_\infty = \sup_{x,y\in V} |\overline\nabla\widehat{\varphi}_n(x,y)|$ is uniformly bounded. 
%As we will see, for  $ \overline\nabla \widehat{\varphi}_n$ we will be able to obtain convergence to $ \overline\nabla\varphi$
%in the \emph{small} Orlicz space  $\mathcal{X}^{\uppsi^*}$,
%cf.\ \eqref{cvg-small-Orl} below which, indeed, comes as no surprise in view of Corollary \ref{cor:identical-spaces}.
%%%%
\begin{lemma}
\label{l:cutoff}
Under Assumptions \ref{Ass:D} and \ref{Ass:F-bis}, 
for every $\varphi \in \mathcal{X}^{\uppsi^*}$
 there exists a sequence $(\widehat{\varphi}_n)_n \subset \Lipb(V)$ such that as $n\to \infty$
 \begin{enumerate}
  \item $ \sup_{n
\in \N} \|\overline\nabla \widehat{\varphi}_n \|_\infty
 \leq 4\|\varphi\|_{\infty}$;
 \item $\widehat{\varphi}_n\to\varphi$ in $\rmL^1(V;\pi)$;
 \item $\overline\nabla \widehat{\varphi}_n \to \overline\nabla \varphi$ in $\rmL^{\uppsi^*}(\Ed;\tetapi)$. 
 %Indeed, we have 
% 
 \end{enumerate}
 \end{lemma}
% \TODOBS{\EEE It would be sufficient indeed to have $\sup_{n
%\in \N} \|\overline\nabla \widehat{\varphi}_n \|_\infty<\infty$, but I don't know how to prove this...}
\begin{proof}
Obviously, we may suppose that $\varphi \not\equiv 0$. 
Let us introduce the truncation operator
\[
\tau_\ell: \R \to \R, \quad 
\tau_\ell(x):= \begin{cases}
x & \text{if } |x|\leq \ell,
\\
\ell & \text{if } x >\ell,
\\
-\ell  &  \text{if } x <-\ell,
\end{cases}
\quad \text{with  the place-holder } \ell:= 2\|\varphi\|_{\infty},
\]
and let us  define 
\begin{equation}
\label{truncated-phin}
\widehat{\varphi}_n(x):= \tau_\ell(\varphi_n(x))\,.
\end{equation}
Clearly, $ \widehat{\varphi}_n \subset \Lipb(V)$, with   $ \sup_{n
\in \N} \|\overline\nabla \widehat{\varphi}_n \|_\infty
\leq 2\ell$.  We now check the remaining items of the statement in separate claims.  
\smallskip

\par
\noindent
{\sl \textbf{Claim $1$:} we have  $\widehat{\varphi}_n\to\varphi$ in $\rmL^1(V;\pi)$.}
\\
Since $\ell =2 \|\varphi\|_{\infty}$, we have that 
\[
A_n:= \{ x\in V\, : \ |\varphi_n(x) | >\ell \} \subset\left \{ x\in V\, : \ |\varphi_n(x){-} \varphi(x) | >\frac{\ell}2 \right \}\,.
\]
Therefore, since $\varphi_n \to \varphi$ in $\rmL^1(V;\pi)$, we infer that 
\[
\lim_{n\to \infty} \pi (A_n) \leq \lim_{n\to \infty} \pi \left(\left\{ x\in V\, : \ |\varphi_n(x){-} \varphi(x) | >\frac{\ell}2 \right \}\right) =0,
\]
so that 
\[
 \int_{A_n} |\widehat{\varphi_n}{-}\varphi(x)| \, \pi(\dd x) \longrightarrow  0 \text{ as } n \to \infty\,.
\]
In this way, we have that 
\[
\lim_{n\to\infty}\| \widehat{\varphi}_n{-} \varphi\|_{\rmL^1(V;\pi)} = \lim_{n\to\infty} \int_{V{\setminus}A_n} |\varphi_n(x){-}\varphi(x)| \, \pi(\dd x) =0\,.
\]
\smallskip

\par
\noindent
{\sl \textbf{Claim $2$:} we have 
\begin{equation}
 \label{cvg-small-Orl}
 \lim_{n\to\infty} \iint_{\Ed}   \uppsi^* \big[ \beta \big(\overline\nabla \widehat{\varphi}_n(x,y){-} \overline\nabla \varphi(x,y)\big) \big]  \, \tetapi (\dd x \dd y) =0\qquad \text{for all } \beta>0\,.
 \end{equation}
%$\overline\nabla \widehat{\varphi}_n \to \overline\nabla \varphi$ in $\rmL^{\uppsi^*}(\Ed;\tetapi)$;
}
\\
First of all, we observe that, since $\widehat{\varphi}_n\to \varphi$ $\pi$-almost everywhere in $V$, by Lemma \ref{l:properties-tetakappa}
we have 
\begin{equation}
\label{pointwise-cvg-gradients}
\overline\nabla \widehat{\varphi}_n  \to  \overline\nabla \varphi \qquad \tetapi\text{-a.e.\ in } \Ed\,.
\end{equation}
A direct calculation shows that 
$|\overline\nabla \widehat{\varphi}_n(x,y)| \leq |\overline\nabla \varphi_n (x,y)|$ for every $(x,y) \in \Ed$, therefore we have 
\[
f_n(x,y):= |\overline\nabla (\widehat{\varphi}_n {-} \varphi)(x,y)|  \leq  |\overline\nabla \widehat{\varphi}_n(x,y) |  +  |\overline\nabla \varphi(x,y)| 
\leq  |\overline\nabla \varphi_n (x,y)| +  |\overline\nabla \varphi(x,y)|  \doteq g_n(x,y) \qquad \text{for all }(x,y) \in \Ed\,,
\]
and, a fortiori, since $\uppsi^*$ is non-decreasing on $[0,+\infty)$ we have 
\begin{equation}
\label{domination}
\uppsi^* (\beta f_n(x,y)) \leq \uppsi^* (\beta g_n(x,y)) \qquad \text{for all }(x,y) \in \Ed\,,  \ \text{for all } \beta>0\,.
\end{equation}
Now, observe that, while $\beta f_n\to 0$ $\tetapi$-a.e.\ in $ \Ed$ by \eqref{pointwise-cvg-gradients}, we  have that 
\[
\beta g_n\to \beta g    \quad \tetapi\text{-a.e.\ in } \Ed \qquad \text{with }  g:=   2|\overline\nabla \varphi| \,.
\]
We will now prove that, indeed, 
\begin{equation}
\label{next-aim-gn}
\lim_{n\to\infty} \iint_{\Ed} \left|  \uppsi^* (\beta g_n(x,y)){-}   \uppsi^* (\beta g(x,y)) \right|  \, \tetapi (\dd x \dd y) =0\qquad  \text{for all } \beta>0\,..
\end{equation}
Combining \eqref{domination} and  a modified version of the dominated convergence theorem (cf., e.g., \cite[\S 2, Thm.\ 2.8.8]{Bogachev07}), we 
will then conclude that $\lim_{n\to\infty} \iint_{\Ed}  \uppsi^*(\beta f_n(x,y))  \, \tetapi (\dd x \dd y) =0$, namely the desired
\eqref{cvg-small-Orl}.
Let us then check \eqref{next-aim-gn}. First of all, we observe that 
\[
|g_n{-}g| = ||\overline\nabla \varphi_n|{-}|\overline\nabla \varphi|| \leq  |\overline\nabla \varphi_n{-}\overline\nabla \varphi|    \qquad \tetapi\text{-a.e.\ in } \Ed\,,
\]
hence $\uppsi^*( |g_n{-}g| ) \leq \uppsi^*(  |\overline\nabla \varphi_n{-}\overline\nabla \varphi| ) $  $\tetapi$-a.e.\ in $ \Ed$.  Hence, from
$\overline\nabla \varphi_n \to \overline\nabla \varphi$ in  $\rmL^{\uppsi^*}(\Ed;\tetapi)$,
taking into account the characterization of Orlicz convergence provided by \eqref{characterization-of-Orlicz}, 
 we deduce that 
\begin{equation}
\label{only-this}
\lim_{n\to\infty} \iint_{\Ed} \uppsi^*(\beta |g_n(x,y){-}g(x,y)| )  \, \tetapi (\dd x \dd y) =0 \qquad \text{for all } \beta>0\,.
\end{equation}
%\TODOBS{\EEE Caution! the convergence above CANNOT be improved to  $\lim_{n\to\infty} \iint_{\Ed} \uppsi^*(\beta |g_n(x,y){-}g(x,y)| )  \, \tetapi (\dd x \dd y) =0
%$ for all $\beta>0$ because for $ g_n= |\overline\nabla \varphi_n | +  |\overline\nabla \varphi)| $ we have LOST the uniform
%bounds granted by the truncation}
%%in view of 
%%\begin{equation}
%%\label{rely-on-this}
%%
%%\end{equation}
We now use the convexity inequality
\begin{equation}
\label{useful-convexity-ineq}
 \uppsi^*(a ) \leq \alpha  \uppsi^*(\alpha^{-1} b) + (1{-}\alpha)  \uppsi^*((1{-}\alpha)^{-1} (a{-}b)) \qquad \text{for all } a,b \in [0,+\infty ), \  \alpha \in (0,1)\,.
\end{equation}
Plugging in $a= \beta g_n$ and $b= \beta g$ we obtain
\[
\begin{aligned}
&  \iint_{\Ed}  \uppsi^*( \beta g_n )  \, \tetapi (\dd x \dd y) 
\\
& \leq  \alpha  \iint_{\Ed}  \uppsi^*(\alpha^{-1} \beta g )  \, \tetapi (\dd x \dd y) 
 + (1{-}\alpha)   \iint_{\Ed}   \uppsi^*((1{-}\alpha)^{-1} \beta(g_n{-}g)) \, \tetapi (\dd x \dd y) \,.
 \end{aligned}
\]
so that, in view of \eqref{only-this}, we infer 
\[
\limsup_{n\to\infty} \iint_{\Ed}  \uppsi^*( \beta g_n )  \, \tetapi (\dd x \dd y)  \leq \alpha  \iint_{\Ed}  \uppsi^*(\alpha^{-1} \beta g )  \, \tetapi (\dd x \dd y)  \quad \text{for all } \alpha \in (0,1)\,.
\]
Plugging then  $a= \beta g$ and $b= \beta g_n$ in  \eqref{useful-convexity-ineq}
we get 
\[
\iint_{\Ed}  \uppsi^*( \beta g )  \, \tetapi (\dd x \dd y)  \leq \alpha \liminf_{n\to\infty} \iint_{\Ed}  \uppsi^*( \alpha^{-1} \beta g_n )   \quad \text{for all } \alpha \in (0,1)\,,
\]
so that, letting $
\alpha \uparrow 1$ we conclude 
\[
\lim_{n\to\infty} \iint_{\Ed}  \uppsi^*( \beta g_n )  \, \tetapi (\dd x \dd y) =   \iint_{\Ed}  \uppsi^*( \beta g )  \, \tetapi (\dd x \dd y)  \quad \text{for all } \alpha \in (0,1)\,.
\]
Similarly, \eqref{next-aim-gn} follows from the enhanced convexity inequality 
\begin{equation}
\label{enhanced-convexity-ineq}
| \uppsi^*(a ){-} \uppsi^*(b )|  \leq \alpha  \uppsi^*(\alpha^{-1} b) - \uppsi^*(b ) + 
\alpha  \uppsi^*(\alpha^{-1} a) - \uppsi^*(a ) +
(1{-}\alpha)  \uppsi^*((1{-}\alpha)^{-1} |a{-}b|)
\end{equation}
for all $a,b \in [0,+\infty )$, $ \alpha \in (0,1)$: it suffices to choose $a= \beta g_n$ and $b= \beta g$ in  \eqref{enhanced-convexity-ineq}, and then to send first   $n\to \infty$
and the $\alpha 
\uparrow 1$. 
%\TODOBS{\EEE Caution!  But we cannot deduce that $\lim_{n\to\infty}   \iint_{\Ed}   \uppsi^*((1{-}\alpha)^{-1} \beta(g_n{-}g)) =0$ because we only have \eqref{only-this}!
%This is regardless of $\beta$! Even if we set $\beta=1$ we still have $  \iint_{\Ed}   \uppsi^*((1{-}\alpha)^{-1} (g_n{-}g)) \, \tetapi (\dd x \dd y) 
%$ for which we have no information... }


%%%Finally, we  verify item $\# 2$ of the statement. 
%%%{\color{red} My idea to show
%%%\[
%%%\lim_{n\to\infty} \| \overline\nabla \widehat{\varphi}_n{-} \overline\nabla\varphi   \|_{\rmL^{\uppsi^*}(\Ed;\tetapi)}   =0,
%%%\]
%%%would be to split the above integral by partitioning $E'$ in the subsets corresponding to the various cases, and then use the dominated convergence theorem
%%%(since we have pointwise convergence). But I only have one-sided inequalities, below, and $\tetapi$ is a singular measures.. so the $\rmL^\infty$ bounds that we have 
%%% won't work...}
%%%%\begin{equation}
%%%%\label{last-partproof}
%%%% \uppsi^*(\beta \overline\nabla [\widehat{\varphi}_n{-} \varphi] (x,y) ) \leq \uppsi^*(\beta \overline\nabla [\varphi_n{-} \varphi] (x,y) ) \qquad \text{for all } (x,y) \in \Ed \text{ and all } \beta>0
%%%%\,.
%%%%\end{equation}
%%%Indeed, we calculate 
%%%\[
%%%\begin{aligned}
%%%& \displaystyle \text{if } |\varphi_n(x)|,  \  |\varphi_n(y)| \leq \ell \qquad && \overline\nabla [\widehat{\varphi}_n{-} \varphi] (x,y) && \hspace{-0,3cm}  =     \overline\nabla [\varphi_n{-} \varphi] (x,y)  
%%%\\
%%%& \displaystyle \text{if } \varphi_n(x)>\ell,  \  |\varphi_n(y)| \leq \ell \qquad && \overline\nabla [\widehat{\varphi}_n{-} \varphi] (x,y)   &&   \hspace{-0,3cm} = \varphi_n(y) -\ell + \varphi(x) -  \varphi(y)  
%%%\\ & && 
%%% &&  \hspace{-0,3cm} > \varphi_n(y) - \varphi_n(x)  + \varphi(x) -  \varphi(y)  =  \overline\nabla [\varphi_n{-} \varphi] (x,y) 
%%% \\
%%% & \displaystyle \text{if } \varphi_n(x)<-\ell,  \  |\varphi_n(y)| \leq \ell \qquad && \overline\nabla [\widehat{\varphi}_n{-} \varphi] (x,y)   &&   \hspace{-0,3cm} = 
%%% \varphi_n(y) +\ell + \varphi(x) -  \varphi(y)  
%%%\\ & && 
%%% &&  \hspace{-0,3cm} < \varphi_n(y) - \varphi_n(x) + \varphi(x) -  \varphi(y)  =  \overline\nabla [\varphi_n{-} \varphi] (x,y) 
%%% \\
%%% & \displaystyle \text{if } \varphi_n(y)>\ell,  \  |\varphi_n(x)| \leq \ell \qquad && \overline\nabla [\widehat{\varphi}_n{-} \varphi] (x,y)   &&   \hspace{-0,3cm} = \ell -\varphi_n(x)  + \varphi(x) -  \varphi(y)  
%%%\\ & && 
%%% &&  \hspace{-0,3cm} < \varphi_n(y) - \varphi_n(x)  + \varphi(x) -  \varphi(y)  =  \overline\nabla [\varphi_n{-} \varphi] (x,y) 
%%% \\
%%% & \displaystyle \text{if } \varphi_n(y)<-\ell,  \  |\varphi_n(x)| \leq \ell \qquad && \overline\nabla [\widehat{\varphi}_n{-} \varphi] (x,y)   &&   \hspace{-0,3cm} = 
%%%-\ell -  \varphi_n(x)  
%%% + \varphi(x) -  \varphi(y)  
%%%\\ & && 
%%% &&  \hspace{-0,3cm} > \varphi_n(y) - \varphi_n(x) + \varphi(x) -  \varphi(y)  =  \overline\nabla [\varphi_n{-} \varphi] (x,y) 
%%% \\
%%% & \displaystyle \text{if } \varphi_n(y)>\ell,  \  \varphi_n(x)<- \ell \qquad && \overline\nabla [\widehat{\varphi}_n{-} \varphi] (x,y)   &&   \hspace{-0,3cm} = 
%%%2\ell -  \varphi_n(x)  
%%% + \varphi(x) -  \varphi(y)  
%%%\\ & && 
%%% &&  \hspace{-0,3cm} > \varphi_n(y) - \varphi_n(x) + \varphi(x) -  \varphi(y)  =  \overline\nabla [\varphi_n{-} \varphi] (x,y) 
%%% \\
%%% & \displaystyle \text{if } \varphi_n(y)<-\ell,  \  \varphi_n(x)>\ell \qquad && \overline\nabla [\widehat{\varphi}_n{-} \varphi] (x,y)   &&   \hspace{-0,3cm} = 
%%%-2\ell -  \varphi_n(x)  
%%% + \varphi(x) -  \varphi(y)  
%%%\\ & && 
%%% &&  \hspace{-0,3cm} < \varphi_n(y) - \varphi_n(x) + \varphi(x) -  \varphi(y)  =  \overline\nabla [\varphi_n{-} \varphi] (x,y) \,.
%%%\end{aligned}
%%%\]
%%%{\color{red}....... how to proceed??}
%%%
%%%Finally, in the cases. $[ \varphi_n(y)>\ell, \,  \varphi_n(x)>\ell]$ and  $[ \varphi_n(y)<-\ell, \,  \varphi_n(x)<-\ell]$ we have 
%%%\[
%%%\left|  \overline\nabla [\widehat{\varphi}_n{-} \varphi] (x,y)\right| = | \varphi] (x,y)|
%%%\]
%%%and then we can use the dominated convergence theorem.....
\end{proof}




\EEE

\section{Applications}

\TODO{maybe this section could be moved right BEFORE the sections with the proof...}


\appendix 


\section{Proof of Lemma \ref{l:crucial-F}}
\label{s:a.1}
\RNEW
\noindent For statements \textbf{(1)} \& \textbf{(2)} we refer to the proof of the analogous items in  \cite[Lemma 2.3]{PRST22}. 
We only address here \EEE the proof of  \textbf{(3)}, by adapting the argument carried out for $V=\R^d$ 
in the proof of \cite[Thm.\ 2.34]{AmFuPa05FBVF}.
Exploiting \cite[Prop.\ 2.31, Lemma 2.31]{AmFuPa05FBVF} we  sequences $(a_j)_j \subset \R^m$ and $(b_j)_j \subset \R$ such that, setting $L_j (x) :=
\pairing{}{}{a_j}x +b_j$ for all $x\in \R^m$, we have the representation formulae
\begin{equation}
\label{representation-AFP}
\uppsi(x) = \sup_{j\in \N} L_j(x), \qquad \uppsi^\infty(x) = \sup_{j\in \N}\pairing{}{}{a_j}x \qquad \text{for all } x\in \R^m.
\end{equation}
Let us now consider sequences $(\mu_n)_n\, \mu \in \Mloc(Y;\R^m)$ 
and $(\nu_n)_n,\, \nu \in  \Mloc^+(Y)$ such that 
$\mu_n \to \mu$ and $\nu_n \to \nu$ vaguely and 
$ \liminf_{n\to\infty}  \calF_\uppsi(\mu_n|\nu_n)<\infty$;
  let 
$g_n: = \frac{\dd \mu_n}{\dd \nu_n}$, $g = \frac{\dd \mu}{\dd \nu} $, $h_n = \frac{\dd \mu_n^\perp}{\dd |\mu_n^\perp|} $
and $h: =\frac{\dd \mu^\perp}{\dd |\mu^\perp|}$.  
Furthermore, let $(A_j)_{j=0}^k$, $k\in \N$, be pairwise disjoint open subsets of $V$ with compact closure. For any family  of functions
$(\phi_j)_{j=0}^k$ such that $\phi_j \in \mathrm{C}_{\mathrm{c}}^1(A_j;\R)$ and $0\leq \phi_j \leq 1$ for all $j=\{0,\ldots, k\}$ there holds
\[
\int_{A_j} b_j \phi_j \, \dd \nu_n + \left\langle a_j, \int_{A_j} \phi_j \, \dd \nu_n  \right \rangle 
\leq \int_{A_j} \uppsi(g_n) \, \dd \nu_n + \int_{A_j} \uppsi^\infty(h_n) \, \dd |\mu_n^\perp| 
\]
so that, since $\mu_n \to \mu$ and $\nu_n \to \nu$ vaguely, we have  
\[
\sum_{j=0}^k \int_{A_j} b_j \phi_j \, \dd \nu+ \left\langle a_j, \int_{A_j} \phi_j \, \dd \nu \right \rangle 
\leq \liminf_{n\to\infty} \sum_{j=0}^k\int_{A_j} b_j \phi_j \, \dd \nu_n + \left\langle a_j, \int_{A_j} \phi_j \, \dd \nu_n  \right \rangle
\leq \liminf_{n\to\infty}  \calF_\uppsi(\mu_n|\nu_n)\,.
\]
We now introduce the functions $f_j : \R \to \R$ defined by 
\[
f_j (v) : = \begin{cases}
L_j(g(v)) & \text{if } v \in V\setminus N,
\\
\pairing{}{}{a_j}{h(v)} & \text{if } v \in N,
\end{cases}
\text{ with $N$ a $\nu$-negligible set where $|\mu^\perp|$ is concentrated,}
\]
and rewrite the above estimate as 
\begin{equation}
\label{estimate-AFP}
\sum_{j=0}^k \int_{A_j} f_j \phi_j \, \dd \eta \leq 
 \liminf_{n\to\infty}  \calF_\uppsi(\mu_n|\nu_n) \qquad \text{with } \eta: = \nu+|\mu^\perp\,.
\end{equation}
Now, let us set $D_j : = \{ v\in A_j\, : \ f_j(v)>0\}$ and let $\mathbf{1}_{D_j}$ be its characteristic function. It follows from \eqref{estimate-AFP} that 
$\eta(D_j)<\infty$ for all $j=0,\ldots, k$. Now, 
by
the inner regularity property 
\eqref{inner-regularity}, for every $j \in \{0,\ldots, k\}$ we find   compact sets $K_j \subset D_j \subset A_j$ such that  $\eta(D_j) \setminus \eta(K_j) \leq \frac1j $. Let now 
$O_j$ be open sets such that $K_j \subset  O_j \subset A_j$, and let us pick functions $\overline{\phi}_j  \in  \mathrm{C}_{\mathrm{c}}^1(A_j;\R)$ such that 
$0 \leq \overline{\phi}_j  \leq 1$ and 
\[
\overline{\phi}_j(v) \begin{cases}
\equiv 1  & \text{if } v \in K_j,
\\
\equiv 0 & \text{if } v \in A_j \setminus O_j\,.
\end{cases}
\]
From \eqref{estimate-AFP} we deduce 
\begin{equation}
\label{almost-final-step}
\sum_{j=0}^k \int_{A_j} (f_j)^+ \, \dd \eta  = \sum_{j=0}^k \int_{A_j} f_j \, \sup_{j \in \N} \overline{\phi}_j  \, \dd \eta \leq  \liminf_{n\to\infty}  \calF_\uppsi(\mu_n|\nu_n) \,.
\end{equation}
Let now
\[
f(v): =  \begin{cases}
\uppsi(g(v)) & \text{if } v\in V\setminus N,
\\
\uppsi_\infty(h(v)) & \text{if } v \in N,
\end{cases}
\]
and observe  that $f  =\sup_{j\in \N}   f_j  $ by \ref{representation-AFP} and hence $f = \sup_{j\in \N} f_j^+$. Since the family $(A_j)_{j=0}^k$ is arbitrary in the family 
$\mathcal{A}$ of pairwise disjoint open subsets of $V$ with compact closure
of all 
from \eqref{almost-final-step} we deduce  that 
\[
\calF_\uppsi(\mu|\nu) = \int_{V} f \, \dd \eta \stackrel{(1)}= \sup_{(A_j)_{j=0}^k \in \mathcal{A}} \sum_{j=0}^k \int_{A_j} (f_j)^+ \, \dd \eta \stackrel{(2)}\leq  \liminf_{n\to\infty}  \calF_\uppsi(\mu_n|\nu_n) 
\]
with {\footnotesize (1)} from \cite[Lemma 2.35]{AmFuPa05FBVF} and  {\footnotesize (2)}  due to \eqref{almost-final-step}. This concludes the proof. 
\QED

{\small

\markboth{References}{References}




\bibliographystyle{alpha}

%\bibliographystyle{alpha_AMs}
\bibliography{ricky_lit}
}



\end{document}




